\chapter{Flux compactifications and (exceptional) generalised
geometry\label{chap:FluxCompactifications}}

In this chapter we study (exceptional) generalised geometry. These are geometries based on double geometries and exceptional geometries with a global solution to the section constraint. Especially this implies fields are acted on naturally by the relevant duality group and are patched together by the geometric subgroup of the duality group. Importantly this allows for the presence of a non-trivial form field as a part of the geometry. Especially geometrical tools used for fluxless compactifications can be generalised and applied also in the presence of fluxes in the generalised geometric formulation. This merger of gravitational degrees of freedom and non-gravitiational degrees of freedom is behind the effectivness of generalised geometry in the study of flux compactifications. 

In the first two sections we discuss shortly some phenomenology of compactification as well as introducing geometrical tools to study supersymmetric compactifications without fluxes. In section we start consider supersymmetric backgrounds with fluxes and introduce some basics of (exceptional) generalised geometry. The generalised geometrical tools are then used in \ref{sec:GeneralisedStructures} to re-express the conditions of supersymmetric flux compactifications as integrability conditions of generalised structures. Since this topic by itself could be a whole thesis we will not be able to give all the details, for more on compactifications in general as well as Calabi-Yau compatictifications we refer to \cite{Blumenhagen2013,Grana:2005jc}. For the formulation of generalised geometry we refer to \cite{Hull:2007zu,Coimbra:2011ky,Coimbra:2011nw} and to \cite{Ashmore:2015joa,Grana:2009im} for its application to study flux compactifications. 

\section{Phenomenology and moduli stabilisation}
As is well-known string theory/M-theory lives in a 10/11-dimensional spacetime, yet our everyday experience tells us otherwise. Hence, in order for string theory/M-theory to be able to describe the real world it has to have a consistent low-energy description that matches our observations. Firstly such a theory should reproduce the standard model of particle physics which has been confirmed to high precision, i.e.\ among other things it should be described by a gauge theory with gauge group $SU(3)\times SU(2)\times U(1)$ together with the correct particle spectra. Another important notion is that of supersymmetry, notably the standard model is not supersymmetric in contrast to string theory/M-theory. The latter two are maximally supersymmetric theories and thus most, perhaps all, supersymmetry needs to be spontaneously broken. %Generically supersymmetry constrains a theory often yielding important physical consequences.

The idea of compactification, where a compact internal manifold is ``small'' compared to the typical size of say strings, is one way to obtain an effective theory in lower dimensions. A generic feature of such compactification is the presence of scalar fields typically due internal legs of some tensor field. Of especial importance are massless such scalar fields, called \emph{moduli}. The vacuum expection values of moduli are not fixed and appear as free parameters in the theory. For a given compactification ansatz there are possibly hundreds or thousands of such moduli which determine physical properties in the external theory, e.g.\ gauge couplings, Yukawa couplings as well as the size and structure of the internal manifold. However, if all these properties are free parameters the theory loses most of its predictive power. Moreover, massless scalar fields would give rise to a long-ranged force which would invalidate Einstein's equivalence principle. Of course the coupling could be extremely small such that these forces are not experimentally observable. However, this approach comes with a lot of problems and instead one tries to generate mass to these scalar fields, this is called \emph{moduli stabilisation}. Generating a potential for the scalar fields would imply that the moduli fields are no longer free parameters but rather fixed by the procedure that stabilises the moduli, e.g.\ a complicated internal geometry or presence of gauge fluxes. 

The simplest internal manifold to compactify on is a torus but such a compactification preserves all the supersymmetry, i.e.\ $\mathcal{N}=8$ in four external dimensions. In order to break some supersymmetry we could abandon the torus and compactify on something more complicated, e.g.\ a Calabi-Yau three-fold which we will discuss further below. Another possibility is to introduce a non-zero flux for some $(p+1)$-form field 
\begin{equation}
    \int_{\Sigma_{p+1}}C_{p+1}\neq 0,
\end{equation}
corresponding to a $p$-brane winding some non-trivial cycle $\Sigma_{p+1}$ in the internal manifold. Such a configuration schematically generates a potential for the moduli through the kinetic term of such a form field 
\begin{equation}
    V\sim \int_{\mathcal{M}_{\text{int}}} \dd C_{p+1}\wedge \star_{g}\,\dd C_{p+1}, 
\end{equation}
where $\mathcal{M}_\text{int}$ denotes the internal manifold and the presence of the metric moduli in the hodge dual $\star$ is the reason for a potential being generated by expanding $C_{p+1}$ around its vacuum expectation value. This is one mechanism for generating mass to a modulus (which by definition no longer is a modulus). 

Phenomenology of flux compactifications could fill out a thesis of its own and for this reason we refer the reader to the literature for further discussion about this. See e.g.\ \cite{Grana:2005jc} and references therein.

\section{Zero flux compactification\label{sec:VanishingFluxCompactification}}
To begin with we will start by reviewing compactifications on backgrounds without fluxes. The idea is to find vacua (solutions to the equations of motion and the Bianchi identities) such that spacetime seperates into an external spacetime $\mathcal{M}_{\text{ext}}$ and an internal spacetime $\mathcal{M}_{\text{int}}$ such that 
\begin{equation}
    \mathcal{M} = \mathcal{M}_{\text{ext}}\times \mathcal{M}_{\text{int}}.
\end{equation}
However, finding such solutions by solving the equations of motion are generally rather difficult and we will take a different route. Supergravity in $D=11$ has $\mathcal{N}=1$ local supersymmetry, this is the maximal amount of supersymmetry with a total of $32$ supercharges and a generic vacuum solution will (spontaneuously) break all of this supersymmetry. By restricting to solutions that preserve a certain amount of supersymmetry, say $\mathcal{N}=1$ in $D=4$, we will find constraints that give a more tractable way to finding vacua. Moreover, we will assume that the external spacetime is Minkowski with possible generalisations, especially to AdS. 

Supersymmetry is generated by a supercharge $\mathcal{Q}$ and a parameter $\epsilon$, both being spinors of $SO(1,D-1)$. A vacuum $|0\rangle$ is supersymmetric if it is annihilated by the supersymmetry transformation
\begin{equation}
    \overbar{\epsilon}\mathcal{Q}|0\rangle = 0.
\end{equation}
By taking an arbitrary field $\Theta$ and looking at its variation under the supersymmetry transformation we find 
\begin{equation}
    \langle \delta_{\epsilon}\Theta\rangle = \langle 0|\left[\overbar{\epsilon}\mathcal{Q},\Theta\right]|0\rangle = 0.
\end{equation}
Schematically we have (from now variations should be understood implicitly to be valid as expectation values)
\begin{equation}
    \delta_\epsilon \Theta_{\text{boson}} \sim \overbar{\epsilon}\Theta_{\text{fermion}},\qquad \delta_\epsilon \Theta_{\text{fermion}}\sim \epsilon\Theta_{\text{boson}}.
\end{equation}
By the assumption that $\text{dim}\,\mathcal{M}{_{\text{ext}}}=d$ being a $d$-dimensional Minkowski space we have to decompose the spinor representations under $SO(1,d-1)\times SO(D-d)$ where $D$ is the dimension of the full spacetime. Typically these will transform in a non-trivial representation under $SO(1,d-1)$ and hence have to vanish in order to preserve Poincaré symmetry of the vacuum. Thus in order to make sure that the vacuum preserve some amount of supersymmetry we need to find non-trivial solutions of $\delta_\epsilon\Theta_{\text{fermions}}=0$. Note that if such a solution is found we also have to make sure the equations of motion and Bianchi identities are fulfilled. 

With a constant dilaton and vanishing fluxes the supersymmetry variation of the gravitino $\psi_M$ (which is always present in supergravity) \cite{Blumenhagen2013}
\begin{equation}
    \delta_\epsilon\psi_M = \nabla_M\epsilon = 0 \Leftrightarrow \nabla_\mu \epsilon = 0,\qquad \nabla_m\epsilon = 0,
\end{equation}
where we have used that there is no mixing between greek external indices and latin internal indices in the connection due to the metric being of a product form. A spinor satisfying these conditions is called a \emph{Killing spinor}. From this one can easily show by taking a commutator of covariant derivatives that a necessary condition for the existence of a covariantly constant spinor is that the internal manifold is Ricci-flat (obviously also the external Minkowski space is Ricci-flat)
\begin{equation}
    R_{mn} = 0.
\end{equation}

Curvature of a manifold is closely connected to the way an object transforms upon parallel transport. As such take some object $v$ and parallel transport it around a closed loop on the manifold. Generally the object transforms in some representation of the holonomy group $\mathscr{H}\subseteq O(d)$ under parallel transport for a $d$-dimensional riemannian manifold, since the norm does not change assuming a metric compatible connection. Likewise a spinor parallel transported in a loop transforms in a representation of $\mathscr{H}$ as well and hence, especially, also the Killing spinor. But since the Killing spinor is covariantly constant it does not transform when parallel transported and needs to be a singlet under the holonomy group. As such a necessary condition for the existence of a covariantly constant spinor is related to the restriction of the holonomy group which put constraints on the metric. Such features as reduction of holonomy and structure group are conveniently described in terms of bundles briefly explained below.

\subsection{Mathematical interlude -- Fibre bundles}
A short introduction to the notion of fibre bundles are introduced in order to set the terminology which will be convenient for more complicated compactifications later on as well as when discussing generalised geometry. Just as a manifold is a topological space that looks locally like $\mathbb{R}^n$, i.e.\ there exists some coordinate chart on $U_i\subseteq \mathcal{M}$ such that $\phi: \mathcal{M}\to \mathbb{R}^n$, a bundle is a topological space that ``looks locally'' like a product of two topological spaces. Moreover, in the cases we are interested in these topological spaces will furthermore be differentiable manifolds. 

A bundle is a triple $(P,\mathcal{M},\pi)$ typically also denoted $P\overset{\pi}{\longrightarrow}\mathcal{M}$, where $P$ is the total space, $\mathcal{M}$ the base space and $\pi: P\to \mathcal{M}$ the projection, which is a surjection onto the base space. The notion of a bundle is a rather general concept and we will instead only be interested in (differentiable) fibre bundles. A (differentiable) fibre bundle $(P,\mathcal{M},\pi,F,G)$ is defined by a bundle $(P,\mathcal{M},\pi)$, a differentiable manifold $F$ called the fibre and a Lie group $G$ called the structure group. In order for a fibre bundle to ``look locally'' like a product manifold there exists patches on $U_i\subseteq \mathcal{M}$ diffemorphisms and linear maps $\phi_i: U_i\times F\to \pi^{-1}(U_i)$ such that $\pi\circ \phi_i(u,f)=u$, where $u\in U_i$, $f\in F$ and $\pi^{-1}(U_i)=F_p\cong F$ is the fibre at $u\in U_i$. The maps $\phi_i$ are usually called local trivialisations since the map $\phi^{-1}\circ \pi^{-1}$ maps any $U_i$ to $U_i\times F$ which ``looks like a product manifold''. In order for this to make sense there needs to be some compatibility on overlapping patches $U_i\cap U_j\neq \emptyset$. Take a point $p\in P$ such that $\pi(p)=u$ with $u\in U_i\cap U_j$. Consider the mapping $\phi_{u,i}$
\begin{equation}
    \begin{aligned}
        \phi_{u,i}: F&\to F_u,\\
            f&\mapsto f_u = \phi_i(u,f),
    \end{aligned}
\end{equation}
which assigns to each point in $F$ an element in the fibre $F_u$ at $u$. On the overlap we can look at 
\begin{equation}
    \phi_{j}(u,f) = \phi_{i}(u,t_{ij}(f)),
\end{equation}
where $t_{ij}:f\to f$ defined on $U_i\cap U_j$ are called transition functions. Composing with $\phi_i^{-1}$ we find 
\begin{equation}
    \phi_{i}^{-1}\circ\phi_j (u,f) = (u,\phi^{-1}_{i,u}\circ \phi_{j,u}f),
\end{equation}
and the by the definition of the transition functions we find $t_{ij,u}=\phi^{-1}_{i,u}\circ \phi_{j,u}$. The compatibility requirement is then to demand that $t_{ij}: U_i\cap U_j\to R(G)$, with $R(G)$ some representation of $G$, such that fibres are patched together by the structure group $G$. This moreover defines a left action of $G$ on the bundle.

A section of a bundle $P\overset{\pi}{\longrightarrow}\mathcal{M}$ is defined as a map $\sigma: \mathcal{M}\to P$ such that $\pi\circ\sigma = \text{id}_{\mathcal{M}}$. A well known example is a section of the tangent bundle $T\mathcal{M}$ which is a bundle defined as
\begin{equation}
    TM = \bigcup_{x\in \mathcal{M}}\{(x,y)|x\in\mathcal{M}, y\in T_x\mathcal{M} \},
\end{equation}
with $T_x\mathcal{M}$ being the tangent space at point $x\in\mathcal{M}$. A section then associates a vector in the fibre $T_x\mathcal{M}$ to each point $x\in\mathcal{M}$
\begin{equation}
    \begin{aligned}
        \sigma: &\mathcal{M}\to \mathcal{M}\times T_x\mathcal{M}\\
                x&\mapsto (x,X)\qquad X\in T_x\mathcal{M},
    \end{aligned}
\end{equation}
which is just an ordinary vector field on $\mathcal{M}$. Moreove, the space of sections of a bundle $P$ is often denoted $\gamma(P)$. 

%then applying the two local trivialisations $\phi^{-1}_i$ and $\phi^{-1}_j$ we find two points in $P$ according to 
%\begin{equation}
%    \phi^{-1}_i(u) = (u,f_i)\qquad \text{and}\qquad \phi^{-1}_j(u) = (u,f_j).
%\end{equation}
%Obviously $(u,f_i) = \phi^{-1}_i\circ \phi_j (u,f_j)$ where $\phi^{-1}_i\circ \phi_j = t_{ij}(u)$ is called a transition function $t_{ij}(u):F\to F$ defined trough a left action of the structure group $G$.\todo{left action of G} 

An important class of fibre bundles are principal fibre bundles. A principal fibre bundle ($P$,$\mathcal{M}$,$\pi$,$G$) is a fibre bundle with a typical fibre being the structure group $G$ itself, i.e.\ for some $u\in U\subseteq \mathcal{M}$ we have $\pi^{-1}(u)\cong G$. Moreover, there is a right action of $G$ on $P$ as 
\begin{equation}
    \triangleleft: P\times G \to P,
\end{equation}
which is free. An important example of a principle fibre bundle is the frame bundle. The frame bundle is a fibre bundle with a typical fibre consisting of all ordered basis of the tangent space. At any patch $U_i$ on a $d$-dimensional manifold $\mathcal{M}$ there exist local coordinates which spans a basis $\{\pd_\mu\}_{\mu=1,2,\ldots d}$ and any other base can be written as 
\begin{equation}
    e_\alpha = \tensor{e}{_\alpha^\mu}\pd_\mu,
\end{equation}
with $\alpha=1,2,\ldots d$. The elements $\tensor{e}{_\alpha^\mu}$ can be seen as $d\times d$-dimensional matrices and thus the fibre is isomorphic to $GL(d)$. The right action is then given by 
\begin{equation}
    e_\alpha \mapsto e_\beta \tensor{\Lambda}{^\beta_a},
\end{equation}
for some $\Lambda\in GL(d)$. 

Importantly we also want to consider tensor fields on a manifold $\mathcal{M}$ which is formally introduced by looking at associated vector bundles to a principle fibre bundle. Suppose that $(P,\pi,\mathcal{M},G)$ is a principle fibre bundle and $(\rho,\mathbb{V})$ a $G$-representation. Then an associated vector bundle $P_{\mathbb{V}}$ is given by 
\begin{equation}
    P_{\mathbb{V}} = (P\times \mathbb{V})/\sim_G,
\end{equation}
with $\sim_G$ defined by the natural action of $G$ on both $P$ and $\mathbb{V}$. By construction elements of the associated vector bundle is thus invariant under the action of the structure group $G$ and the action of $G$ on the first factor $P$ can be considered as the inverse action of $G$ on $\mathbb{V}$. As an examaple consider an element $v=v^\alpha e_\alpha$ in tangent bundle $v\in T\mathcal{M}$ which is an associated vector bundle on which an element $\Lambda\in G$ acts as 
\begin{equation}
    v \mapsto \tensor{\rho(\Lambda)}{^\alpha_\beta} v^{\beta}e_\gamma \tensor{\Lambda}{^\gamma_\alpha},
\end{equation}
which from invariance of $v$ we find that $\tensor{\rho(\Lambda)}{^\alpha_\beta}= \tensor{(\Lambda)}{^{-1}}\tensor{}{^{\alpha}_\beta}$. 

What we mainly are interested in is the possibility of finding so-called $H$-structures for some some closed subgroup $H\subset G$. A principal bundle $(P,\pi,\mathcal{M},G)$ has a $H$-structure if it is possible to choose frames in $P$ such that the structure group reduces to $H$. A typical example of this is an $O(d)\subset GL(d)$ structure on Riemannian manifolds. Due to the presence of a metric we can consider orthonormal frames $\tilde{e}_a$ such that 
\begin{equation}
    g(\tilde{e}_a,\tilde{e}_b) = \delta_{ab},
\end{equation}
with $g$ being the metric. This defines an $O(d)$ structure since on overlapping patches orthonormal frames are related by $O(d)$ transformations. The importance of this in what will follow is the fact that finding a reduced structure is equivalent to the presence of a globally defined non-vanishing tensor on $\mathcal{M}$ which is invariant under some subgroup $H\subset G$. Since it is globally defined we can choose a frame $\{e_{\alpha}\}_{a=1,2,\ldots d}$ such that the tensor is constant on overlapping patches which effectively reduce $GL(d)$ to $H$. 


\begin{comment}
The existence of invariant tensors on $\mathcal{M}$ is equivalent to adding further structure to a principal $G$-bundle such that it is possible to impose a restriction to a principal $H$-bundle, where $H\subset G$ with the same base manifold. Typically this is defined the other way around,  


In order to define a reduced structure group we need a so-called bundle map. A bundle map is a linear map $\rho: P\to \overbar{P}$ and continuous map between the base manifolds $f: \mathcal{M}\to \overbar{\mathcal{M}}$ such that the following diagram commutes
\begin{center}
\begin{tikzpicture}
  \matrix (m) [matrix of math nodes,row sep=3em,column sep=4em,minimum width=2em]
  {
     P & \overbar{P} \\
     \mathcal{M} & \overbar{\mathcal{M}} \\};
  \path[-stealth]
    (m-1-1) edge node [left] {$\pi$} (m-2-1)
            edge [] node [above] {$\rho$} (m-1-2)
    (m-2-1.east|-m-2-2) edge node [below] {$f$}
            node [above] {} (m-2-2)
    (m-1-2) edge node [right] {$\overbar{\pi}$} (m-2-2);
\end{tikzpicture}
\end{center}
Moreover, if $\rho$ and $f$ are invertible and both $(\rho,f)$ and $(\rho^{-1},f^{-1})$ are bundle maps then $(P,\mathcal{M})$ and $(\overbar{P},\overbar{\mathcal{M}})$ are isomorphic as bundles. 

\todo{Bundle maps, when reduced structure group, Berger's list}
\todo{Reduced structure group}
\end{comment}
\subsection{Compactification on Calabi-Yau}\label{sec:CalabiYau}
Having reviewed the notion of reduced structure group and special holonomy a Calabi-Yau $n$-fold $CY_{n}$ can be defined as a $2n$-dimensional compact Riemannian manifolds with holonomy contained in $SU(n)$\cite{Blumenhagen2013}. We will be particularly interested in $CY_3$ such that starting with $D=10$ supergravity, or string theory, the external spacetime is four-dimensional. As such we know that for a $CY_3$ there exists some $p$-forms, or equivalently a Killing spinor, invariant under the holonomy group $\mathscr{H}=SU(3)$ which can be found by looking at the decomposition of one-, two- and three-forms of $Spin(6)\cong SU(4)\to SU(3)$
\begin{align*}
    \mathbf{6}&\to \mathbf{3}+\mathbf{\overbar{3}},\\
    \mathbf{15}&\to \mathbf{8}+\mathbf{3}+\overbar{\mathbf{3}}+\mathbf{1},\\
    \mathbf{20}&\to \mathbf{6}+\overbar{\mathbf{6}}+\mathbf{3}+\overbar{\mathbf{3}}+2\cdot\mathbf{1}.
\end{align*}
It thus exist a one-form $\omega$ and a two three-forms (these actually combine into a complex three-form) $\Omega$ that are singlets under $\mathscr{H}$, such pair of nowhere vanishing tensors defines a \emph{structure}. Note that we already know that such a structure is equally well defined by a nowhere vanishing globally defined spinor $\mathbf{4}$ of $SU(4)$ which decomposes under $\mathscr{H}$ as 
\begin{equation}
    \mathbf{4}\to \mathbf{3}+\mathbf{1}.
\end{equation}
Explicitly, a Weyl-spinor $\psi$ of $Spin(1,9)$ decomposed under $Spin(1,3)\times Spin(6)$ can be parameterised as 
\begin{equation}
    \psi = \eta_+\otimes\epsilon_++\eta_-\otimes\epsilon_-,
\end{equation}
where $\eta_\pm$ and $\epsilon_\pm$ are Weyl-spinors with chirality $\pm 1$ under $SO(1,3)$ and $SO(6)$ respectively. Morever, imposing that $\psi$ is a Majorana-Weyl spinor in ten dimensions further implies that $\eta_+$ is the charge conjugate of $\eta_-$ and likewise for $\epsilon_\pm$.

What follows will mostly consider structures defined by some bosonic forms such as ($\omega,\Omega$) but from the observation above we know that this corresponds to the existence of a globally defined nowhere vanishing spinor. To preserve supersymmetry such a spinor also satifies a differential condition and hence also the structure $(\omega,\Omega)$ needs to satisfy certain differential conditions. In fact, the structure $(\omega,\Omega)$ is related to the Killing spinor as
\begin{align}
    \omega_{mn} = \mp\ii \epsilon^\dagger_\pm\gamma_{mn}\epsilon_\pm,\\
    \Omega_{mnp} = -\ii\epsilon^\dagger_-\gamma_{mnp}\epsilon_+.
\end{align}
Moreover, in order for the structures to be compatible they should satisfy the compatibility conditions 
\begin{equation}\label{eq:CompatibilityCalabiYau}
    \omega\wedge\Omega = 0\qquad \omega\wedge\omega\wedge\omega = \frac{3\ii}{4}\Omega\wedge\overbar{\Omega}.
\end{equation}
Taking the covariant derivative of these definitions we find that solving the Killing spinor equations in terms of the structure $(\omega,\Omega)$ is simply the statement that they are both closed forms 
\begin{equation}
    \dd \omega = 0, \qquad \dd \Omega = 0.
\end{equation}
%This condition is an integrability condition and is necessary to obtain a consistent patching of the structure. %Moreover, we see that starting with $\mathcal{N}$ supersymmetries in the full theory compactification on a $CY_3$ yields a theory which again has $\mathcal{N}$ supersymmetries.
%Note that transformations that preserves the non-degenerate two-for in $d-$dimensions are by definition elements in $Sp(d/2,\mathbb{R})$. 

\section{Flux compactifications and generalised geometry}
In the case of vanishing flux we have seen that the Killing spinor equation led to special holonomy, which in turn put restrictions on the connection of the internal manifold. In the presence of fluxes the Killing spinor is no longer covariantly constant, however, the existence of globally defined nowhere vanishing spinor still implies a reduced structure group. This is summarised by 

\begin{equation*}
    \begin{aligned}
    \text{vanishing flux}&\Longleftrightarrow \text{special holonomy},\\
    \text{non-zero flux}&\Longleftrightarrow \text{reduced structure group}.\\
    \end{aligned}
\end{equation*}
In the presence of fluxes the Killing spinor equations of type II supergravity are given by \cite{Blumenhagen2013}
\begin{equation}\label{eq:KillingSpinorWithFluxGravitino}
\begin{aligned}
    \delta_\epsilon \psi_M &= \left(\nabla_M+\frac{1}{8}\Gamma^{NP}H_{MNP}\mathscr{P}+\frac{1}{16}\ee^\Phi\sum_{n}\slashed{F}_n\Gamma_M\mathscr{P}_n\right)\epsilon=0\\,
    \delta_\epsilon \lambda &= \left(\slashed{\pd}\Phi+\frac{1}{12}\Gamma^{MNP}H_{MNP}\mathscr{P}+\frac{1}{8}\ee^\Phi\sum_{n}(-1)^n(5-n)\slashed{F}_n\mathscr{P}_n\right)\epsilon=0,
    \end{aligned}
\end{equation}
where $\mathscr{P}$ and $\mathscr{P_n}$ are $2\times 2$-matrices, $\slashed{F}_n=\frac{1}{n!}\Gamma^{M_1M_2\ldots M_n}F_{M_1M_2\ldots M_n}$ and $n\in\{0,1,\ldots,10\}$ are even(odd) for type IIA(B). In the case of vanishing flux and a constant dilation this reduces to a covariantly constant spinor $\epsilon$ as discussed above. Presence of gauge fluxes is according to \eqref{eq:KillingSpinorWithFluxGravitino} obstructions to the Killing spinor being covariantly constant with respect to the Levi-Civita connection. 

The obstruction to the Killing spinor being covariantly constant is conveniently analysed in terms of \emph{intrinsic torsion}. It is always possible to construct a connection $\nabla$ with torsion such that the Killing spinor is covariantly constant w.r.t.\ to that connection. Thus consider a connection $\nabla=\nabla'-\kappa$ such that $\nabla_n V_m=\pd_nV_m-\tensor{\Gamma}{^'_{nm}^p}V_p-\tensor{\kappa}{_{nm}^p}V_p$, where $\nabla'$ is the Levi-Civita connection and $\kappa$ is the so-called contorsion tensor. Generically the connection coefficents $\tensor{\Gamma}{_{mn}^p}$ take values in $\Lambda^1\otimes\mathfrak{g}$, where $\Lambda^1$ is the one-form representation of $G=GL(d)$ and $\mathfrak{g}=\mathfrak{gl}(d)$ is the corresponding Lie algebra. Assuming metric compatibility fixes the part in $\mathfrak{gl}(d)/\mathfrak{so}(d)$ as we saw in section \ref{sec:CovariantFormalism} of the pair $\tensor{}{_m^p}$. This is contained in the Levi-connection but the Levi-Civita connection contain further terms that ensures that the torsion $\tensor{\Gamma}{_{[mn]}^p}$ vanishes. This implies that the contorsion take values in 
\begin{equation}
    \tensor{\kappa}{_{mn}^p}\in \Lambda_1\otimes \mathfrak{h},
\end{equation}
where $\mathfrak{h}=\mathfrak{so}(d)$ is the maximal compact subalgebra of $\mathfrak{gl}(d)$. Assuming a reduced structure group $G'\subseteq SO(d)$ implies that we can separate the contorsion into two parts 
\begin{equation}
    \kappa = \kappa_\text{int}+\kappa_{0}\qquad \text{where} \quad \kappa_{0}\in \Lambda_1\otimes \mathfrak{g}'\quad \text{and}\quad \kappa_{\text{int}}\in \Lambda_1\otimes \tensor{\mathfrak{g}}{^'^{\perp}}.
\end{equation}
The point now being that since the Killing spinor is a singlet under $\mathfrak{g}'$ the failure of the of Killing spinor to be covariantly constant \emph{with respect to the Levi-Civita connection} is given by $\kappa_{\text{int}}$. The part $\kappa_{\text{int}}$ can further be decomposed into irreducible representations of $\mathfrak{g}'$ called torsion classes. As an example take $G=GL(6)$ and $G'=SU(3)$. Under $G'$ we have the following decompositions $\Lambda_1\to \mathbf{3}\oplus\overbar{\mathbf{3}}$, $\mathfrak{g}'=\mathbf{8}$ and $\tensor{\mathfrak{g}}{^'^{\perp}}=\mathbf{1}\oplus\mathbf{3}\oplus\overbar{\mathbf{3}}$ and 
\begin{equation}
    \kappa_{\text{int}} \in \bigoplus_{i=1}^5 \mathscr{W}_i,
\end{equation}
where $\mathscr{W}_i$ are the five torsion classes given by 
\begin{align*}
    \mathscr{W}_1 = \mathbf{1}\oplus\mathbf{1},\\
    \mathscr{W}_2 = \mathbf{8}\oplus\mathbf{8},\\
    \mathscr{W}_3 = \mathbf{6}\oplus\overbar{\mathbf{6}},\\
    \mathscr{W}_4 = \mathbf{3}\oplus\overbar{\mathbf{3}},\\
    \mathscr{W}_5 = \mathbf{3}\oplus\overbar{\mathbf{3}}.\\
\end{align*}
Using complex geometry and cohomology one can show, see \cite{Blumenhagen2013} for details, that the exterior derivative of the structure $(\omega,\Omega)$ can be written in terms of torsion classes as 
\begin{equation}
    \dd \omega = \frac{3}{2}\text{Im}(\overbar{\mathscr{W}}_1\Omega)+\mathscr{W}_4\wedge \omega+\mathscr{W}_3
\end{equation}
and 
\begin{equation}
    \dd \Omega = \mathscr{W}_1\omega\wedge\omega+\mathscr{W}_2\wedge\omega+\overbar{\mathscr{W}}_5\wedge\Omega.
\end{equation}
We thus see that all five torsion classes vanish for $CY_3$ in order for $(\omega,\Omega)$ to be closed. Moreover, we have seen explicitly that the presence of intrinsic torsion corresponds to compacitifications with fluxes which in turn is an obstruction to finding a covariantly constant Killing spinor.


\subsection{Generalised geometry}
We have seen above that demanding some preserved supersymmetry in fluxless compactification led to two constraints:
\begin{itemize}
    \item Demanding existence of globally defined spinor led to a reduced structure group characterised by ($\omega,\Omega$)
    \item Imposing the Killing spinor equation led to a reduced holonomy. 
\end{itemize}
Including fluxes alters the Killing spinor equation and the holonomy group need no longer be reduced. This effectively weakens the constraints of the internal manifold which at the same time make things more complicated. In the spirit of extended geometries we will try to generalise the notion of geometry in order to accommodate flux degree of freedoms in terms of what is called (exceptional) \emph{generalised geometry}. The reason for doing this being that it is then possible to define an appropriate generalisation of the holonomy, or rather of the structure together with some integrability conditions. This will imply that a supersymmetric background compactification should have vanishing generalised intrinsic torsion. Especially since gauge degrees of freedom are included in the geometry this holds true even when non-trivial fluxes are present. Below we will briefly introduce (exceptional) generalised geometry needed for discussing flux compactifications and generalised structures, for a complete introduction to this field we refer to the literature \cite{Hull:2007zu,Coimbra:2011nw,Coimbra:2011ky,Ntokos:2016vnm}.

In extended geometries we enlarged spacetime by introducing coordinates in some module of the underlying structure group $G$ and a section constraint specifying a $GL(d)\hookrightarrow G$ embedding. Generalised geometry can be considered as an extended geometry with a global solution to the section constraint and a structure group given by the ``geometric'' subgroup of $G$ as we will see.  

Therefore consider a manifold $\mathcal{M}$ and its associated tangent bundle $T\mathcal{M}$. A basis of the tangent space in local coordinates is simply given by $\{\pd/\pd x^i\}_{i=1,\ldots d}$, where $d=\text{dim}\,\mathcal{M}$. Now recall that in DFT spacetime is enlarged to $2d$ dimensions by letting $x^i\to (x^i,\tilde{x}_i)$, where the latter is the vector representation of $O(d,d)$. In generalised geometry one instead considers only a generalised tangent bundle on which in the DFT case $O(d,d)\times \mathbb{R}^+$ acts naturally and in the exceptional case $E_{d(d)}\times\mathbb{R}^+$ acts naturally.  In the former case the generalised tangent bundle is isomorphic to a sum of the tangent bundle and the cotangent bundle
\begin{equation}\label{eq:GenTangentBundleDef}
E \cong T\mathcal{M}\oplus T^*\mathcal{M},
\end{equation}
i.e.\ at a point $x\in\mathcal{M}$ the typical fibre is isomorphic to the formal sum of a vector and a one-form. 

There exists a natural inner product on sections $X^I=\xi+\lambda$, $Y^I=\eta+\rho$, where $X,Y\in\Gamma(TM\oplus T^*\mathcal{M})$, given by
\begin{equation}
    (X,Y) = \frac{1}{2}\left(\rho(x)+\lambda(y)\right),
\end{equation}
which also can be written as $\eta_{MN}X^MY^N$ with $X^M,Y^N$ being the components of $X,Y$ respectively and 
\begin{equation}
    \eta_{MN} = \frac{1}{2}\begin{pmatrix}0&\mathbf{1}\\\mathbf{1}&0\end{pmatrix}.
\end{equation}
This inner product is manifestly invariant under $SO(d,d)$ transformations parameterised by elements in the algebra 
\begin{equation}
    O = \begin{pmatrix}
            A & \beta \\
            B & -A^{T}
        \end{pmatrix},
\end{equation}
with $A\in \mathfrak{gl}(d)$, an anti-symmetric two-vector $\beta=\beta^{ij}$ and $B$ a two-form. 

In order to include the presence of the NS-NS two-form $B$ we will look at ``twisted'' elements. These are sections of $E$ that provide explicitly the isomorphism stated between an element $\tilde{X}\in\Gamma(E)$ and $X=v+\lambda\in\Gamma(TM\oplus T^*\mathcal{M})$ as
\begin{equation}
    \tilde{X} := \ee^{B}X = v+\left(\lambda+\iota_v B\right),
\end{equation}
with $\iota_vB(y)=B(v,y)$ denote the interior product. The two-form $B$ is not globally defined, instead on overlapping patches $U_i\cap U_j\neq 0$ is shifted by an exact two-form
\begin{equation}
    B_i-B_j = \dd \lambda_{ij},
\end{equation}
with $\lambda_{ij}$ a one-form. This means that the twisted vectors are patched together as 
\begin{equation}
    \tilde{X}_i-\tilde{X}_j = \iota_v\dd \lambda_{ij}. 
\end{equation}
Patching of twisted vectors are therefore done by elements in the geometric subgroup $GL(d)\ltimes \Lambda^2_{\text{cl}}(\mathcal{M})$, with $\Lambda^2_{\text{cl}}(\mathcal{M})$ being closed two-forms on $\mathcal{M}$. There are some further consistency conditions such that this specifies a ``gerbe structure'' \cite{Hitchin:1999fh}, but this is outside the scope of the thesis and will not be needed. 
The infinitesimal action of a generalised diffeomorphism acting by $O(d,d)\times \mathbb{R}^+$ is parameterised by elements in $Y=w+\rho\in\Gamma(E)$ using the generalised Lie derivative on $X=x+\lambda\in\Gamma(E)$ as 
\begin{equation}
    \mathscr{L}_YX^M = V^N\pd_NX^M-(\pd\times_{\text{ad}} V)\tensor{}{^M_N} X^N.
\end{equation}
Here we embedded the ordinary derivative in $\pd_M$ as $\pd_M=(\pd_i,0)$ and we denoted $(V\times D)_{\text{ad}}$ the projection on the adjoint of $\overbar{E}\times E$ as  $(D\times_{\text{ad}} V)\tensor{}{^M_N}=D_NV^M-\eta_{NP}\eta^{MS}D_SV^P$. This agrees with the definition of the generalised Lie derivative introduced in DFT.


\begin{comment}
Moreover, an analogue of a Lie bracket on $TM$ is the Courant bracket defined as anti-symmetric part of the generalised Lie derivative 
\begin{equation}
    \begin{aligned}
    \left[X,Y\right]_C &=\frac{1}{2} \\
    &[x,y]_{\text{Lie}}+\mathcal{L}_x\rho-\mathcal{L}_y\lambda -\frac{1}{2}\dd\left(\rho(x)-\lambda(y)\right),
    \end{aligned}
\end{equation}
where $\mathcal{L}$ is the ordinary Lie derivative. Note the Courant bracket is anti-symmetric and does not satisfy the Jacobi-identity. Just as the ordinary Lie bracket is preserved under diffemorphisms also the Courant bracket is. Importantly the Courant bracket possess another symmetry under closed $B$-shifts 
\begin{equation}
    \begin{aligned}
        \left[\ee^B X,\ee^B Y\right]_{\text{C}} &= [X,Y]_{\text{C}}+[\ii_xB,y]+[x,[\ii_xB,y]]\\
                        &=[X,Y]_{\text{C}}+ \mathcal{L}_{x}\ii_yB-\mathcal{L}_{y}\ii_xB-\frac{1}{2}\dd\left(\ii_x\ii_yB-ii_y\ii_xB\right)\\
                        &= \mathcal{L}_x\ii_y B-\ii_y\mathcal{L}_xB+\ii_y\ii_x \dd B\\
                        &= \ee^B[X,Y]_{\text{C}}+\ii_y\ii_x \dd B,
    \end{aligned}
\end{equation}
where we used Cartan's identity $\mathcal{L}_x\lambda = \ii_x\dd\lambda+\dd\ii_x\lambda$ repeatedly. This tells us that both the inner product and the Courant bracket is preserved not only by diffeomorphism $\text{Diff}(\mathcal{M})$, but also under closed two-form transformations $\Lambda^2_{\text{cl}}(\mathcal{M})$. These combine into the semi-direct product of transformations 
\begin{equation}
    \text{Diff}(\mathcal{M})\ltimes\Lambda^2_{\text{cl}}(\mathcal{M}).
\end{equation}
We can introduce a twisting of the Courant bracket by an three-form $H$ as 
\begin{equation}
    [X,Y]_{H} := [X,Y]_{\text{C}}+\ii_x\ii_y H,
\end{equation}
it then follows that $[\ee^BX,\ee^BY]_{H}=\ee^{B}[X,Y]_{H-\dd B}$ for an arbitrary two-form $B$. We can choose $B$ transformations such that $H=\dd B$\todo{Do this!!}

The appearance of a two-form potential $B$ comes from the NS-NS field in ten dimensional supergravity. Note that the $B$-field need only be locally defined such that on overlapping patches $U_\alpha,U_\beta$ it can transform as 
\begin{equation}
   B_{(\alpha)}-B_{(\beta)} = \dd \Lambda_{(\alpha\beta)},
\end{equation}
where $\Lambda_{(\alpha\beta)}$ is an one-form defind on $U_\alpha\cap U_\beta$. For consistency such one-forms needs to satisfy the following on a triple overlap $U_\alpha\cap U_\beta\cap Y_\gamma\neq\emptyset$
\begin{equation}
    \Lambda_{(\alpha\beta)}+\Lambda_{(\beta\gamma)}+\Lambda_{(\gamma\alpha)} =\dd \Lambda_{\alpha\beta\gamma},
\end{equation}
with $\Lambda_{\alpha\beta\gamma}$ a function. This consistency condition follows from the reducibility of a closed $B$-shift transformation. 
\end{comment}

\subsection{Exceptional generalised geometry}
In the same spirit we discuss exceptional generalised geometry ($d\leq 7$) based on $E_{d(d)}\times \mathbb{R}^+$ and a generalised tangent bundle 
\begin{equation}
    E\cong T\mathcal{M}\oplus T^*\mathcal{M}\oplus\ldots,
\end{equation}
where $\ldots$ denote extra terms coming from the decomposition of the coordinate module of $E_{d(d)}$ under the $GL(d,\mathbb{R})$ subgroup. As an example consider $\text{dim}\,\mathcal{M}=6$, then the coordinate module is the fundamental $\mathbf{27}$ module of $E_{6(6)}$ which decompose under $GL(6,\mathbb{R})$ as 
\begin{equation}
    \mathbf{56}\to \mathbf{7}+\overbar{\mathbf{7}}+\mathbf{21}+\overbar{\mathbf{21}},
\end{equation}
corresponding to a vector, a one-form, a two-form and a five-form. The generalised tangent bundle $E$ is then given by 
\begin{equation}
    E \cong T\mathcal{M}\oplus\Lambda^2T^*\mathcal{M}\oplus\Lambda^5T^*\mathcal{M}\oplus\left(T^*\mathcal{M}\otimes \Lambda^7T^*\mathcal{M}\right).
\end{equation}
This is the general form for $E$ for $d\leq7$ with the last term absent if $d<7$. A section $X\in\Gamma(E)$ on a patch $U_i$ given by 
\begin{equation}
    X = v+\omega+\sigma+\tau,
\end{equation}
with $v\in\Gamma(TU_i)$, $\omega\in\Gamma(\Lambda^2T^*U_i)$, $\sigma\in\Gamma(\Lambda^5T^*U_i)$ and $\tau \in\Gamma(T^*U_i\otimes \Lambda^7T^*U_i)$.  Likewise the adjoint of $\mathfrak{e}_{7(7)}\times\mathbb{R}$ decompose as
\begin{equation}
    \mathbf{133}\to \mathbf{1}\oplus\mathbf{48}\oplus\mathbf{7}\oplus\overbar{\mathbf{7}}\oplus\mathbf{35}\oplus\overbar{\mathbf{35}}
\end{equation}
and the corresponding adjoint tangent bundle is given by
\begin{equation}
    \text{ad}\, F \cong \mathbb{R}\oplus \left(T^*\mathcal{M}\otimes T\mathcal{M}\right)\oplus \Lambda^3 T^*\mathcal{M}\oplus\Lambda^6 T^*\mathcal{M}\oplus\Lambda^3 T\mathcal{M}\oplus\Lambda^6 T\mathcal{M},
\end{equation}
with a section $W\in\Gamma(\text{ad}\,F)$ on a patch $U_i$ is given by
\begin{equation}\label{eq:AdDecompE7}
    W = l+r+a+\tilde{a}+\alpha+\tilde{\alpha},
\end{equation}
with $l\in\mathbb{R}$, $r\in (T^*U_i\otimes TU_i)$, $a\in \Lambda^3 T^*U_i$, $\tilde{a}\in \Lambda^6 T^*U_i$, $\alpha\in\Lambda^3 TU_i$ and $\tilde{\alpha}\in\Lambda^6 TU_i$. Note especially the presence of a three-form and a six-form which corresponds to the three-form and the dualised six-form of M-theory. Moreover, a section of $\Gamma(E)$ contains a vector, a one-form, a two-form and a five-form which parameterise diffeomorphisms and gauge transformations of these form fields. 

As in the introduction to $O(d,d)$ generalised geometry above sections of $E$ and $\text{ad}\,F$ are twisted by the form fields of the adjoint bundle which provides the isomorphism stated above. Explicitly a section $\tilde{X}\in \Gamma(E)$ of the generalised tangent bundle is twisted by the form fields in the adjoint bundle as
\begin{equation}
    \tilde{X} = \ee^{a+\tilde{a}}X,
\end{equation}
with an action of $E_{7(7)}\times\mathbb{R}^+$ defined below. Elements in the generalised tangent bundle is therefore patched together on $U_{(i)}\cap U_{(j)}\neq 0$ as 
\begin{equation}
    X_{(i)} = \ee^{\dd\Lambda_{(ij)}+\dd\tilde{\Lambda}_{(ij)}}X_{(j)},
\end{equation}
with a three-form $\dd \Lambda_{(ij)}$  and a six-form $\dd\tilde{\Lambda}_{(ij)}$ corresponding to the gauge symmetry of the three-form and six-form potential in M-theory respectively. 


There is a natural action of $E_{d(d)}\times\mathbb{R}^+$, given in e.g.\ \cite{Ashmore:2015joa}, on a section $X\in\Gamma(E)$ as $E' = W\cdot E$ which in components is given by 
\begin{equation}
    \begin{aligned}
        v'&= lv+r\cdot v+\alpha\lrcorner\omega-\tilde{\alpha}\lrcorner\sigma,\\
        \omega'&=lv+r\cdot \omega+v\lrcorner a+\alpha\lrcorner\sigma+\tilde{\alpha}\lrcorner\tau,\\
        \sigma'&= l\sigma+r\cdot \sigma+v\lrcorner\tilde{a}+a\wedge\omega+\alpha\lrcorner\tau,\\
        \tau'&= l\tau+r\cdot\tau-j\tilde{a}\wedge\omega+ja\wedge\sigma,
    \end{aligned}
\end{equation}
with the definitions 
\begin{equation}
    \begin{aligned}
        (v\wedge v')^{i_1i_2\ldots i_{n+n'}} &:= \frac{(n+n')!}{n!n'!}v^{[i_1i_2\ldots}v'^{i_{n+1}\ldots i_{n'}]},\\
        (\omega\wedge\omega')_{i_1\ldots i_{n+n'}} &:= \frac{(n+n')!}{n!n'!}\omega_{[i_1i_2\ldots}\omega'_{i_{n+1}\ldots i_{n+n'}]},\\
        (v\lrcorner\omega)_{i_1\ldots i_{n'-n}} &:= \frac{1}{n!}v^{j_1j_2\ldots j_n}\omega_{j_1j_2\ldots j_ni_{1}\ldots i_{n'-n}} \qquad \text{if }\, n<n',\\
        (v\lrcorner\omega)^{i_1\ldots i_{n'-n}} &:= \frac{1}{n'!}v^{j_1j_2\ldots j_{n'}i_{n'+1}\ldots i_{n-n'}}\omega_{j_1j_2\ldots j_{n'}} \qquad \text{if }\, n'<n,\\
        (jv\lrcorner j\omega)\tensor{}{^i_j}&:= \frac{1}{(n-1)!}v^{ik_1k_2\ldots k_{n-1}}\omega_{jk_1k_2\ldots k_{n-1}},\\
        (j\omega\wedge\omega')_{i,i_1i_2\ldots i_{d}} &:= \frac{d!}{(n-1)!(d+1-n)!}\omega_{i[i_1i_2\ldots i_{n-1}}\omega'_{i_ni_{n+1}\ldots i_d]}.
    \end{aligned}
\end{equation}
Moreover, the $\mathfrak{gl}(d)$ subalgebra parameterised by an element $r$ acts naturally as $r\cdot v^i=\tensor{r}{^i_j}v^i$, $r\cdot\omega_{ij}=-\tensor{r}{^k_i}\omega_{kj}-\tensor{r}{^k_j}\omega_{ik}$ with obvious extensions to other fields. The adjoint action $W''=[W,W']$ with elements  $W\in\Gamma(\text{ad}\,F)$ and $W'\in\Gamma(\text{ad}\,F)$ is further given by 
\begin{equation}\label{eq:E7AdjointAction}
    \begin{aligned}
        l'' &= \frac{1}{3}(\alpha\lrcorner a'-\alpha'\lrcorner a)+\frac{2}{3}(\tilde{\alpha}'\lrcorner\tilde{a}-\tilde{\alpha}\lrcorner\tilde{a}'),\\
        r'' &= r\cdot r'+j\alpha\lrcorner ja'-j\alpha'\lrcorner ja-\frac{1}{3}\mathbf{1}(\alpha\lrcorner a'-\alpha'\lrcorner a)\\
        &\qquad+j\tilde{\alpha}'\lrcorner j\tilde{a}-j\tilde{\alpha}\lrcorner j\tilde{a}'-\frac{2}{3}\mathbf{1}(\tilde{\alpha}'\lrcorner \tilde{a}-\tilde{\alpha}\lrcorner \tilde{a}'),\\
        a'' &= r\cdot a'-r'\cdot a+\alpha'\cdot\tilde{a}-\alpha\lrcorner \tilde{a}'\\
        \tilde{a}'' &= r\cdot\alpha'-r'\cdot\alpha-a\wedge a',\\
        \alpha'' &= r\cdot\alpha'-r'\cdot\alpha+\tilde{\alpha}'\lrcorner a-\tilde{\alpha}\lrcorner a',\\
        \tilde{\alpha}'' &= r\cdot\tilde{\alpha}'-r'\cdot\tilde{\alpha}-\alpha\wedge\alpha'.
    \end{aligned}
\end{equation}

Just as in the $O(d,d)$ generalised case as well as extended geometries we can introduce a generalised Lie derivative by the now well-known structure
\begin{equation}
    \mathscr{L}_Y X^M := Y^N\pd_NX^M-(\pd_N\times_{\text{ad}}Y)\tensor{}{^M_N}X^N,
\end{equation}
with again embedding the ordinary partial derivative as $\pd_M=(\pd_i,0,0,0)$ in obvious notation and $\times_{\text{ad}}$ denoting the projection of $\overbar{E}\otimes E$ to the adjoint. Using $(\pd\times_{\text{ad}}X)=\pd\times v+\dd\omega+\dd\sigma$ it is easily found that 
\begin{equation}\label{eq:GengeometryLieDerivate}
    \begin{aligned}
        \mathscr{L}_X X' &= L_vv'+(L_v\omega-v'\lrcorner\dd\omega)+(L_v\sigma'-v'\lrcorner\dd\sigma-\omega'\wedge\dd\omega)\\
        &+(L_v\tau'-j\omega'\wedge\dd\sigma-j\sigma'\wedge\dd\omega)\\
        &= L_v X'-W\cdot X',
    \end{aligned}
\end{equation}
with $L_v$ being the ordinary Lie derivative with respect to the vector component $GL(7)$ vector $v$ and $W=\dd\omega+\dd\sigma \in\Gamma(\text{ad}\,F)$. 


\section{Generalised structure\label{sec:GeneralisedStructures}}
The bosonic sector of M-theory is given by the metric $g_{MN}$ and the three-form form field $A$ and its corresponding four-form field strength $F=\dd A$. We consider a warped product ansatz for the metric 
\begin{equation}
    \dd s^2 = \ee^{2A(y)}\dd s^2(\mathbb{R}^{1,D-1})+\dd s^2(\mathcal{M}),
\end{equation}
where $x,y$ are coordinates on $\mathbb{R}^{1,D-1}$ and $\mathcal{M}$ respectively and $A$ depends only on the internal manifold to preserve external Poincaré invariance. The non-trivial fluxes will be chosen to lie entirely in $\mathcal{M}$, again to preserve the symmetries in the external spacetime. For simplicity we will consider the specific example of $D=4$ such that the internal manifold is seven-dimensional following the work of \cite{Ashmore:2015joa}. The generalised tangent space with respect to an $E_{7(7)}\times\mathbb{R}^+$ exceptional generalised geometry is then given by 
\begin{equation}
    E \cong T\mathcal{M}\oplus\Lambda^2T^*\mathcal{M}\oplus\Lambda^5T^*\mathcal{M}\oplus\left(T^*\mathcal{M}\otimes\Lambda^7 T^*\mathcal{M}\right).
\end{equation}

\begin{comment}
Moreover, given $E$ we can construct the generalised frame bundle $F$ given by frames $\{E_A\}$, where $A=1,2,\ldots 56$. Explicitly, in a patch $U$ there exists a canonical basis given by 
\begin{equation}
    \{E_A\} = \{\pd/\pd y^m\}\cup \{\dd y^{m_1}\wedge\dd y^{m_2}\}\cup\{\dd y^{m_1}\wedge\ldots\wedge\dd y^{m_5}\}\cup \{\dd y^m\otimes\dd y^{m_1}\wedge\ldots\wedge\dd y^{m_7}\},
\end{equation}
and any other frame is related to such a frame by an $E_{7(7)}\times \mathbb{R}^+$ transformation, this then defines a principle bundle $F$ with structure group $E_{7(7)}\times \mathbb{R}^+$. Any other tensor field can then be constructed as sections of an associated vector bundle, thus making the exceptional group manifest in the formulation. 
\end{comment}

In ordinary geometry the reduction of the structure group to some $G$-structure, where $G$ is a subgroup of the structure group, proved fruitful to characterise geometries preserving some amount of supersymmetry. The goal is therefore to again find ``reduced structures'', this time for some $G\subset E_{7(7)}$ subgroup. This is done by looking for tensors invariant under the action of $G$, or in other words, finding singlets when decomposing modules of $E_{7(7)}$ under $G$. 

To this end decompose the adjoint representation $\mathbf{133}$ of $E_{7(7)}$ under $SU(2)\times Spin^*(12)\subset E_{7(7)}$, with $Spin^*(12)$ a specific real form of the double cover of $SO(12)$,
\begin{equation}
    \mathbf{133}\to (\mathbf{1},\mathbf{66})+(\mathbf{2},\mathbf{32})+(\mathbf{3},\mathbf{1}).
\end{equation}
We thus find a triplet of $E_{7(7)}$ tensors which are invariant under $Spin(12)^*(12)$. Such a triplet $J_\alpha$ of invariant tensors define a structure, called the hypermultiplet structure, and are sections of the adjoint bundle 
\begin{equation}
    J_\alpha \in \Gamma(\text{ad}\, F\otimes \left(T^*\mathcal{M}\right)^{1/2}).
\end{equation}
The hypermultiplet structure satisfies the $\mathfrak{su}(2)$ algebra 
\begin{equation}
    [J_\alpha,J_\beta] =2\rho \epsilon_{\alpha\beta\gamma}J_\gamma,
\end{equation}
with $\rho\in\text{det}\,T^*\mathcal{M}$. As in the case of vanishing flux the hypermultiplet structure $J_\alpha$ can be written in terms of the Killing spinors. Before moving on to the integrability condition that the hypermultiplet structure should satisfy we look for another structure. Decomposing the fundamental $\mathbf{56}$ of $E_{7(7)}$ under $E_{6(2)}$
\begin{equation}
    \mathbf{56}\to\mathbf{27}+\overbar{\mathbf{27}}+2\cdot \mathbf{1}.
\end{equation}
The two singlets in this decomposition $V$ and $\hat{V}$ defines the vector multiplet structure, and hence also an $E_{6(2)}$ sub-bundle. There is one further requirement that the singlet $V$ is chosen such that 
\begin{equation}
    q(V)>0,
\end{equation}
where $q$ is the quartic invariant of $E_7$. This ensures that the stabiliser group is precisely the real form $E_{6(2)}$ \cite{Ferrara:1997uz}. The structure $\hat{V}$ is further defined in terms of $V$ as 
\begin{equation}
    s(Y,\hat{V}) = \frac{2}{\sqrt{q(V)}}q(Y,V,V,V),
\end{equation}
with $s(\cdot,\cdot)$ being the symplectic invariant of $E_{7(7)}$ and for any section $Y\in\Gamma(E)$. It is also convenient to combine these into a complex object as $X:= V+\ii\hat{V}$. 

Just as in ordinary Calabi-Yau there are certain consistency conditions between the invariant tensors that defines the structure. Specifically this further restricts the structure group to their overlap defining a $SU(6)=E_{6(2)}\cap Spin^*(12)$ structure \cite{Grana:2009im}. The compatibility condition between the hypermultiplet $J_\alpha$ and the vector multiplet $X=V+\ii \hat{V}$ is given by 
\begin{equation}\label{eq:CompatibilityVectorHyperStructure}
    \begin{aligned}
        J_+ \cdot X = J_-\cdot X = 0\qquad&\Longleftrightarrow\qquad J_\alpha\cdot V = 0\\
        \frac{1}{2}\ii s(X,\overbar{X}) = \rho^2\qquad &\Longleftrightarrow\qquad \kappa(J_\alpha,J_\beta) = -2\sqrt{q(V)}\delta_{\alpha\beta}
    \end{aligned}
\end{equation}
with $\rho\in\text{det}\, T^*\mathcal{M}^\frac{1}{2}$ and $J_\pm = J_1\pm \ii J_2$. 

%To determine that $\mathfrak{E}_{6(2)}\subset \mathfrak{e}_{7(7)}$ look at the decomposition of the adjoint  
%\begin{equation}\label{eq:e7toe6}
%    \mathbf{133}\to \overbar{\mathbf{27}}\oplus\mathbf{78}\oplus\mathbf{1}\oplus\mathbf{27}
%\end{equation}

%The missing $27$ non-compact generators in $\eqref{eq:e7toe6}$ thus corresponds to $(\mathbf{27},\mathbf{1})_{1}$ in \eqref{eq:e7toe6adj} which indeed are non-compact since acting in any representation they are triangular matrices and hence non-compact. 


\subsection{Integrability of structures}

Integrability of the hypermultiplet structure and of the vector multiplet structure are in analogy to the fluxless case given by differential conditions on the corresponding invariant tensors. In other words, these differential conditions are necessary for the Killing spinor satisfying the Killing spinor equation, which in turn is the condition for preserving a certain amount of supersymmetry. In the case of vanishing flux this was related to vanishing intrinsic torsion, i.e.\ given a $G$-structure the existence of an $G$-compatible torsion-free connection implied that the structure was integrable. Motivated by this analogy we will look at the conditions of vanishing generalised intrinsic torsion.

\subsubsection{Generalised intrinsic torsion}
To define generalised intrinsic torsion we first have to define a covariant derivative $D$ which is defined in the following way 
\begin{equation}
    D_M\cdot V^N = \pd_MV_N+\tensor{\Gamma}{_M_P^N}V^P,
\end{equation}
where $V\in\Gamma(E)$, $\tensor{\Gamma}{_{MP}^N}\in\Gamma(E^*\otimes \text{ad}\,F)$ and $\pd_M=(\pd_m,0,0,0)$. Imposing Leibniz rule 
\begin{equation}
    D_M(fV) = (D_Mf)\otimes V+f(D_MV),
\end{equation}
the action of a covariant derivative is easily extended to sections of any associated vector bundle, i.e.\ to any other generalised tensor field, analagous to the covariant derivative in section \ref{sec:CovariantFormalism}. The torsion of such a connection is then defined by 
\begin{equation}
    \mathscr{T}(V)\cdot X = L_V^D X-L_V X,
\end{equation}
where $X$ is an arbitrary generalised tensor field, $L_V^D$ is the generalised Lie derivative with the replacement $\pd\to D$ and $\cdot$ denotes the adjoint action. Note that this definition agrees with the one presented in extended geometries, i.e.\ torsion is the part of the connection that transforms homogeneously under generalised diffeomorphisms, at least when ancillary transformations are absent. In the $E_{7(7)}$ case we already know from section \ref{sec:TorsionSubSec} that the torsion $\mathscr{T}$ of the $E_{7(7)}$ covariant derivative are given by
\begin{equation}
    \mathscr{T}\in \mathbf{56}\oplus\mathbf{912}.
\end{equation}

We define a $G$-compatible connection as one that preserves $G$-invariant tensors defining the structure. As an example an $E_{6(2)}$ structure were defined by an $E_{6(2)}$ invariant tensor $V\in\Gamma(E)$, the connection $D$ is then $E_{6(2)}$ compatible if $DV=0$. Assuming a connection $D$ is compatible with the $G$-structure any other compatible connection is given by $D\to D+\Sigma$ with 
\begin{equation}
    \Sigma \in \Gamma(E^*\otimes \text{ad}\,P_G),
\end{equation}
where $\text{ad}\,P_G$ has a typical fibre $\text{Lie}\,G=\mathfrak{g}$. Given a $G$-compatible connection with torsion we define the intrinsic torsion as the part of torsion that can not be removed by adding a term $\Sigma$. With this definition it is clear that the presence of intrinsic torsion is an obstruction to finding a $G$-compatible torsion-free connection. This is analogous to the appearence of intrinsic torsion introduced for ordinary geometry. 

Another way to put it is to define a map $\phi$ that projects $\Sigma$ onto its torsion part. The intrinsic torsion is then given by 
\begin{equation}
    W_{\text{int}} = W/\text{im}\,\phi,
\end{equation}
with $W_{\text{int}}$ and $W$ being the space of intrinsic torsion and the space of torsion respectively. 

To clarify this take our main example with $E_{7(7)}\times R^+$ and $G=Spin^*(12)\times SU(2)$. To see what parts are intrinsic torsion decompose the torsion $\mathbf{56}\oplus\mathbf{912}$ under $G$ to find 
\begin{equation}
    \mathbf{56}\oplus\mathbf{912} \to 2\cdot (\mathbf{12},\mathbf{2})+(\mathbf{32},\mathbf{1})+(\mathbf{32},\mathbf{3})+(\mathbf{220},\mathbf{2})+(\mathbf{352},\mathbf{1}).
\end{equation}
Likewise we decompose $\Sigma$ under $G$ using $\mathbf{56}\to (\mathbf{12},\mathbf{2})+(\mathbf{32},\mathbf{1})$ 
\begin{equation}
    \begin{aligned}
        \left((\mathbf{12},\mathbf{2})+(\mathbf{32},\mathbf{1})\right)\times (\mathbf{66},\mathbf{1})=&(\mathbf{220},\mathbf{2})+(\mathbf{12},\mathbf{2})+(\mathbf{560},\mathbf{2})+(\mathbf{22},\mathbf{1})\\
        &+(\mathbf{1728},\mathbf{1})+(\mathbf{352},\mathbf{1}).
    \end{aligned}
\end{equation}
The two big modules containing $\mathbf{1728}$ and $\mathbf{560}$ are not torsion and hence we find the intrinsic torsion 
\begin{equation}
    W_{\text{int}} = (\mathbf{12},\mathbf{2})+(\mathbf{32},\mathbf{3}).
\end{equation}

Performing the same analysis for $G=E_{6(2)}$ is straightforward and the torsion decompose as
\begin{equation}
    \mathbf{56}+\mathbf{912}\to \mathbf{1}+2\cdot\mathbf{27}+\mathbf{78}+\mathbf{351}+\text{c.c.}
\end{equation}
and we also have 
\begin{equation}
    \left(\mathbf{1}+\mathbf{27}+\text{c.c.})\right)\times \mathbf{78} = \mathbf{78}+\mathbf{27}+\mathbf{351}+\mathbf{1728}+\text{c.c.}
\end{equation}
We thus find the intrinsic torsion in the $E_{6(2)}$ case to be 
\begin{equation}
    W_{\text{int}}^{E_{6(2)}} = \textbf{1}+\mathbf{27}+\text{c.c.}
\end{equation}


Consider the definition of an $E_{6(2)}$ compatible covariant derivative $DV=0$. From the definition of torsion it is easily found that 
\begin{equation}
    \begin{aligned}
    \mathscr{L}_VV^M &= \mathscr{L}^D_VV^M-\mathscr{T}\cdot V^M\\
                    &= \underbrace{V^ND_NV^M}_{=0}-\underbrace{D_NV^MV^N}_{=0}-\mathscr{T}\cdot V^M.
    \end{aligned}
\end{equation}
By definition of intrinsic torsion we can split $T=T_{\text{int}}+T_0$, with $T_{\text{int}}\in \mathbf{56}\otimes (\mathfrak{e}_{7(7)}\times\mathbb{R}/\mathfrak{e}_{6(2)})$ and $T_0\in \mathbf{56}\otimes\mathfrak{e}_{6(2)}$, and since $V$ is invariant under $\mathfrak{e}_{6(2)}$ we find 
\begin{equation}
    \mathscr{L}_V V = -\mathscr{T}_{\text{int}}\cdot V.
\end{equation}
Demanding that the intrinsic torsion vanish thus gives the integrability condition on the vectormultiplet structure 
\begin{equation}\label{eq:IntegrabilityVectorMultiplet}
    \mathscr{L}_K K = 0.
\end{equation}
This is further equivalent to $\mathscr{L}_X\overbar{X}=0$. 

It was further shown in \cite{Ashmore:2015joa} that the vanishing of intrinsic torsion and hence integrability of the generalised structure is equivalent to vanishing of so-called \emph{momentum maps}. For the hyper-multiplet structure this is given by a triplet of momentum maps $\mu_\alpha$ acting on any element $Y\in\Gamma(E)$ and is given by 
\begin{equation}
    \mu_\alpha(Y):= -\frac{1}{2}\epsilon_{\alpha\beta\gamma}\int_\mathcal{M}\kappa(J_\beta,\mathscr{L}_YJ_\gamma),
\end{equation}
with $\kappa$ being the $\mathfrak{e}_{7(7)}$ Killing form. The integrability is then the requirement that the triplet of momentum maps vanish for any element in the generalised tangent bundle
\begin{equation}
    \mu_\alpha(Y) = 0\qquad \text{for any }\, Y\in\Gamma(E).
\end{equation}
Likewise the integrability condition in \eqref{eq:IntegrabilityVectorMultiplet} can be equivalently written as a momentum map $\mu$ as well
\begin{equation}
    \mu(Y) := -\frac{1}{2}\int_\mathcal{M}s(V,\mathscr{L}_YV),
\end{equation}
and an integrable vector-multiplet structure $V$ satisfies 
\begin{equation}
    \mu(Y) = 0\qquad \text{for any }\, Y\in\Gamma(E).
\end{equation}

If the hyper-multiplet structure and the vector-multiplet structure separately satisfies the integrability condition they define a reduced structure. To define a $SU(6)$ structure they also need to satisfy the compatiblity condition given in \eqref{eq:CompatibilityVectorHyperStructure} as well as the condition 
\begin{equation}
\mathscr{L}_XJ_\alpha = 0,
\end{equation}
with $X=V+\ii \hat{V}$, in order for the intrinsic torsion with respect to the $SU(6)$ structure to vanish. \cite{Ashmore:2015joa}
%Given the representations of intrinsic torsion we want to show that vanishing intrinsic torsion is equivalent to vanishing of so-called momentum maps which are differential conditions on the underlying structures. 

In \cite{Coimbra:2014uxa,Ashmore:2015joa} it was shown the vanishing of intrinsic torsion and hence vanishing of the momentum maps with respect to a $Spin^*(12)\cap E_{6(2)}=SU(6)$ structure is equivalent to preserving $\mathcal{N}=2$ supersymmetry. The idea is that vanishing of the supersymmetry variation of fermionic fields are naturally expressed as Killing spinors being generalised covariantly constant. One then shows that this demands vanishing intrinsic torsion with respect to a $SU(6)$ structure. Moreover, since the gauge degrees of freedom are naturally included in the geometry these also appear naturally through the covariant derivative; which should be compared to ordinary formulation discussed above were they instead were obstructions to finding covariantly constant Killing spinors. Using exceptional generalised geometry we have thus seen how to use generalised geometrical tools to study supersymmetric flux compactifications similar to the way geometrical tools introduced in section \ref{sec:VanishingFluxCompactification} were used to study supersymmetric compactifications with vanishing flux. The appearance of $SU(6)$ can be further understood by noting that the Killing spinor transforms in $\mathbf{8}$ under (double cover) of the maximal compact subgroup $SU(8)$ of $E_{7(7)}$. Demanding $\mathcal{N}=2$ preserved supersymmetry is then equivalent to the Killing spinor being stabilised by $SU(6)\subset SU(8)$. 


There are further interesting models to study using the tools introduced above. In \cite{Grana:2016dyl,Ashmore:2016qvs} compactifications with an external external AdS spacetime are examined. This is further applied to study the AdS/CFT correspondence in \cite{Ashmore:2016oug} in which they compare deformations of the CFT dual to deformations of the hyper multiplet and vector multiplet in generalised geometry. In \cite{Coimbra:2016ydd} it was further shown that any supersymmetric flux compactification is equivalent to some integrable $G$ structure. 

%In \cite{Ashmore:2015joa} they further encode the vanishing of intrinsic torsion by constructing momentum maps 

%It then follows that a reduced structure on which the momentum maps vanish corresponds to a supersymmetric compactification with fluxes. 

\subsection{Example Calabi-Yau}
To clarify some of the concepts introduced above we consider a specific example of compactification on $CY_3\times S^1$ following \cite{Ashmore:2015joa}. Especially we would like to reproduce the known results obtained from ordinary geometric tools. However, in order to continue with a generalised geometry based on $E_{7(7)}\times \mathbb{R}^+$ we consider a slight extension by adding a circle $S^1$. Just as in the Calabi-Yau case discussed above there exists a closed symplectic two-form $\omega$, a closed complex three-form $\Omega$ and then there also exists a closed one-form $\zeta=\dd y$ with $y$ being the coordinate on $S^1$. These structures satisfy the following compatibility conditions 
\begin{equation}
    \begin{cases}
        \omega\wedge\Omega = 0,\\
        \omega\wedge\omega\wedge\omega = \frac{3\ii}{4}\Omega\wedge\overbar{\Omega},\\
        \iota_{\zeta^{\#}}\omega = \iota_{\zeta^{\#}}\Omega = 0,
    \end{cases}
\end{equation}
where we have defined an isomorphism $^{\#}$ on any form field as $(\rho^{\#})^{i_1i_2\ldots i_p}=g^{i_1j_1}g^{i_2j_2}\ldots\rho_{j_1j_2\ldots j_p}$, especially $\zeta^{\#}=\pd_y$ with $y$ being the coordinate along $S^1$. Moreover the integrability conditions state that $\omega,\Omega$ and $\zeta$ are all closed and the volume form is given by 
\begin{equation}
    \rho^2 := \text{vol}_7 = \frac{\ii}{8}\Omega\wedge\overbar{\Omega}\wedge \zeta. 
\end{equation}
These invariant tensors can be written explicitly in a basis $\{e^a\}_{a=1,2,\ldots 7}$, with $e_7=\zeta=\dd y$, as 
\begin{equation}
    \begin{aligned}\label{eq:Frame}
        \omega &= e^{12}+e^{34}+e^{56},\\
        \Omega &= (e^1+\ii e^{2})\wedge(e^3+\ii e^{4})\wedge(e^5+\ii e^{6}),
    \end{aligned}
\end{equation}
from which the compatibility conditions follow immediately. Especially, in this basis we define the hodge dual $\star$ as 
\begin{equation}
    \star e^{i_1i_2\ldots i_p} = \frac{1}{q!}\tensor{\epsilon}{^{i_1i_2\ldots i_p}_{j_1j_2\ldots j_q}}e^{j_1j_2\ldots j_q}e^{j_1j_2\ldots j_q},
\end{equation}
and $\tensor{\epsilon}{^{12\ldots d}}=\tensor{\epsilon}{^{12\ldots d}}=1$. Importantly the inner product of forms with the same degree $\lambda,\rho$ is then given by
\begin{equation}
    \text{vol}_7(\rho^{\#}\lrcorner \rho) = \lambda\wedge\star\rho.
\end{equation}
Moreover, it is easily found that $\star\Omega = \ii\Omega\wedge\zeta$ which will be useful below. 
%These together define a $SU(3)$ structure if they moreover satisfy the compatibility conditions 
%\begin{equation}
%    g^{mn}\zeta_m\omega_{np} = 0\qquad \zeta^{\#}\Omega_{npl} = 0,
%\end{equation}
%as well as the compatibility conditions for $\omega$ and $\Omega$ given in \eqref{eq:CompatibilityCalabiYau}. We also defined the operation $^{\#}$ which is an isomorphism between $p$-forms and anti-symmetric $p$-vectors by simply raising indices with the metric, e.g.\ $(\Omega^{\#})^{mnp}=g^{ms}g^{nk}g^{pl}\Omega_{skl}$

We then want to show that there exists an integrable hyper-multiplet structure $J_\alpha$ and an integrable vector-multiplet structure $V$ given in terms of ${\omega,\Omega,\zeta}$ which satisfies the generalised compatibility and integrability conditions. Furthermore, we will assume that the forms$\omega,\Omega,\zeta)$ are given by the expressions above and satisfy the ordinary compatibility and integrability conditions. In \cite{Ashmore:2015joa} these conditions are on the other hand derived from the generalised conditions. 

The hyper-multiplet structure is given by
\begin{equation}\label{eq:HyperMultipletStructureCalabiYau}
    \begin{aligned}
        J_+ &= \underbrace{\frac{1}{2}\rho\Omega}_{a} \underbrace{-\frac{1}{2}\rho\Omega^{\#}}_{\alpha},\\
        J_- &= \underbrace{\frac{1}{2}\rho\overbar{\Omega}}_{a} \underbrace{-\frac{1}{2}\rho\overbar{\Omega}^{\#}}_{\alpha},\\
        J_3 &= \underbrace{\frac{1}{2}\rho I}_{r}\underbrace{-\ii\frac{1}{16}\rho\Omega\wedge\overbar{\Omega}}_{\tilde{a}}\underbrace{-\frac{1}{16}\ii\rho\Omega^{\#}\wedge\overbar{\Omega}^{\#}}_{\tilde{\alpha}},
    \end{aligned}
\end{equation}
with $\tensor{I}{^m_n}=-\tensor{\omega}{^m_n}=\frac{1}{8}\ii (\overbar{\Omega}^{mpq}\omega_{npq}-\Omega^{mpq}\overbar{\Omega}_{npq})$ being the complex structure and we have for convenience identified the elements with the general decomposition given in \eqref{eq:AdDecompE7}. The vector-multiplet structure on the other hand is given by 
\begin{equation}\label{eq:VectorMultipletStructureCalabiYau}
    X = \underbrace{\zeta^{\#}}_{v}+\underbrace{\ii\omega}_{\omega}\underbrace{-\frac{1}{2}\zeta\wedge\omega\wedge\omega}_{\sigma}\underbrace{-\ii\zeta\otimes\text{vol}_7}_{\tau}.
\end{equation}

First we need to check that $J_\alpha$ indeed is a $\mathfrak{su}(2)$ triplet using the action defined in \eqref{eq:E7AdjointAction} by denoting $W''=[J_+,J_-]$ we find 
\begin{equation}
    \begin{aligned}
        l''&= \frac{\rho^2}{12}(-\Omega^{ijk}\overbar{\Omega_{ijk}}+\overbar{\Omega}^{ijk}\Omega_{ijk}) = 0,\\
        r''&= \rho^2j(-\frac{1}{2}\Omega^{\#})\lrcorner j(\frac{1}{2}\overbar{\Omega})-\rho^2j(-\frac{1}{2}\overbar{\Omega}^{\#})\lrcorner j(\frac{1}{2}\Omega)\\
        &\qquad -\frac{\rho^2}{3}\mathbf{1}(-\frac{1}{2}\Omega^{\#}\lrcorner\frac{1}{2}\overbar{\Omega}+\frac{1}{2}\overbar{\Omega}^{\#}\lrcorner\frac{-1}{2}\Omega)\\
        &= -2\ii I\rho^2,\\
        a''&=0,\\
        \tilde{a}'' &= -\frac{\rho^2}{4}\Omega\wedge\overbar{\Omega},\\
        \alpha'' &= 0,\\
        \tilde{\alpha}'' &= -\frac{\rho^2}{4}\Omega^{\#}\wedge\overbar{\Omega}^{\#}.
    \end{aligned}
\end{equation}
It is then seen that indeed $W''=-4\ii\rho J_3$. Now instead let $W''=[J_+,J_3]$ and we find 
\begin{equation}
    \begin{aligned}
        l'' &= 0,\\
        r'' &= 0,\\
        a'' &= -\frac{1}{8}\rho^2 I\cdot \Omega+\frac{\ii}{32}\rho^2\Omega^{\#}\lrcorner (\Omega\wedge\overbar{\Omega}),\\
        \tilde{a}'' = 0,\\
        \alpha '' &= \frac{1}{8}\rho^2 I\cdot \Omega^{\#}-\frac{\ii}{16}\rho^2(\Omega^{\#}\wedge\overbar{\Omega})\lrcorner\Omega,\\
        \tilde{\alpha}'' = 0,
    \end{aligned}
\end{equation}
which using $I\cdot \Omega=-3\ii\Omega$ and $I\cdot \Omega^{\#}=-3\ii\Omega^{\#}$ we find that $[J_+,J_3]=2\ii\rho J_+$ and indeed $\{J_\pm,J_3\}$ is a $\mathfrak{su}(3)$ triplet. 

In order to check the compatibility condition $J_+\cdot X=J_-\cdot X=0$ we repeat the action of an element $W=r+a+\tilde{a}+\alpha+\tilde{\alpha}\in\text{ad}\,F$ on an element $Y=v+\omega+\sigma+\tau$
\begin{equation}\label{eq:E7action}
    \begin{aligned}
        v'&= r\cdot v+\alpha\lrcorner\omega-\tilde{\alpha}\lrcorner\sigma,\\
        \omega'&=r\cdot \omega+v\lrcorner a+\alpha\lrcorner\sigma+\tilde{\alpha}\lrcorner\tau,\\
        \sigma'&= r\cdot \sigma+v\lrcorner\tilde{a}+a\wedge\omega+\alpha\lrcorner\tau,\\
        \tau'&= r\cdot\tau-j\tilde{a}\wedge\omega+ja\wedge\sigma.
    \end{aligned}
\end{equation}

Consider first $X'=J_+\cdot X=0$ then by the action \eqref{eq:E7action} we find 
\begin{equation}
    \begin{aligned}
        v' &= -\frac{\ii}{2}\rho\Omega^{\#}\lrcorner\omega,\\
        \omega' &= \frac{1}{2}\rho\zeta^{\#}\lrcorner \Omega+\frac{1}{4}\rho\Omega^{\#}\lrcorner (\zeta\wedge\omega\wedge\omega),\\
        \sigma' &= \frac{\ii}{2}\rho\Omega\wedge \omega+\frac{\ii}{2}\rho \Omega^{\#}\lrcorner (\zeta\otimes \text{vol}_7),\\
        \tau' &= -\frac{1}{4}\rho \Omega\wedge\zeta\wedge\omega\wedge\omega. 
    \end{aligned}
\end{equation}
Using the explicit form of $\Omega$ and $\omega$ it is easily found that $v'=\Omega^{\#}\lrcorner \omega = 0$ which also sets $\omega'=0$ using that $\iota_{\zeta^{\#}}\Omega=\iota_{\zeta^{\#}}\omega=0$. The second term in $\sigma'$ also vanish since $\text{vol}_7\propto \omega\wedge\omega\wedge\omega$ and $\Omega\wedge\omega =0$. Moreover, since $\Omega\wedge\omega\wedge\omega=0$ it follows that $\tau'$ vanish as well. Likewise it is found that $J_-\cdot X = 0$. We also need to check the normalisation condition $\frac{1}{2}\ii s(X,\overbar{X})=\rho^2$ with the symplectic invariant given for any vector $Y,Y'\in\Gamma(E)$
\begin{equation}
    s(Y,Y') = -\frac{1}{4}(\iota_{v}\tau-\iota_{v'}\tau+\sigma\wedge\omega'-\sigma'\wedge\omega), 
\end{equation}
from which we find 
\begin{equation}
    \begin{aligned}
    \frac{1}{2}\ii s(X,\overbar{X}):&= -\frac{1}{8}\ii \left(-2\ii\iota_{\zeta^{\#}}\zeta\text{vol}_7+\ii\zeta\wedge\omega\wedge\omega\wedge\omega\right)\\
    &= \rho^2,
    \end{aligned}
\end{equation}
where we used $\rho^2=\text{vol}_7=\frac{1}{8}\ii\Omega\wedge\overbar{\Omega}\wedge\zeta$ and the compatibility condition $\omega\wedge\omega\wedge\omega=\frac{3}{4}\ii\Omega\wedge\overbar{\Omega}$. We thus find that $J_\alpha$ and $X$ indeed satisfy a compatible $SU(6)$ structure. Moreover, this reproduce the known compatibility condition known from ordinary Calabi-Yau compactification introduced in section \ref{sec:CalabiYau}. 


We also need to check if this structure is integrable corresponding to satisfying the differential conditions on the underlying Killing spinor. As we will see this reproduce the known integrability condition on the Calabi-Yau structure. Consider the triplet of momentum maps $\mu_\alpha(V)=0$ determining integrability for the hyper multiplet structure. These can be rewritten in in $J_\pm$ basis as 
\begin{equation}
    \begin{aligned}
    \mu_3(Y)=& -\frac{\ii}{2}\int_\mathcal{M} \kappa(J_-,\mathscr{L}_YJ_+)\\
             &-\frac{\ii}{2}\int_\mathcal{M}\kappa(J_-,L_vJ_+)+\frac{\ii}{2}\int_\mathcal{M}\kappa(J_-,(\dd\omega+\dd\sigma)\cdot J_+)\\
            =& -\frac{\ii}{2}\int_\mathcal{M}\kappa(J_-,L_vJ_+)+2\ii  \int_\mathcal{M}\rho\kappa(\dd\omega+\dd\sigma,J_3),
    \end{aligned}
\end{equation}
where going to the second line we have used the observation in \eqref{eq:GengeometryLieDerivate} and going to the last line we used the invariance of the Killing form together with the fact that the hyper multiplet is a $\mathfrak{su}(2)$ triplet. Likewise we can define the plus component of the momentum map as 
\begin{equation}
    \begin{aligned}
    \mu_+(Y) &= -\ii\int_\mathcal{M}\kappa(J_3,\mathscr{L}_YJ_+)\\
          &= -\ii\int_\mathcal{M}\kappa(J_3,L_vJ_+)+2\int_\mathcal{M}\rho\kappa(\dd\omega+\dd\sigma,J_+),
    \end{aligned}
\end{equation}
with similar steps as above and it straightforward to define an analogous expression for $\mu_-$. These momentum maps $\mu_3,\mu_{\pm}$ should vanish for any generalised vector $Y=v+\tilde{\omega}+\sigma+\tau\in\Gamma(E)$ and since the momentum maps are linear we can consider each $GL(7)$ component separately, where we denoted the two-form $\tilde{\omega}$ to not cause confusion with the two-form $\omega$ in the structure. Note also that the component $\tau$ of $Y$ drops out of the momentum maps trivally. To continue we need the $\mathfrak{e}_{7(7)}$ Killing form given by 
\begin{equation}
    \kappa(W,W') = \frac{1}{2}\text{tr}(r)\text{tr}(r')+\text{tr}(rr')+\alpha \lrcorner a'+\alpha'\lrcorner a-\tilde{\alpha}\lrcorner \tilde{a}'-\tilde{\alpha}'\lrcorner \tilde{a}.
\end{equation}

Consider then
\begin{equation}
    \mu_+(\sigma) = 2\int_\mathcal{M}\rho\kappa(\dd\sigma,J_+) = 0,
\end{equation}
since $J_+$ does not contain any six-form. Likewise $L_vJ+$ contains a three-form and an anti-symmetric three-vector but $J_3$ does not so the using the Killing form we find 
\begin{equation}
    \mu_+(v) = 0.
\end{equation}
The only non-zero contribution from $\mu_+$ thus comes from the $\tilde{\omega}$ component $\kappa(\dd\tilde{\omega},\frac{1}{2}\rho\Omega^{\#})=\frac{1}{2}\rho\Omega^{\#}\lrcorner \dd\tilde{\omega}$ and we thus get the momentum map
\begin{equation}
    \mu_+(\tilde{\omega})\propto \int_\mathcal{M}\rho^2\Omega^{\#}\lrcorner \dd\tilde{\omega}.
\end{equation}
Using that 
\begin{equation}
\int_\mathcal{M}\text{vol}_7 \Omega^{\#}\lrcorner \dd\tilde{\omega} =  \int_\mathcal{M}\dd\tilde{\omega}\wedge \star\Omega=\ii\int_\mathcal{M}\dd\tilde{\omega}\wedge \Omega\wedge\zeta
\end{equation}
and integration by parts this reduce to 
\begin{equation}
    \mu_+(\tilde{w}) \propto \int_\mathcal{M}\dd(\Omega\wedge\zeta)\wedge\tilde{\omega} = 0,
\end{equation}
which indeed is fulfilled since $\Omega$ and $\zeta$ are closed. Considering the $\mu_3$ momentum map we find 
\begin{equation}
    \mu_3(\tilde{\omega}) = 2\ii\int_\mathcal{M}\rho\kappa(\dd\tilde{\omega},J_3) = 0,
\end{equation}
sisnce $J_3$ does not contain any anti-symmetric three-vector. For the action of the six-form we find 
\begin{equation}
    \begin{aligned}
    \mu_3(\sigma) &= 2\ii\int_\mathcal{M}\rho\kappa(\dd\sigma,J_3)\\
                  &=-\frac{1}{8}\int_\mathcal{M}\rho^2(\Omega^{\#}\wedge\Omega^{\#})\lrcorner \dd\tilde{\sigma}.
    \end{aligned}
\end{equation}
Using the same procedure as above we find
\begin{equation}
    \int_\mathcal{M}\text{vol}_7 (\Omega^{\#}\wedge \overbar{\Omega}^{\#})\lrcorner \dd\sigma = \int_\mathcal{M} \dd\sigma\wedge \star(\Omega\wedge\overbar{\Omega}) \propto \int_\mathcal{M} \dd \zeta\wedge \sigma = 0,
\end{equation}
which indeed is fulfilled since $\zeta$ is closed. The last condition from the triplet of momentum maps is the following 
\begin{equation}
    \begin{aligned}
    \mu_3(v) &= -\frac{\ii}{2}\int_\mathcal{M}\kappa(J_-,L_vJ_+)
             &= \frac{\ii}{8}\int_\mathcal{M}\rho \left(\overbar{\Omega}^{\#}\lrcorner L_v(\rho \Omega)+L_v(\rho\Omega^{\#})\lrcorner \overbar{\Omega}\right).
    \end{aligned}
\end{equation}
Since $\Omega$ is closed it is also covariantly constant and $\overbar{\Omega}^{\#}\lrcorner \Omega$ is constant. We can then use $\int_\mathcal{M}\overbar{\Omega}^{\#}\lrcorner \Omega L_v\rho^2=0$, assuming the boundary term vanishes, to move $\rho$ outside of the Lie derivative and using $\rho^2\overbar{\Omega}^{\#}\lrcorner L_v \Omega=L_v \Omega\wedge \star\overbar{\Omega}$ 
\begin{equation}
    \begin{aligned}
    \mu_3(v) &\propto \int_\mathcal{M}L_v\Omega\wedge\star\overbar{\Omega}\\
             &=-\ii\int_\mathcal{M}(\dd\iota_v\Omega+\iota_v\dd\Omega)\wedge\overbar{\Omega}\wedge\zeta=0,
    \end{aligned}
\end{equation}
using integration by parts and that $\Omega$ and $\overbar{\Omega}$ are closed. 
 
The integrability condition for the vector multiplet structure is given by $\mathscr{L}_X\overbar{X}=0$ and thus
\begin{equation}
    \begin{aligned}
    \mathscr{L}_X\overbar{X} &= L_{\zeta^{\#}}\overbar{X}-\ii\dd\omega\cdot \overbar{X}-(\zeta\wedge\omega\wedge\dd\omega)\cdot \overbar{X}.
    \end{aligned}
\end{equation}
This can be seen to vanish immediately by noting that $\zeta^{\#}$ is a Killing vector and using that $\dd\omega$ is a closed form. 

In order to define an exceptional Calabi-Yau the structures, i.e.\ to set to zero the intrinsic torsion of the $SU(6)$ structure, they must also satisfy $\mathscr{L}_XJ_+=\mathscr{L}_XJ_-=0$. This is given by 
\begin{equation}
    \begin{aligned}
    \mathscr{L}_X J_+ &= L_{\zeta^{\#}}J_+-\ii\dd\omega\cdot J_++\frac{1}{2}\dd(\zeta\wedge\omega\wedge\omega)\cdot J_+,
    \end{aligned}
\end{equation}
which again is easily seen to be fulfilled by the same reasoning as above. 

We thus see that the hyper multiplet structure in \eqref{eq:HyperMultipletStructureCalabiYau} and the vector multiplet structure in \eqref{eq:VectorMultipletStructureCalabiYau} together with the compatibility conditions and integrability conditions reproduce the known conditions obtained via ordinary results. Although fluxes were not included in this example the formalism applies equally well when they are, for examples of this see \cite{Ashmore:2015joa,Ashmore:2016qvs}. The main difference when including fluxes is that one has to look at elements that are ``twisted'' by the adjoint action of flux fields, the twisting in the case above were trivial. 












\begin{comment}

\subsubsection{Integrability in terms of momentum maps}
As claimed above finding a $G$-compatible connection with vanishing torsion could be written in terms of momentum maps. The reason for wanting to do such a rewriting is that it will allows to examine the moduli space of structure in a clever way. 

We introduce momentum maps in terms of an example of Hamiltonian physics. Given a phase space $X$ of dimension $2n$ there exists a symplectic two-form $\omega$ given by
\begin{equation}
    \omega = \dd p_i\wedge\dd q^i,
\end{equation}
where $q^i$, and $p_i$ are position and momenta respectively and are coordinates on $X$. Given a Hamiltonian function $H: X\to\mathbb{R}$ we can construct a corresponding vector field $X_H$ given by 
\begin{equation}
    \ii _{X_H}\omega = -\dd H.
\end{equation}
The flow of $X$ along $X_H$ should be $0$ in order to preserve the energy which is immediate by 
\begin{equation}
    \dd H (X_H) = -\omega(X_H,X_H) = 0. 
\end{equation}
This in turn implies that the symplectic two-form is preserved by action of the hamiltonian vector field $X_H$ 
\begin{equation}
    \mathcal{L}_{X_H} \omega = \ii_{X_H}\dd \omega + \dd\ii_{X_H} \omega = 0,
\end{equation}
where we used Cartan's identity, $\dd\omega=0$ and that $\ii_{X_H}\omega$ by construction is exact. 

Now consider an arbitrary manifold $\mathcal{M}$ with a closed non-degenerate two-form, i.e.\ a symplectic form $\Omega$. We also define an action of a Lie group $G$ on the phase space $\mathcal{M}$
\begin{equation}
    \mathcal{M}\times G\to \mathcal{M} \qquad g \times m\mapsto g\cdot m.
\end{equation}
This induces an action of the algebra $\mathfrak{g}$ (assume $G$ is connected or restrict to its connected component)
\begin{equation}
    \mathcal{M}\times \mathfrak{g} \to T_m \mathcal{M}\qquad \frac{\dd }{\dd t}\exp (tg)\cdot m|_{t=0} =X_g,
\end{equation}
and we thus see that any element in $g\in\mathfrak{g}$ induces a vector field $X_g$. The symplectic two-form $\Omega$ transforms under the action of $g$ by the Lie derivative with respect to this vector field $\mathcal{L}_{X_g}$ as $\delta_g \Omega = \mathcal{L}_{X_g}\Omega$ and we find
\begin{equation}
    \delta_g \Omega = \dd \ii_{X_g} \Omega + \ii_{X_g}\dd\Omega.
\end{equation}
Since $\Omega$ is closed by assumption we find that $\Omega$ is preserved by the action of $G$ if 
\begin{equation}
    \dd \ii_{X_g}\Omega = 0,
\end{equation}
the group $G$ is then said to act as a symplectomorphism on $\mathcal{M}$. A momentum map $\mu: \mathcal{M}\to\mathfrak{g}^*$ is defined by
\begin{equation}
    \dd \mu (g) = \ii_{X_g}\Omega,
\end{equation}
such that the following diagram commutes 
\begin{center}
\begin{tikzpicture}
  \matrix (m) [matrix of math nodes,row sep=3em,column sep=4em,minimum width=2em]
  {
     \mathcal{M} & \mathfrak{g}^* \\
     \mathcal{M} & \mathfrak{g}^* \\};
  \path[-stealth]
    (m-1-1) edge node [left] {$G$} (m-2-1)
            edge [] node [below] {$\mu$} (m-1-2)
    (m-2-1.east|-m-2-2) edge node [below] {$\mu$}
             (m-2-2)
    (m-1-2) edge node [right] {$G$} (m-2-2)
            (m-2-1);
\end{tikzpicture}
\end{center}
Note that $G$ acts on $\mathfrak{g}^*$ in the coadjoint representation such that $X\in G$ acts as $\text{Ad}_X^*\tilde{g}(g)=\tilde{g}(\text{Ad}_{X^{-1}}g)$, with $g\in \mathfrak{g}$ and $\tilde{g}\in \mathfrak{g}^*$. That the diagram above commutes is the statement that $\mu$ is $G$-equivariant i.e.\
\begin{equation}
    \mu(X\cdot m)\cdot g = \text{Ad}_{X}^*(\mu(m))\cdot g = \mu(m)\cdot(\text{Ad}_{X^{-1}}g). 
\end{equation}
The main reason for introducing momentum maps in our case is that we can define the symplectic quotient as 
\begin{equation}
    \mathcal{M}\sslash G := \mu^{-1}(0)/G = \{[m]\in \mathcal{M} : m\sim g\cdot n, m,n\in \mathcal{M}, g\in G\},
\end{equation}
which is always well-defined as a topological quotient but need not be a manifold in general. The choice of $0\in\mathfrak{g}^*$ ensures that the action of $G$ is free and if we moreover assume that it the action is proper and $0$ is regular value then there is a theorem by Marsden-Weinstein-Meyer stating that 
\begin{itemize}
    \item $\mathcal{M}\sslash G$ is a manifold,
    \item it also inherits a unique symplectic form from $\Omega$. 
\end{itemize}
Note that this could be generalised slightly by replacing $0\in\mathfrak{g}^*\to \tilde{g}\in\mathfrak{g}^*$ and replacing $G$ with the stabiliser of $\tilde{g}$.  
%\todo{https://userpage.fu-berlin.de/hoskins/Sympl_reduction.pdf }

\subsubsection{Momentum map hypermultiplet structure}
Consider the hypermultiplet structure $J_\alpha\in\Gamma(\text{ad}\, F\otimes (\text{det}\, T^*\mathcal{M})^{1/2})$ which defines the $Spin^*(12)\times SU(2)$ structure in terms of a triplet of symplectic forms. As such at each point $m\in\mathcal{M}$ the structure is given by a point in the coset space, which is shown explicitly in \cite{Grana:2009im},
\begin{equation}
    J_\alpha|_x \in F_H = E_{7(7)}\times\mathbb{R}^+/Spin^*(12).
\end{equation}

As such we can consider the following bundle
\begin{center}
\begin{tikzpicture}
  \matrix (m) [matrix of math nodes,row sep=3em,column sep=4em,minimum width=2em]
  {
     E_{7(7)}\times\mathbb{R}^+/Spin^*(12) & P_H \\
                 & \mathcal{M} \\};
  \path[-stealth]
    (m-1-1) edge node [below] {} (m-1-2);
    \path[-stealth]
    (m-1-2) edge node [below] {} (m-2-2);
\end{tikzpicture}
\end{center}
and find that the space of all hypermultiplet structure $S_{H}$ is given by the space of sections of $P_H$, $S_H = \Gamma(P)$. A structure is then given by a point $s\in S_H$ and small deformations of the structures can be considered as vectors in the tangent space $T_sS_H$. On the other a structure can similarly be thought of as a function on $S_H$ valued in $\text{ad}\, F\otimes (\text{det}\, T^*\mathcal{M})^{1/2}$
\begin{equation}
    J_\alpha: S_H \to \text{ad}\, F\otimes (\text{det}\, T^*\mathcal{M})^{1/2},
\end{equation}
on which the tangent vector $v_\alpha\in\Gamma(T_{J_\alpha}S_H)$ acts as 
\begin{equation}
    v_\alpha = i_v \delta J_\alpha,
\end{equation}
where $\delta J_\alpha$ denotes the exterior derivative of on the space of sections. With this we can define a triplet of symplectic forms that characterise the geometry 
\begin{equation}
    \Omega_\alpha(v,w) = \epsilon_{\alpha\beta}\int_\mathcal{M} \kappa(v,w),
\end{equation}
where $\kappa$ is the Killing form on $\mathfrak{e}_{7(7)}$ and $v,w\in T_s S_H$ at some point $s\in S_H$. Note that $\kappa(v,w)\in \text{det}\, T^*\mathcal{M}$ and can thus be integrated over.

In order to put some differential condition on the structures in a covariant manner we look at the deformation due to generalised diffeomorphisms. These are parameterised by an element $V\in \Gamma(E)$ and the action is given by the generalised Lie derivative 
\begin{equation}
    \delta_V J_\alpha = L_V J_\alpha. 
\end{equation}
Since $\delta J_\alpha$ is an element in the tangent space of $S_H$ we can define a mapping from the algebra of generalised diffeomorphisms $\mathfrak{gdiff}$ to a vector field on $S_H$ defined through its action on the structure $J_\alpha$ as 
\begin{align}
    X: \mathfrak{gdiff}&\to \Gamma(TS_H)\\
        V&\mapsto X_V \qquad\qquad  \text{s.t.}\qquad X_V(J_\alpha)=L_VJ_\alpha.
\end{align}
Consider now the triplet of symplectic form evaluated on $X_V$
\begin{align}\label{eq:sympev}
    \Omega_\alpha (X_V,v) &= \epsilon_{\alpha\beta\gamma}\int_\mathcal{M}\kappa(v_\beta,L_VJ_\gamma).
\end{align}
Since $\kappa(v_\beta,L_VJ_\gamma) \in \text{det}\, T^{*}M$ it follows that \todo{CHECK THIS}
\begin{equation}
    \int_\mathcal{M} L_V \left(\kappa(v_\beta,L_VJ_\gamma)\right) = \int_\mathcal{M}\pd_i \left(v^i\kappa(v_\beta,L_VJ_\gamma)\right) = 0,
\end{equation}
where $v\in\Gamma(TM)$. By the Leibniz property of the generalised derivative we can thus rewrite \eqref{eq:sympev} as 
\begin{equation}
    \Omega_\alpha(X_V,v) = \frac{1}{2}\epsilon\int_\mathcal{M}\left[\kappa(v_\beta,L_VJ_\gamma)+\kappa(L_Vv_\gamma,J_\beta)\right],
\end{equation}
where we split the original integrand into two and relabeled some dummy indices. This more symmetrical form allow us to rewrite this expression as the variation of a triplet of momentum maps $\mu_\alpha$ defined as 
\begin{equation}
    \mu_\alpha = \frac{1}{2}\epsilon_{\alpha_\beta\gamma}\int_\mathcal{M}\kappa(J_\beta,L_VJ_\gamma).
\end{equation}
Using the exterior derivative on $S_H$ which is defined as $i_v\delta J_\alpha=v_\alpha$, the variation of the momentum maps is given by 
\begin{equation}
    i_v\delta\mu_\alpha = \frac{1}{2}\epsilon_{\alpha\beta\gamma}\int_{\mathcal{M}}\left[\kappa(v_\beta,L_VJ_\gamma)+\kappa(J_\beta,L_Vv_\gamma)\right].
\end{equation}

In order to find a torsion-free $G$-compatible connection the corresponding intrinsic torsions have to vanish. We will now show that the momentum maps contain precisely the intrinsic torsion and that vanishing of the momentum maps thus imply vanishing intrinsic torsion. By definition of torsion the generalised Lie derivative can be rewritten as $L_V J_\alpha = L_V^DJ_\alpha-[T(V),J_\alpha]$ and hence the momentum maps are given by 
\begin{equation}
    \mu_\alpha = \frac{1}{2}\epsilon_{\alpha\beta\gamma}\int_\mathcal{M}\left(\kappa(J_\beta,V\cdot DJ_\gamma)+\kappa(J_\beta,(D\times_{\text{ad}} V)J_\gamma)-\kappa(J_\beta,[T(V),J_\gamma])\right).
\end{equation}
The first term in this expression vanish by definition of a $G$-compatible connection.  

\subsubsection{Momentum map vector multiplet structure}
Consider the vectormultiplet structures $V$ and $\hat{V}$ which define the $E_{6(2)}$ sub bundle. Then at each point in $m\in\mathcal{M}$ the choice of either $V$ or $\hat{V}$ is equivalent to a point in the coset space \cite{Grana:2009im}
\begin{equation}
    V|_m \in F_V = E_{7(7)}\times\mathbb{R}^+/E_{6(2)},
\end{equation}
and similarly for $\hat{V}$. As in the hypermultiplet case we thus consider the bundle 
\begin{center}
\begin{tikzpicture}
  \matrix (m) [matrix of math nodes,row sep=3em,column sep=4em,minimum width=2em]
  {
     E_{7(7)}\times\mathbb{R}^+/E_{6(2)} & P_V \\
                 & \mathcal{M} \\};
  \path[-stealth]
    (m-1-1) edge node [below] {} (m-1-2);
    \path[-stealth]
    (m-1-2) edge node [below] {} (m-2-2);
\end{tikzpicture}
\end{center}
such that the space of vector multiplet structures $S_V$ is given by the space of sections of $S_V=\Gamma(P_V)$. Deformations of the structure are again generated by the generalised Lie derivative 
\begin{equation}
    \delta_W V = L_WV,
\end{equation}
where $W\in\Gamma(E)$. Likewise this defines a vector field on $S_V$, i.e. a section of $TS_V$ as for the hypermultiplet case 
\begin{align}
    X_W: \mathfrak{gdiff}&\to \Gamma(TS_V)\\
        W&\mapsto X_W \qquad\qquad  \text{s.t.}\qquad X_W(V)=L_WV.
\end{align}
With this construction we define a symplectic form also for the vector multiplet structure, if $Z\in\Gamma(TS_V)$ and $X_W\in\Gamma(TS_V)$ as defined above then the symplectic form is defined as  
\begin{equation}
    \Omega(X_W,Z) =\int_\mathcal{M}s(X_W,Z),
\end{equation}
where $s$ is the symplectic invariant of $E_{7(7)}$. We then define the momentum map for the vector multiplet structure as 
\begin{equation}
    \mu(W) := \frac{1}{2}\int_\mathcal{M}s(L_WV,V). 
\end{equation}
Using the exterior derivative on $S_V$ we have 
\begin{equation}
    i_Z\delta\mu(W) = \frac{1}{2}\int_\mathcal{M}\left(s(L_WZ,V)+s(L_WV,Z)\right),
\end{equation}
which by the anti-symmetry of the argument of $s$ and using the Leibniz property of $L$ reduce to $\Omega(X_W,Z)$. 

\end{comment}
%As mentioned several times, preserving supersymmetry of the vacuum corresponds to two conditions, one topological and one differential. To start with we consider the constraints that follows from the existence of a globally defined nowhere vanishing spinor. First we note the existence of a generalised metric on $E_{7(7)}\times\mathbb{R}^+$ allows us to introduce an $\tilde{H}$- structure where $H=SU(8)$ is the double cover of the maximal compact subalgebra of $E_{7(7)}$. This is important since spinors then transforms under  
%\begin{equation}
%    Spin(1,3)\times SU(8).
%\end{equation}
%In terms of bundles, the gravitino is a section of the spin-bundle $\Gamma(J)$ and the dilatino and the supersymmetry parameter $\epsilon$ are sections of $\Gamma(S)$ with 
%$J=\mathbf{56}$ and $S=\mathbf{8}$ of $SU(8)$. Under $Spin(7)\subset Spin(8)\subset SU(8)$ these decompose as 
%\begin{equation}
%    \mathbf{8}\to \mathbf{8}, \qquad  \mathbf{56}\to \mathbf{8}+\mathbf{48},
%\end{equation}
%of $Spin(7)$. 




\chapter{Summary}
String theory/M-theory are the most prominent frameworks of quantum gravity but there are still much about these theories that we do not know. Especially the notion of describing extended objects have led to many interesting physical phenomena such as $T$-duality and $U$-duality as well as many interesting mathematical results. As we have argued the corresponding low-energy effective theories have ``hidden'' symmetry groups when compactified on torii. The appearance and rôle of these symmetries in M-theory is the motivation for thesis and the idea behind extended geometries. 

Gravitational degrees of freedom and non-gravitational degrees of freedom are combined in order to make the hidden duality groups manifest which calls for the extension of ordinary geometry. This is done by introducing generalised diffeomorphism which governs and consistency of such transformations together with demanding that the action invariant under such transformations determines a theory ox extended geometries. Repeatedly we have seen the requirement of the section condition in order to define a consistent theory which further specifies the connection with ordinary geometry. 

Generalised geometry introduced in this thesis is another approach of generalising geometry in order to make the duality groups manifest. This formalism was used to study some parts of supersymmetric flux compactifications which appears natural in this setting due to unification of gravity with gauge fields. Including fluxes in a compactification typically make the useful tools of ordinary geometry less effective but with appropriate generalisation of these such compactifications turn out more tractable. 

In chapter \ref{chap:ExtendedGeometries} a covariant formalism was introduced and a generalised Ricci tensor was introduced. This is a candidate for the equations of motion for the generalised metric and it would be interesting to compare this with the equations of motion derived from the action introduced in the same chapter. In order to do this one should find a metric-compatible and torsion-free connection in terms of generalised metric. However, these two conditions do not fully specify the connection and we were not able to find such an expression in the general case. This would be interesting for future studies. Moreover, one would need to make sure that the expression of the generalised Ricci tensor does not contain the non-determined parts of the connection. 


\chapter{Flux compactifications and (exceptional) generalised geometry\label{chap:FluxCompactifications}}

\section{Phenomenology and moduli stabilization}
As is well-knwon string theory/M-theory lives in a 10/11-dimensional spacetime, yet our everyday experience tells us otherwise. Hence, in order for string theory/M-theory to be able to describe the real world it has to have consistent low-energy description that matches our observations. 

Firstly such a theory should reproduce the standard model of particle physics which has been confirmed to great precision, i.e.\ among other things it should be described by a gauge theory with gauge group $SU(3)\times SU(2)\times U(1)$ together with the correct particle spectra. Another important notion is that of supersymmetry, notably the SM is not supersymmetric in contrast to string theory/M-theory. The latter two are maximally supersymmetric and thus most, perhaps all, supersymmetry needs to be spontaneously broken. Generically supersymmetry constrains a theory often yielding important physical consequences.

The idea of compactification, where a compact internal manifold is ``small'' compared to the typical size of say strings, is one way to obtain an effective theory in lower dimensions. A generic feature of compactification is presence of scalar fields typically due internal legs of some tensor field. Of especial importance are massless such scalar fields, called \emph{moduli}. The vacuum expection value of moduli is not fixed and appears as free parameters in the theory. Since there possibly is hundred or thousands of such moduli which determine physical properties such as gauge couplings, Yukawa couplings as well as the size and structure of the internal manifold the theory loses most of its predictive power. Moreover, massless scalar fields would give rise to a long-range force which would invalidate Einstein's equivalence principle. Of course the coupling could be extremely small such that these forces are not experimentally observable. However, this approach comes with a lot of problems \todo{What problems?} and instead one tries to give mass to these scalar fields, this is called \emph{moduli stabilization}. Generating a potential for the scalar fields would imply that the moduli fields are no longer free parameters but rather fixed by the procedure that stabilises the moduli, e.g.\ complicated geometry or presence of fluxes. 

The easiest internal manifold to compactify on is a torus but such a compactification preserves all the supersymmetry, i.e.\ $\mathcal{N}=8$ in four external dimensions. In order to break some supersymmetry we could abandon the torus and compactify on something more complicated, e.g.\ a Calabi-Yau three-fold which we will discuss further below. Another possibility is to introduce to some non-zero flux for some form field 
\begin{equation}
    \int_{\Sigma_{p+1}}C_{p+1}\neq 0,
\end{equation}
corresponding to a $p$-brane winding some non-trivial cycle $\Sigma_{p+1}$ in the internal manifold. Such a configuration schematically generates a potential for the moduli through the kinetic term of such a form field 
\begin{equation}
    V\sim \int_{\mathcal{M}_{\text{int}}} \dd C_{p+1}\wedge \star_{g}\,\dd C_{p+1}, 
\end{equation}
where $\mathcal{M}_\text{int}$ denotes the internal manifold and the presence of the metric moduli in the hodge dual $\star$ is the reason for a potential being generated by expanding $C_{p+1}$ around its vev. 

\todo[inline]{String landscape}

\todo[inline]{Cosmology}

\section{Zero flux compactification}
To begin with we will start by reviewing compactifications on backgrounds without fluxes. The idea is to find vacua (solutions to the equations of motion and the Bianchi identities) such that spacetime seperates into an external spacetime $\mathcal{M}_{\text{ext}}$ and an internal spacetime $\mathcal{M}_{\text{int}}$ such that 
\begin{equation}
    \mathcal{M} = \mathcal{M}_{\text{ext}}\times \mathcal{M}_{\text{int}}.
\end{equation}
However, finding such solutions by solving the equations of motion are generally rather difficult and we will take a different route. Supergravity in $D=11$ has $\mathcal{N}=1$ local supersymmetry, this is the maximal amount of supersymmetry with a total of $32$ supercharges. A generic vacuum solution however will (spontaneuously) break all of this supersymmetry. By restricting to solutions that preserve some amount of supersymmetry, say $\mathcal{N}=1$ in $D=4$, we will find constraints that give a more tractable way to finding vacua. Moreover, the external spacetime is assumed to be Minkowski with possible generalisation to AdS later on. 

Supersymmetry is generated by a supercharge $\mathcal{Q}$ and a parameter $\epsilon$, both being spinors of $SO(1,D-1)$ and a vacuum $|0\rangle$ is supersymmetric if it is annihilated by the supersymmetry transformation
\begin{equation}
    \overbar{\epsilon}\mathcal{Q}|0\rangle = 0.
\end{equation}
By taking an arbitrary field $\Theta$ and looking at its variation under the supersymmetry transformation we find 
\begin{equation}
    \langle \delta_{\epsilon}\Theta\rangle = \langle 0|\left[\overbar{\epsilon}\mathcal{Q},\Theta\right]|0\rangle = 0.
\end{equation}
Generally we have that (from now variations should be understood implicitly as valid as expectation values)
\begin{equation}
    \delta_\epsilon \Theta_{\text{boson}} \sim \overbar{\epsilon}\Theta_{\text{fermion}},\qquad \delta_\epsilon \Theta_{\text{fermion}}\sim \epsilon\Theta_{\text{boson}}.
\end{equation}
By the assumption that $\text{dim}\,\mathcal{M}{_{\text{ext}}}=d$ being a $d$-dimensional Minkowski space we have to decompose the spinor representations under $SO(1,d-1)\times SO(D-d)$ where $D$ is the dimension of the full spacetime. Typically these will transform in a non-trivial representation under $SO(1,d-1)$ and hence have to vanish in order to preserve Poincaré symmetry. Thus in order to make sure that the vacuum preserve some amount of supersymmetry we need to find solutions of $\delta_\epsilon\Theta_{\text{fermions}}=0$. Note if such a solution is found we also have to make sure the equations of motion and Bianchi identities are fulfilled. 

With a constant dilaton and vanishing fluxes we find for the supersymmetry variation of the gravitino $\psi_M$ (that is always present in supergravity)
\begin{equation}
    \delta_\epsilon\psi_M = \nabla_M\epsilon = 0 \Leftrightarrow \nabla_\mu \epsilon = 0,\qquad \nabla_m\epsilon = 0,
\end{equation}
where we have used that there is no mixing between greek external indices and latin internal indices in the connection due to the metric being of a product form. A spinor satisfying these condition is called a \emph{Killing spinor}. From this one can show that a necessary condition for the existence of a covariantly constant spinor is that the internal manifold is Ricci-flat (obviously also the external spacetime is Ricci-flat)
\begin{equation}
    R_{mn} = 0.
\end{equation}

Curvature is closely connected to the way an object transforms upon parallel transport. As such take some object $v$ and parallel transport it in a closed loop on a manifold. Generally the object transforms in some representation of $\mathscr{H}\subseteq O(d)$ under parallel transport for a d-dimensional riemannian manifold, since the norm does not change assuming a metric compatible connection. The group $\mathscr{H}$ is called the holonomy group. Likewise a spinor parallel transported in a loop transforms in a representation of $\mathscr{H}$ and hence, especially, the Killing spinor. But since the Killing spinor is covariantly constant it does not transform and needs to be a singlet under the holonomy group. As such a necessary condition for the existence of a covariantly constant spinor is related to the restriction of the holonomy group which put constraints on the metric. Such features as reduction of holonomy and structure group is conveniently described in terms of bundles briefly explained below.

\subsection{Mathematical interlude -- Fibre bundles}
A short introduction to the notion of fibre bundles are introduced in order to set the terminology which will be convenient for more complicated compactifications later on. Just as a manifold is a topological space that looks locally like $\mathbb{R}^n$, i.e.\ there exists some coordinate chart on $U_i\subseteq \mathcal{M}$ such that $\phi: \mathcal{M}\to \mathbb{R}^n$, a bundle is a topological space that looks locally like a product of two topological spaces. In the cases we are interested these topological spaces will be differentiable manifolds. 

A bundle is a triple $(P,\mathcal{M},\pi)$ typically also denoted $P\overset{\pi}{\longrightarrow}\mathcal{M}$, where $P$ is the total space, $\mathcal{M}$ the base space and $\pi: P\to \mathcal{M}$ the projection which is a surjection onto the base space. The notion of a bundle is a very general concept and we will instead only be interested in (differentiable) fibre bundles. A (differentiable) fibre bundle $(P,\mathcal{M},\pi,F,G)$ where $F$ is the fibre and $G$ the structure group. In order for a fibre bundle to ``look locally'' like a product manifold there exists on patches $U_i\subseteq \mathcal{M}$ diffemorphisms $\phi_i: U_i\times F\to \pi^{-1}(U_i)$ such that $\pi\circ \phi_i(u,f)=u$, where $u\in U_i$, $f\in F$ and $\pi^{-1}(U_i)=F_p\cong F$ is the fibre at $u\in U_i$. The maps $\phi_i$ are usually called local trivializations since the map $\phi^{-1}\circ \pi^{-1}$ maps any $U_i$ to $U_i\times F$ which ``looks like a product manifold''. In order for this to make sense there needs to be some compatibility on overlapping patches $U_i\cap U_j\neq \emptyset$. Take a point $p\in P$ such that $\pi(p)=u$ with $u\in U_i\cap U_j$ then applying the two local trivialisations $\phi^{-1}_i$ and $\phi^{-1}_j$ we find two points in $P$ according to 
\begin{equation}
    \phi^{-1}_i(u) = (u,f_i)\qquad \text{and}\qquad \phi^{-1}_j(u) = (u,f_j).
\end{equation}
Obviously $f_i = \phi^{-1}_i\circ \phi_j (f_j)$ where $\phi^{-1}_i\circ \phi_j = t_{ij}(u)$ is called a transition function $t_{ij}(u):F\to F$ defined trough a left action of the structure group $G$.\todo{left action of G} 

A principal fibre bundle ($P$,$\mathcal{M}$,$\pi$,$G$) is a fibre bundle with a typical fibre being the structure group $G$ itself, i.e.\ for some $u\in U\subseteq \mathcal{M}$ $\pi^{-1}(u)\cong G$. Moreover, there is a right action of $G$
\begin{equation}
    \triangleleft: P\times G \to P,
\end{equation}
which is free. An important example of a principle fibre bundle is the frame bundle. The frame bundle is a fibre bundle with a typical fibre consisting of all ordered basis, such a basis being a frame, of the tangent space. 

The existence of invariant tensors on $\mathcal{M}$ is equivalent to adding further structure to a principal $G$-bundle such that it is possible to impose a restriction to a principal $H$-bundle, where $H\subset G$ with the same base manifold. Typically this is defined the other way around,  


In order to define a reduced structure group we need a so-called bundle map. A bundle map is a linear map $\rho: P\to \overbar{P}$ and continuous map between the base manifolds $f: \mathcal{M}\to \overbar{\mathcal{M}}$ such that the following diagram commutes
\begin{center}
\begin{tikzpicture}
  \matrix (m) [matrix of math nodes,row sep=3em,column sep=4em,minimum width=2em]
  {
     P & \overbar{P} \\
     \mathcal{M} & \overbar{\mathcal{M}} \\};
  \path[-stealth]
    (m-1-1) edge node [left] {$\pi$} (m-2-1)
            edge [] node [above] {$\rho$} (m-1-2)
    (m-2-1.east|-m-2-2) edge node [below] {$f$}
            node [above] {} (m-2-2)
    (m-1-2) edge node [right] {$\overbar{\pi}$} (m-2-2);
\end{tikzpicture}
\end{center}

\todo{Bundle maps, when reduced structure group, Berger's list}


\subsection{Compactification on Calabi-Yau}
Having reviewed the notion of reduced structure group and special holonomy a Calabi-Yau n-fold $CY_{n}$ can be defined as a $2n$-dimensional compact Riemannian manifolds with holonomy contained in $SU(N)$\cite{Blumenhagen2013}. We are particularly interested in $CY_3$ such that starting in $D=10$ supergravity, or string theory, the external spacetime is 4-dimensional. As such we know that for a $CY_3$ there exists some p-forms invariant under $\mathscr{H}=SU(3)$ which can be found by looking at the decomposition of 1-, 2- and 3-forms of $SO(6)\cong SU(4)$
\begin{align*}
    \mathbf{6}\to \mathbf{3}+\mathbf{\overbar{3}},\\
    \mathbf{15}\to \mathbf{8}+\mathbf{3}+\overbar{\mathbf{3}}+\mathbf{1},\\
    \mathbf{20}\to \mathbf{6}+\overbar{\mathbf{6}}+\mathbf{3}+\overbar{\mathbf{3}}+2\cdot\mathbf{1}.
\end{align*}
There thus exists a real 2-form $\omega$ and a complex 3-form $\Omega$ that are singlets under $\mathscr{H}$, such pair of nowhere vanishing tensors defines a \emph{structure}. Note that we already know that such a structure is equally well defined by a nowhere vanishing globally defined spinor $\mathbf{4}$ of $SU(4)\cong SO(6)$ that decomposes under $\mathscr{H}$ as 
\begin{equation}
    \mathbf{4}\to \mathbf{3}+\mathbf{1}.
\end{equation}
Explicitly, a Weyl-spinor $\psi$ of $SO(1,9)$ decomposed under $SO(1,3)\times SO(6)$ can be parameterized as 
\begin{equation}
    \psi = \eta_+\otimes\epsilon_++\eta_-\otimes\epsilon_-,
\end{equation}
where $\eta_\pm$ and $\epsilon_\pm$ are Weyl-spinors with eigenvalue $\pm 1$ under $SO(1,3)$ and $SO(6)$ respectively. Morever, imposing that $\psi$ is Majorana-Weyl in ten dimensions further implies that $\eta_+$ is the charge conjugate of $\eta_-$ and likewise for $\epsilon_\pm$.

What follows will mostly consider structures defined by some bosonic forms such as ($\omega,\Omega$) but from the observation above we know that this corresponds to the existence of a globally defined nowhere vanishing spinor. To preserve supersymmetry such a spinor also satifies a differential condition and hence also the structure $(\omega,\Omega)$ need to satisfy certain differential conditions. In fact, the structure $(\omega,\Omega)$ is related to the Killing spinor as 
\begin{align}
    \omega_{mn} = \mp\ii \epsilon^\dagger_\pm\gamma_{mn}\epsilon_\pm,\\
    \Omega_{mnp} = -\ii\epsilon^\dagger_-\gamma_{mnp}\epsilon_+.
\end{align}
Taking the covariant derivative of these definitions we find that the Killing spinor equations in terms of the structure $(\omega,\Omega)$ is simply the statement that they are both closed forms 
\begin{equation}
    \dd \omega = 0, \qquad \dd \Omega = 0.
\end{equation}
This condition is usually called integrability condition and is necessary to obtain a consistent patching of the structure. Note that transformations that preserves the non-degenerate two-for in $d-$dimensions are by definition elements in $Sp(d/2,\mathbb{R})$. 

Moreover, we see that starting with $\mathcal{N}$ supersymmetries in the full theory compactification on a $CY_3$ yields a theory which again has $\mathcal{N}$ supersymmetries.

\section{Flux compactifications and generalised geometry}
In the case of vanishing flux we have seen that the Killing spinor equation led to special holonomy, which in put restrictions on the connection of the internal manifold. In the presence of fluxes the Killing spinor is no longer covariantly constant, however, the existence of globally defined nowhere vanishing spinor still implies a reduced structure group. This is summarized by 

\begin{align*}
    \text{vanishing flux}&\Longleftrightarrow \text{special holonomy},\\
    \text{non-zero flux}&\Longleftrightarrow \text{reduced structure group}.\\
\end{align*}
In the presence of fluxes the Killing spinor equations of type II supergravity are given by \cite{Blumenhagen2013}
\begin{equation}
    \delta_\epsilon \psi_M = \left(\nabla_M+\frac{1}{8}\Gamma^{NP}H_{MNP}\mathscr{P}+\frac{1}{16}\ee^\Phi\sum_{n}\slashed{F}_n\Gamma_M\mathscr{P}_n\right)\epsilon,
\end{equation}
\begin{equation}
    \delta_\epsilon \lambda = \left(\slashed{\pd}\Phi+\frac{1}{12}\Gamma^{MNP}H_{MNP}\mathscr{P}+\frac{1}{8}\ee^\Phi\sum_{n}(-1)^n(5-n)\slashed{F}_n\mathscr{P}_n\right)\epsilon,
\end{equation}
where $\mathscr{P}$ and $\mathscr{P_n}$ are $2\times 2$-matrices, $\slashed{F}_n=\frac{1}{n!}\Gamma^{M_1M_2\ldots M_n}F_{M_1M_2\ldots M_n}$ and $n\in\{0,1,\ldots,10\}$ is even(odd) for type IIA(B). In the case of vanishing flux and a constant dilation this is seen to reduce to a covariantly constant spinor $\epsilon$ as discussed above. 

The obstruction to the Killing spinor being covariantly constant is typically discussed in terms of intrinsic torsion. However, there is always possible to construct a connection $\nabla$ with torsion such that the Killing spinor is covariantly constant w.r.t. to that connection. Thus consider a connection $\nabla=\nabla'-\kappa$ such that $\nabla_n V_m=\pd_nV_m-\tensor{\Gamma}{^'_{nm}^p}V_p-\tensor{\kappa}{_{nm}^p}V_p$, where $\nabla'$ is the Levi-Civita connection and $\kappa$ is the so-called contorsion tensor. Generically the connection coefficents $\tensor{\Gamma}{_{mn}^p}$ take values in $\Lambda^1\otimes\mathfrak{g}$, where $\lambda^1$ is the one-form representation of $G=GL(d)$ and $\mathfrak{g}=\mathfrak{gl}(d)$ is the corresponding Lie algebra. Assuming metric compatibility fixes the ``symmetric'' part of the pair $_m^p$ which is contained in the Levi-connection. Note that the Levi-Civita connection have extra terms such that the torsion $\tensor{\Gamma}{_{[mn]^p}}$ vanishes. This implies that the contorsion take values in 
\begin{equation}
    \tensor{\kappa}{_{mn}^p}\in \Lambda_1\otimes \mathfrak{h},
\end{equation}
where $\mathfrak{h}=so(d)$ is the maximal compact subalgebra of $\mathfrak{gl}(d)$ (since this is the ``anti-symmetric'' part). Assuming a reduced structure group $G'\subseteq SO(d)$ implies that we can separate the contorsion into two parts 
\begin{equation}
    \kappa = \kappa_0+\kappa_{\text{int}}\quad \text{where} \quad \kappa_{\text{int}}\in \Lambda_1\otimes \mathfrak{g}'\quad \text{and}\quad \kappa_{0}\in \Lambda_1\otimes \tensor{\mathfrak{g}}{^'^{\perp}}.
\end{equation}
The point now being that since the Killing spinor is a singlet under $G'$ the failure of the of Killing spinor being covariantly constant \emph{with respect to the Levi-Civita connection} is given by $\kappa_0$. The part $\kappa_0$ can further be decomposed into irreducible representations of $\mathfrak{g}'$ called torsion classes. As an example take $G=GL(6)$ and $G'=SU(3)$. Under $G'$ we have the following decompositions $\Lambda_1\to \mathbf{3}\oplus\overbar{\mathbf{3}}$, $\mathfrak{g}'=\mathbf{8}$ and $\tensor{\mathfrak{g}}{^'^{\perp}}=\mathbf{1}\oplus\mathbf{3}\oplus\overbar{\mathbf{3}}$ and 
\begin{equation}
    \kappa_0 \in \bigoplus_{i=1}^5 \mathscr{W}_i,
\end{equation}
where $\mathscr{W}_i$ are the five torsion classes given by 
\begin{align*}
    \mathscr{W}_1 = \mathbf{1}\oplus\mathbf{1},\\
    \mathscr{W}_2 = \mathbf{8}\oplus\mathbf{8},\\
    \mathscr{W}_3 = \mathbf{6}\oplus\overbar{\mathbf{6}},\\
    \mathscr{W}_4 = \mathbf{3}\oplus\overbar{\mathbf{3}},\\
    \mathscr{W}_5 = \mathbf{3}\oplus\overbar{\mathbf{3}}.\\
\end{align*}
Using complex geometry and cohomology one can show, see \cite{Blumenhagen2013} for details, that the exterior derivative of the structure $(\omega,\Omega)$ can be written in terms of torsion classes as 
\begin{equation}
    \dd \omega = \frac{3}{2}\text{Im}(\overbar{\mathscr{W}}_1\Omega)+\mathscr{W}_4\wedge \omega+\mathscr{W}_3
\end{equation}
and 
\begin{equation}
    \dd \Omega = \mathscr{W}_1\omega\wedge\omega+\mathscr{W}_2\wedge\omega+\overbar{\mathscr{W}}_5\wedge\Omega.
\end{equation}
We thus find that for all five torsion classes vanish for $CY_3$ in order for $(\omega,\Omega)$ to be closed. 



\section{Generalized geometry}
We have seen above that demanding some preserved supersymmetry in fluxless compactification led to two constraints:
\begin{itemize}
    \item Demanding existence of globally defined spinor led to a reduced structure group characterized by ($\omega,\Omega$)
    \item Imposing the Killing spinor equation led to a reduced holonomy. 
\end{itemize}
Including fluxes alters the Killing spinor equation and the holonomy group need no longer be reduced. This effectively weakens the constraints of the internal manifold which at the same time makes things more complicated. In the spirit of extended geometries we will try to generalize the notion of geometry in order to accommodate flux degree of freedoms in terms of what is called \emph{generalized geometry}. The reason for doing this being that it is then possible to define an appropriate generalisation of the holonomy, or rather of the structure together with some integrability conditions. 

In extended geometries we enlarged spacetime by introducing coordinates in some module of the underlying structure algebra $\mathfrak{g}$. In generalized geometry on the other hand one does not go that far, instead only the tangent space of the (internal) manifold is being enhanced. 

To start with consider a manifold $\mathcal{M}$ together with its tangent bundle $TM$ given by 
\begin{equation}
    TM = \bigcup_{x\in \mathcal{M}}\{(x,y)|x\in\mathcal{M}, y\in T_x\mathcal{M} \},
\end{equation}
where $T_x\mathcal{M}$ is the tangent space of $\mathcal{M}$ at the point $x\in\mathcal{M}$. A basis of the tangent space in local coordinates is simply given by ${\pd/\pd x^i}_{i=1,\ldots d}$, where $d=\text{dim}\,\mathcal{M}$. Now recall that in DFT spacetime is enlarged to a $2d$ dimensions by letting $x^i\to (x^i,\tilde{x}_i)$, where the latter is the vector representation of $O(d,d)$. In generalized geometry the manifold is not enlarged in the same way. Instead one considers only a generalised tangent space and, hence, also a generalized tangent bundle on which in the DFT case $O(d,d)\times \mathbb{R}^+$ acts naturally and in the exceptional case $E_{d(d)}\times\mathbb{R}^+$ acts naturally.  In the former case we define the generalised tangent bundle as 
\begin{equation}
    E\cong T\mathcal{M}\oplus T^*\mathcal{M},
\end{equation}
that is, at a point $x\in\mathcal{M}$ the typical fibre is the formal sum of a vector and a one-form. Moreover, patching between overlapping regions $U_i\cap U_j$ is done by elements in $O(d,d)\times\mathbb{R}^+$ being the structure group. Decomposing the DFT coordinate module under its $GL(d,\mathbb{R})$ subgroup one finds the vector representation as well as a one-form, corresponding to elements of the typical fibre in generalized geometry. There exists a natural inner product on sections $X^I=\xi^i+\lambda_i$, $Y^I=\eta^i+\rho_i$, where $X,Y\in\Gamma(TM\oplus T^*\mathcal{M})$, given by
\begin{equation}
    (X,Y) = \frac{1}{2}\left(\rho(x)+\lambda(y)\right).
\end{equation}
Moreover, there is a natural action on an element $X=\xi+\lambda\in \Gamma(TM\oplus T^*\mathcal{M})$ by a two-form $B\in \Gamma(\Lambda^2T^*\mathcal{M})$ defined by 
\begin{equation}
    \ee^B: X\mapsto X+\ii_xB,
\end{equation}
where $\ii_x$ denote the interior product. Notably the inner product is invariant under such $B$-shifts, i.e.\ $(\ee^BX,\ee^BY)=(X,Y)$.

Moreover, an analogue of a Lie bracket on $TM$ is the Courant bracket defined as 
\begin{equation}
    \left[X,Y\right]_C = [x,y]_{\text{Lie}}+\mathcal{L}_x\rho-\mathcal{L}_y\lambda -\frac{1}{2}\dd\left(\rho(x)-\lambda(y)\right),
\end{equation}
where $\mathcal{L}$ is the ordinary Lie derivative. Note the Courant bracket is anti-symmetric and does not satistfy the Jacobi-identity. \todo{Show?} Just as the ordinary Lie bracket is preserved under diffemorphisms also the Courant bracket is. Importantly the Courant bracket possess another symmetry under closed $B$-shifts 
\begin{equation}
    [\ee^BX,\ee^BY]_C = \ee^B[X,Y]_C,
\end{equation}
to see this just insert the $B$-shift into the definition of the Courant bracket and use Cartan's identity $\mathcal{L}_x\lambda = \ii_x\dd\lambda+\dd\ii_x\lambda$ repeatedly. This tells us that both the inner product and the Courant bracket is preserved not only by diffeomorphism $\text{Diff}(\mathcal{M})$, but also under closed two-form transformations $\Lambda^2_{\text{cl}}(\mathcal{M})$. These combine into the semi-direct product of transformations 
\begin{equation}
    \text{Diff}(\mathcal{M})\ltimes\Lambda^2_{\text{cl}}(\mathcal{M}).
\end{equation}
The appearance of a two-form potential $B$ comes from the NS-NS field in ten dimensional supergravity. Note that the $B$-field need only be locally defined such that on overlapping patches $U_\alpha,U_\beta$ it can transform as 
\begin{equation}
   B_{(\alpha)}-B_{(\beta)} = \dd \Lambda_{(\alpha\beta)},
\end{equation}
where $\Lambda_{(\alpha\beta)}$ is a one-form defind on $U_\alpha\cap U_\beta$. For consistency such one-forms needs to satisfy the following on a triple overlap $U_\alpha\cap U_\beta\cap Y_\gamma\neq\emptyset$
\begin{equation}
    \Lambda_{(\alpha\beta)}+\Lambda_{(\beta\gamma)}+\Lambda_{(\gamma\alpha)} =\dd \Lambda_{\alpha\beta\gamma},
\end{equation}
with $\Lambda_{\alpha\beta\gamma}$ a function, or a $0$-form. This consistency condition follows from the reducibility of a closed $B$-shift transformation. 

In the same spirit we discuss exceptional generalized geometry ($d<7$) with structure group $E_{d(d)}\times \mathbb{R}^+$ and a generalised tangent bundle 
\begin{equation}
    E\cong T\mathcal{M}\oplus T^*\mathcal{M}\oplus\ldots.
\end{equation}
Here $\ldots$ denotes extra terms coming from the decomposition of the cordinate module of $E_{d(d)}$ under the $GL(d,\mathbb{R})$ subgroup. As an example consider $\text{dim}\,\mathcal{M}=6$, then the coordinate module is the fundamental $\mathbf{27}$ module of $E_{6(6)}$ which decompose under $GL(6,\mathbb{R})$ as 
\begin{equation}
    \mathbf{27}\to \mathbf{6}+\mathbf{15}+\mathbf{6},
\end{equation}
corresponding to a vector, a two-form and a five-form. The generalized tangent bundle is then given by 
\begin{equation}
    E \cong T\mathcal{M}\oplus\Lambda^2T^*\mathcal{M}\oplus\Lambda^5T^*\mathcal{M}.
\end{equation}

Generally for $d\leq 7$ the generalised tangent bundle is given by 
\begin{equation}
    E \cong T\mathcal{M}\oplus\Lambda^2T^*\mathcal{M}\oplus\Lambda^5T^*\mathcal{M}\oplus\left(T^\mathcal{M}\otimes \Lambda^7T^*\mathcal{M}\right),
\end{equation}
where of course forms of degree higher than the dimension of the manifold vanish. A section $V\in\Gamma(E)$ is given by 
\begin{equation}
    V = v+\omega+\sigma+\tau.
\end{equation}
Likewise the adjoint bundle is given by 
\begin{equation}
    \text{ad}\, F \cong \mathbb{R}\oplus \left(T^*\mathcal{M})\otimes T\mathcal{M}\right)\oplus \Lambda^3 T^*\mathcal{M}\oplus\Lambda^6 T^*\mathcal{M}\oplus\Lambda^3 T\mathcal{M}\oplus\Lambda^3 T^6\mathcal{M},
\end{equation}
and with a section $W\in\Gamma(\text{ad}\,F)$
\begin{equation}
    W = c+r+a+\tilde{a}+\alpha+\tilde{\alpha}.
\end{equation}


\section{Generalized structure}
The bosonic sector of M-theory is given by the metric $g_{MN}$ and the three-form form field $A$ and its corresponding four-form field strength $F=\dd A$. We will then consider a warped product ansatz for the metric 
\begin{equation}
    \dd s^2 = \ee^{2A(y)}\dd s^2(\mathbb{R}^{1,D-1})+\dd s^2(\mathcal{M}),
\end{equation}
where $x,y$ are coordinates on $\mathbb{R}^{1,D-1}$ and $\mathcal{M}$ respectively and $A$ depends only on the internal manifold to preserve Poincaré invariance. The non-trivial fluxes will be chosen to lie entirely in $\mathcal{M}$, again to preserve the symmetries in the external spacetime. For simplicity we will consider $D=4$ such that the internal manifold is seven-dimensional. The generalised tangent space with a structure group $E_{7(7)}\times\mathbb{R}^+$ is then given by 
\begin{equation}
    E \cong T\mathcal{M}\oplus\Lambda^2T^*\mathcal{M}\oplus\Lambda^5T^*\mathcal{M}\oplus\left(T^*\mathcal{M}\otimes\Lambda^7 T^*\mathcal{M}\right).
\end{equation}
Moreover, given $E$ we can construct the generalized frame bundle $F$ given by frames $\{E_A\}$, where $A=1,2,\ldots 56$. Explicitly, in a patch $U$ there exists a canonical basis given by 
\begin{equation}
    \{E_A\} = \{\pd/\pd y^m\}\cup \{\dd y^{m_1}\wedge\dd y^{m_2}\}\cup\{\dd y^{m_1}\wedge\ldots\wedge\dd y^{m_5}\}\cup \{\dd y^m\otimes\dd y^{m_1}\wedge\ldots\wedge\dd y^{m_7}\},
\end{equation}
and any other frame is related to such a frame by an $E_{7(7)}\times \mathbb{R}^+$ transformation, this then defines a principle bundle $F$ with structure group $E_{7(7)}\times \mathbb{R}^+$. Any other tensor field can then be constructed as sections of an associated vector bundle, thus making the exceptional group manifest in the formulation. An especially important such associated vector bundle is the adjoint bundle which decompose under $\mathfrak{gl}(7,\mathbb{R})$ to\todo{Correct this, not a bundle}
\begin{equation}
    \mathfrak{e}_{7(7)}\to \left(T\mathcal{M}\oplus T^*\mathcal{M}\right)\oplus\Lambda^6T\mathcal{M}\oplus\Lambda^6T^*\mathcal{M}\oplus\Lambda^3T\mathcal{M}\oplus\Lambda^3T^*\mathcal{M}
\end{equation}
which gives 
\begin{equation}
    \text{ad}\,F \cong  \mathbb{R}\oplus\left(T\mathcal{M}\oplus T^*\mathcal{M}\right)\oplus\Lambda^6T\mathcal{M}\oplus\Lambda^6T^*\mathcal{M}\oplus\Lambda^3T\mathcal{M}\oplus\Lambda^3T^*\mathcal{M}.
\end{equation}
The reduction of the structure group to some $G$-structure, where $G$ is a subgroup of the structure group, proved fruitful to characterise geometries preserving some amount of supersymmetry. The goal is therefore to again find reduced structures, this time for some $G\subset E_{7(7)}$ subgroup. Moreover, this was equivalent to finding a set of invariant tensors under $G$, or in other words, finding singlets when decomposing the fields under $E_{7(7)}\to G$. 

As such we decompose the adjoint representation $\mathbf{133}$ of $\mathfrak{e}_{7(7)}$ under $\mathfrak{su}(2)\times\mathfrak{so}^*(12)\subset$\todo{Spin12, why this real form?} to find 
\begin{equation}
    \mathbf{133}\to (\mathbf{1},\mathbf{66})+(\mathbf{2},\mathbf{32})+(\mathbf{3},\mathbf{1}).
\end{equation}
We thus find an $E_{7(7)}$ tensor which is invariant under $\mathfrak{so}^*(12)$ which, moreover, transforms as a triplet of $\mathfrak{su}(2)$. Such a triplet $J_\alpha$ of invariant tensors define a structure, called the hypermultiplet structure, and are sections of the adjoint bundle 
\begin{equation}
    J_\alpha \in \Gamma(\text{ad}\, F\otimes \left(T^*\mathcal{M}\right)^{1/2}).
\end{equation}
As in the case of vanishing flux the hypermultiplet structure $J_\alpha$ can be written in terms of the Killing spinor. Before moving on to the integrability condition that the hypermultiplet structure should satisfy we look for another structure. Decompose the fundamental $\mathbf{56}$ of $\mathfrak{e}_{7(7)}$ under $\mathfrak{e}_{6(2)}$
\begin{equation}
    \mathbf{56}\to\mathbf{27}+\overbar{\mathbf{27}}+2\cdot \mathbf{1}.
\end{equation}
The two singlets in this decomposition $V$ and $\tilde{V}$ defines the vector multiplet structure, and hence also an $E_{6(2)}$ sub-bundle. There is one further requirement that the singlet $V$ is chosen such that 
\begin{equation}
    q(V)>0,
\end{equation}
where $q$ is the quartic invariant of $E_7$. This ensures that the stabiliser group is precisely the real form $E_{6(2)}$ \cite{Ferrara:1997uz}. As we will argue below the coset space with these particular real forms $E_{7(7)}/E_{6(2)}$, have a natural special kähler structure that is demanded by supersymmetry. 

To determine that $E_{6(2)}\subset E_{7(7)}$ look at the following cosets 
\begin{equation}\label{eq:e7toe6}
    \underbrace{\frac{E_{7(7)}}{SU(8)}}_{70}\to\underbrace{\frac{E_{6(2)}}{SU(6)\times SU(2)}}_{40}\times \underbrace{\frac{SL(2)}{SO(2)}}_{2}\times \underbrace{\mathbb{R}}_{1},
\end{equation}
where we denoted the number of non-compact generators in each term. The decomposition of the adjoint of $E_{7(7)}$ under $E_{6(2)}\times SL(2)\times \mathbb{R}$ is given by 
\begin{equation}\label{eq:e7toe6adj}
    (\overbar{\mathbf{27}},\mathbf{1})_{-1}\oplus\left[(\mathbf{78},\mathbf{1})\oplus(\mathbf{1},\mathbf{3})\oplus(\mathbf{1},\mathbf{1})\right]_0\oplus(\mathbf{27},\mathbf{1})_{1}.
\end{equation}
The missing $27$ non-compact generators in $\eqref{eq:e7toe6}$ thus corresponds to $(\mathbf{27},\mathbf{1})_{1}$ in \eqref{eq:e7toe6adj} which indeed are non-compact since acting in any representation they are triangular matrices and hence non-compact. 


\subsection{Integrability}

Integrability of the hypermultiplet structure and of the vector multiplet structure are in analogy to the fluxless case given by differential conditons the corresponding invariant tensors. In other words, these differential conditions are necessary for the Killing spinor satifsying the Killing spinor equation, which in turn preserve some amount of supersymmetry. In the case of vanishing flux this was related to vanishing intrinsic torsion, i.e.\ given a $G$-structure the existence of an $G$-compatible torsion-free connection implied that the structure was integrable. Motivated by this analogy we will look at the conditions of vanishing intrinsic torsion, and later on see that this is enough to satisfy the Killing equations. 
\subsubsection{Generalised intrinsic torsion}
To define the generalised intrinsic torsion we define a covariant derivative $D$ which is defined in the following way 
\begin{equation}
    D_M\cdot V_N = \pd_MV_N+\tensor{\Gamma}{_M^P_N}V_N,
\end{equation}
where $V\in\Gamma(E)$, $\Gamma_M\in\Gamma(E^*\otimes \text{ad}\,F)$ and 
\begin{equation}
    \pd_M = \pd_m \qquad \text{for} \quad M=m\qquad 0\qquad \text{otherwise}.
\end{equation}
Imposing Leibneiz rule 
\begin{equation}
    D_M(fV) = (D_Mf)\otimes V+f(D_MV),
\end{equation}
the action of a covariant derivative is easily extended to sections of any associated vector bundle, i.e.\ to any other generalised tensor field. The torsion of such a connection is then defined by 
\begin{equation}
    \mathscr{T}(V)\cdot X = L_V^D X-L_V X,
\end{equation}
where $X$ is an arbitrary generalised tensor field, $L_V^D$ is the generalised Lie derivative with the replacement $\pd\to D$ and $\cdot$ denotes the adjoint action. Note that this definition agrees with the one presented in extended geometries, i.e.\ torsion is the part of the connection that transforms homogeneously under generalised diffeomorphisms, at least when ancillary transformations are absent. Moreover, $\mathscr{T}$ can be seen as a map
\begin{equation}
    \mathscr{T}: E\to \text{ad}\, F. 
\end{equation}
In the $E_{7(7)}$ case we already know that 
\begin{equation}
    \mathscr{T}\in \mathbf{56}\oplus\mathbf{912}
\end{equation}
under $E_{7(7)}$. Assuming a connection $D$ is compatible with the $G$-structure any other compatible connection is given by $D\to D+\Sigma$ with 
\begin{equation}
    \Sigma \in \Gamma(E^*\otimes \text{ad}\,P_G),
\end{equation}
where $\text{ad}\,P_G$ has a typical fibre $\text{Lie}\,G=\mathfrak{g}$ and is an associated vector bundle to the principal sub-bundle $P_G$. Given a compatible connection with the torsion we define the intrinsic torsion as the part of torsion that can not be removed by adding a term $\Sigma$. With this definition it is clear that the presence of intrinsic torsion is an obstruction to finding a $G$-compatible torsion-free connection. Another we to put it is to define a map $\phi$ that projects $\Sigma$ onto its torsion part. The intrinsic torsion is then given by 
\begin{equation}
    W_{\text{int}} = W/\text{im}\,\phi,
\end{equation}
with $W_{\text{int}}$ and $W$ being the space of intrinsic torsion and the space of torsion respectively. 

Take our main example with $E_{7(7)}\times R^+$ and $G=Spin^*(12)\times SU(2)$. To see what parts are intrinsic torsion we decompose $\mathbf{56}\oplus\mathbf{912}$ under $G$ to find 
\begin{equation}
    \mathbf{56}\oplus\mathbf{912} \to 2\cdot (\mathbf{12},\mathbf{2})+(\mathbf{32},\mathbf{1})+(\mathbf{32},\mathbf{3})+(\mathbf{220},\mathbf{2})+(\mathbf{352},\mathbf{1}).
\end{equation}
Likewise we decompose $\Sigma$ under $G$ using $\mathbf{56}\to (\mathbf{12},\mathbf{2})+(\mathbf{32},\mathbf{1})$ 
\begin{equation}
    \left((\mathbf{12},\mathbf{2})+(\mathbf{32},\mathbf{1})\right)\times (\mathbf{66},\mathbf{1})=(\mathbf{220},\mathbf{2})+(\mathbf{12},\mathbf{2})+(\mathbf{560},\mathbf{2})+(\mathbf{22},\mathbf{1})+(\mathbf{1728},\mathbf{1})+(\mathbf{352},\mathbf{1}).
\end{equation}
The two big modules containing $\mathbf{1728}$ and $\mathbf{560}$ are not torsion and hence we find the intrinsic torsion 
\begin{equation}
    W_{\text{int}} = (\mathbf{12},\mathbf{2})+(\mathbf{32},\mathbf{3}).
\end{equation}

Performing the same analysis for $G=E_{6(2)}$ is straightforward and the torsion decompose as
\begin{equation}
    \mathbf{56}+\mathbf{912}\to \mathbf{1}+2\cdot\mathbf{27}+\mathbf{78}+\mathbf{351}+\text{c.c.}
\end{equation}
and we also have 
\begin{equation}
    \left(\mathbf{1}+\mathbf{27}+\text{c.c.})\right)\times \mathbf{78} = \mathbf{78}+\mathbf{27}+\mathbf{351}+\mathbf{1728}+\text{c.c.}
\end{equation}
We thus find the intrinsic torsion in the $E_{6(2)}$ case to be 
\begin{equation}
    W_{\text{int}}^{E_{6(2)}} = \textbf{1}+\mathbf{27}+\text{c.c.}
\end{equation}

Given the representations of intrinsic torsion we want to show that vanishing intrinsic torsion is equivalent to
\begin{itemize}
    \item vanishing of so-called momentum maps which are differential conditions on the underlying structures,
    \item Killing spinor equations being satisfied. 
\end{itemize}
It then follows that a reduced structure on which the momentum maps corresponds a supersymmetric flux compactification. 


\subsubsection{Integrability in terms of momentum maps}
As claimed above the finding a $G$-compatible connection with vanishing torsion could be written in terms of momentum maps. The reason for such a rewriting is that it will allows to examine the moduli space of structure in a clever way. 

We introduce momentum maps in terms of an example of Hamiltonian physics. Given a phase space $X$ of dimension $2n$ there exists a symplectic two-form $\omega$ given by
\begin{equation}
    \omega = \dd p_i\wedge\dd q^i,
\end{equation}
where $q^i$, and $p_i$ are position and momenta respectively and are coordinates on $X$. Given a Hamiltonian function $H: X\to\mathbb{R}$ we can construct a corresponding vector field $X_H$ given by 
\begin{equation}
    \ii _{X_H}\omega = -\dd H.
\end{equation}
The flow of $X$ along $X_H$ should be $0$ in order to preserve the energy which is immediate by 
\begin{equation}
    \dd H (X_H) = —\omega(X_H,X_H) = 0. 
\end{equation}
This in turn implies that the symplectic two-form is preserved by action of the hamiltonian vector field $X_H$ 
\begin{equation}
    \mathcal{L}_{X_H} \omega = \ii_{X_H}\dd \omega + \dd\ii_{X_H} \omega = 0,
\end{equation}
where we used Cartan's identity, $\dd\omega=0$ and that $\ii_{X_H}$ is by construction exact. 

Now consider an arbitrary manifold $\mathcal{M}$ with a closed non-degenerate two-form, i.e.\ a symplectic form $\Omega$. We also define an action of a Lie group $G$ on the phase space $\mathcal{M}$
\begin{equation}
    \mathcal{M}\times G\to \mathcal{M} \qquad g \times m\mapsto g\cdot m.
\end{equation}
This induces an action of the algebra $\mathfrak{g}$ (assume $G$ is connected or restrict to its connected component)
\begin{equation}
    \mathcal{M}\times \mathfrak{g} \to T_m \mathcal{M}\qquad \frac{\dd }{\dd t}\exp (tg)\cdot m|_{t=0} =X_g,
\end{equation}
and we thus see that any element in $g\in\mathfrak{g}$ induces a vector field $X_g$. The symplectic two-form $\Omega$ transforms under the action of $g$ by the Lie derivative with respect to this vector field $\mathcal{L}_{X_g}$ as $\delta_g \Omega = \mathcal{L}_{X_g}$ and we find
\begin{equation}
    \delta_g \Omega = \dd \ii_{X_g} \Omega + \ii_{X_g}\dd\Omega.
\end{equation}
Since $\Omega$ is closed by assumption we find that $\Omega$ is preserved by the action of $G$ if 
\begin{equation}
    \dd \ii_{X_g}\Omega = 0,
\end{equation}
the group $G$ is then said to act as a symplectomorphism on $\mathcal{M}$. A momentum map $\mu: \mathcal{M}\to\mathfrak{g}^*$ is defined by
\begin{equation}
    \dd \mu (g) = \ii_{X_g}\Omega,
\end{equation}
such that the following diagram commutes 
\begin{center}
\begin{tikzpicture}
  \matrix (m) [matrix of math nodes,row sep=3em,column sep=4em,minimum width=2em]
  {
     \mathcal{M} & \mathfrak{g}^* \\
     \mathcal{M} & \mathfrak{g}^* \\};
  \path[-stealth]
    (m-1-1) edge node [left] {$G$} (m-2-1)
            edge [] node [below] {$\mu$} (m-1-2)
    (m-2-1.east|-m-2-2) edge node [below] {$\mu$}
             (m-2-2)
    (m-1-2) edge node [right] {$G$} (m-2-2)
            (m-2-1);
\end{tikzpicture}
\end{center}
Note that $G$ acts on $\mathfrak{g}^*$ in the coadjoint representation such that $X\in G$ acts as $\text{Ad}_X^*\tilde{g}(g)=\tilde{g}(\text{Ad}_{X^{-1}})g$, with $g\in \mathfrak{g}$ and $\tilde{g}\in \mathfrak{g}^*$. That the diagram above commutes is the statement that $\mu$ is $G$-equivariant and is explicitly the condition 
\begin{equation}
    \mu(X\cdot m)\cdot g = \text{Ad}_{X}^*(\mu(m))\cdot g = \mu(m)\cdot(\text{Ad}_{X^{-1}}g). 
\end{equation}
The main reason for introducing momentum maps in our case is that we can define the symplectic quotient as 
\begin{equation}
    \mathcal{M}\sslash G := \mu^{-1}(0)/G = \{[m]\in \mathcal{M} : m\sim g\cdot n, m,n\in \mathcal{M}, g\in G\},
\end{equation}
which is always well-defined as a topological quotient but need not be a manifold in general. The choice of $0\in\mathfrak{g}^*$ ensures that the action of $G$ is free and if we moreover assume that it the action is proper and $0$ is regular value then there is a theorem by Marsden-Weinstein-Meyer stating that 
\begin{itemize}
    \item $\mathcal{M}\sslash G$ is a manifold,
    \item it also inherits a unique symplectic form from $\Omega$. 
\end{itemize}
Note that this could be generalised slightly by replacing $0\in\mathfrak{g}^*\to \tilde{g}\in\mathfrak{g}^*$ and replacing $G$ with the stabiliser of $\tilde{g}$.  
%\todo{https://userpage.fu-berlin.de/hoskins/Sympl_reduction.pdf }

\subsubsection{Momentum map hypermultiplet structure}
Consider the hypermultiplet structure $J_\alpha\in\Gamma(\text{ad}\, F\otimes (\text{det}\, T^*\mathcal{M})^{1/2}))$ which defines the $Spin^*(12)\times SU(2)$ structure in terms of a triplet of symplectic forms. As such at each point $m\in\mathcal{M}$ the structure is given by a point in the coset space
\begin{equation}
    J_\alpha|_x \in F_H = E_{7(7)}\times\mathbb{R}^+/Spin^*(12).
\end{equation}

As such we can consider the following bundle
\begin{center}
\begin{tikzpicture}
  \matrix (m) [matrix of math nodes,row sep=3em,column sep=4em,minimum width=2em]
  {
     E_{7(7)}\times\mathbb{R}^+/Spin^*(12) & P_H \\
                 & \mathcal{M} \\};
  \path[-stealth]
    (m-1-1) edge node [below] {} (m-1-2);
    \path[-stealth]
    (m-1-2) edge node [below] {} (m-2-2);
\end{tikzpicture}
\end{center}
and find that the space of all hypermultiplet structure $S_{H}$ is given by the space of sections of $P_H$, $S_H = \Gamma(P)$. A structure is then given by a point $s\in S_H$ and small deformations of the structures can be considered as vectors in the tangent space $T_sS_H$. The structure can similarly be thought of as functions on $S_H$ valued in $\text{ad}\, F)\otimes (\text{det}\, T^*\mathcal{M})^{1/2})$
\begin{equation}
    J_\alpha: S_H \to \text{ad}\, F)\otimes (\text{det}\, T^*\mathcal{M})^{1/2}),
\end{equation}
on which the tangent vector $v_\alpha\in\Gamma(T_{J_\alpha}S_H)$ acts as 
\begin{equation}
    v_\alpha = i_v \delta J_\alpha,
\end{equation}
where $\delta J_\alpha$ denotes the exterior derivative of on the space of sections. With this we can define a triplet of symplectic forms that define the geometry 
\begin{equation}
    \Omega_\alpha(v,w) = \epsilon_{\alpha\beta}\int_\mathcal{M} \kappa(v,w),
\end{equation}
where $\kappa$ is the killing form on $\mathfrak{e}_{8(8)}$ and $v,w\in T_s S_H$ at some point $s\in S_H$. 

In order to put some differential condition on the structures in a covariant manner we look at the deformation due to generalised diffeomorphisms. These are parameterised by an element $V\in \Gamma(E)$ and the action is given by the generalised Lie derivative 
\begin{equation}
    \delta_V J_\alpha = L_V J_\alpha. 
\end{equation}
Since $\delta J_\alpha$ is an element in the tangent space of $S_H$ we can define a mapping from the algebra of generalised diffeomorphisms $\mathfrak{gdiff}$ to a vector field on $S_H$ defined through its action on a structure $J_\alpha$ as 
\begin{align}
    X: \mathfrak{gdiff}&\to \Gamma(TS_H)\\
        V&\mapsto X_V \qquad\qquad  \text{s.t.}\qquad X_V(J_\alpha)=L_VJ_\alpha.
\end{align}
Consider now the triplet of symplectic form evaluated on $X_V$
\begin{align}\label{eq:sympev}
    \Omega_\alpha (X_V,v) &= \epsilon_{\alpha\beta\gamma}\int_\mathcal{M}\kappa(v_\beta,L_VJ_\gamma).
\end{align}
Since $\kappa(v_\beta,L_VJ_\gamma) \in \text{det}\, T^{*}M$ it follows that 
\begin{equation}
    \int_\mathcal{M} L_V \left(\kappa(v_\beta,L_VJ_\gamma)\right) = \int_\mathcal{M}v^i\pd_i \left(\kappa(v_\beta,L_VJ_\gamma)\right) = 0,
\end{equation}
where $v\in\Gamma(TM)$. By the Leibniz property of the generalised derivative we can thus rewrite \eqref{eq:sympev} as 
\begin{equation}
    \Omega_\alpha(X_V,v) = \frac{1}{2}\epsilon\int_\mathcal{M}\left[\kappa(v_\beta,L_VJ_\gamma)+\kappa(L_Vv_\gamma,J_\beta)\right],
\end{equation}
where we split the original integrand into two and relabeled some dummy indices. This more symmetrical form allow us to rewrite this expression as the variation of a triplet of momentum maps $\mu_\alpha$ defined as 
\begin{equation}
    \mu_\alpha = \frac{1}{2}\epsilon_{\alpha_\beta\gamma}\int_\mathcal{M}\kappa(J_\beta,L_VJ_\gamma).
\end{equation}
Using the exterior derivative on $S_H$ which is defined as $i_v\delta J_\alpha=v_\alpha$, the variation of the momentum maps is given by 
\begin{equation}
    i_v\delta\mu_\alpha = \frac{1}{2}\epsilon_{\alpha\beta\gamma}\int_{\mathcal{M}}\left[\kappa(v_\beta,L_VJ_\gamma)+\kappa(J_\beta,L_Vv_\gamma)\right].
\end{equation}

In order to find a torsion-free $G$-compatible connection the corresponding intrinsic torsions have to vanish. We will now show that the momentum maps contain precisely the intrinsic torsion and that vanishing of the momentum maps thus imply vanishing intrinsic torsion. By definition of torsion the generalised Lie derivative can be rewritten as $L_V J_\alpha = L_V^DJ_\alpha-[T(V),J_\alpha]$ and hence the momentum maps are given by 
\begin{equation}
    \mu_\alpha = \frac{1}{2}\epsilon_{\alpha\beta\gamma}\int_\mathcal{M}\left(\kappa(J_\beta,V\cdot DJ_\gamma)+\kappa(J_\beta,(D\times_{\text{ad}} V)J_\gamma)-\kappa(J_\beta,[T(V),J_\gamma])\right).
\end{equation}
The first term in this expression vanish by definition of a $G$-compatible connection.  

\subsubsection{Momentum map vector multiplet structure}
Consider the vectormultiplet structures $V$ and $\tilde{V}$ which define the $E_{6(2)}$ sub bundle. Then at each point in $m\in\mathcal{M}$ the choice of either $V$ or $\tilde{V}$ is equivalent to a point in the coset space 
\begin{equation}
    V|_m \in F_V = E_{7(7)}\times\mathbb{R}^+/E_{6(2)},
\end{equation}
and similarly for $\tilde{V}$. As in the hypermultiplet case we thus consider the bundle 
\begin{center}
\begin{tikzpicture}
  \matrix (m) [matrix of math nodes,row sep=3em,column sep=4em,minimum width=2em]
  {
     E_{7(7)}\times\mathbb{R}^+/E_{6(2)} & P_V \\
                 & \mathcal{M} \\};
  \path[-stealth]
    (m-1-1) edge node [below] {} (m-1-2);
    \path[-stealth]
    (m-1-2) edge node [below] {} (m-2-2);
\end{tikzpicture}
\end{center}
such that the space of vector multiplet structures $S_V$ is given by the space of sections of $S_V=\Gamma(P_V)$. Deformations of the structure are again generated by the generalised Lie derivative 
\begin{equation}
    \delta_W V = L_WV,
\end{equation}
where $W\in\Gamma(E)$. Likewise this defines a vector field on $S_V$, i.e. a section of $TS_V$ as for the hypermultiplet case 
\begin{align}
    X_W: \mathfrak{gdiff}&\to \Gamma(TS_V)\\
        W&\mapsto X_W \qquad\qquad  \text{s.t.}\qquad X_W(V)=L_WV.
\end{align}
With this construction we define a symplectic form also for the vector multiplet structure, if $Z\Gamma(TS_V)$ and $X_W\in\Gamma{TS_V}$ as defined above then the symplectic form is defined as  
\begin{equation}
    \Omega(X_W,Z) =\int_\mathcal{M}s(X_W,Z),
\end{equation}
where $s$ is the symplectic invariant of $E_{7(7)}$. We then define the momentum map for the vector multiplet structure as 
\begin{equation}
    \mu(W) = \frac{1}{2}\int_\mathcal{M}s(L_WV,V). 
\end{equation}
Using the exterior derivative on $S_V$ we have 
\begin{equation}
    i_Z\delta\mu(W) = \frac{1}{2}\int_\mathcal{M}\left(s(L_WZ,V)+s(L_WV,Z)\right),
\end{equation}
which by the anti-symmetry of the argument of $s$ and using the Leibniz property of $L$ reduce to $\Omega(X_W,Z)$. 


As mentioned several times, preserving supersymmetry of the vacuum corresponds to two conditions, one topological and one differential. To start with we consider the constraints that follows from the existence of a globally defined nowhere vanishing spinor. First we note the existence of a generalised metric on $E_{7(7)}\times\mathbb{R}^+$ allows us to introduce an $\tilde{H}$- structure where $H=SU(8)$ is the double cover of the maximal compact subalgebra of $E_{7(7)}$. This is important since spinors then transforms under  
\begin{equation}
    Spin(1,3)\times SU(8).
\end{equation}
In terms of bundles, the gravitino is a section of the spin-bundle $\Gamma(J)$ and the dilatino and the supersymmetry parameter $\epsilon$ are sections of $\Gamma(S)$ with 
$J=\mathbf{56}$ and $S=\mathbf{8}$ of $SU(8)$. Under $Spin(7)\subset Spin(8)\subset SU(8)$ these decompose as 
\begin{equation}
    \mathbf{8}\to \mathbf{8}, \qquad  \mathbf{56}\to \mathbf{8}+\mathbf{48},
\end{equation}
of $Spin(7)$. 




\chapter{Extended Geometries}
In this chapter we will describe the general construction of extended geometries recently constructed in \cite{CederwallPalmkvist2017}. These are generalisations of double geometry and exceptional geometry to a geometry based on a Kac-Moody algebra $\mathfrak{g}$ and an irreducible highest weight module $R(\lambda)$. The construction is built around the generalised diffemorphisms and closure of their algebra. Moreover, a section condition is again crucial in order for the consistency of the algebra. 

\todo[inline]{physics}
\section{Extended spacetime and setup}
An extended geometry is based on the split real-form of a Kac-Moody algebra $\mathfrak{g}$ exponentiated to a group $G$ and an irreducible highest weight coordinate module $R(\lambda)$. As introduced in section \ref{chap:symmetries} a Kac-Moody algebra is described in terms of its Cartan matrix $a_{ij}$ which we assume to be symmetrisable, i.e.\ that there exists a diagonal matrix $d$ with nonzero entries $d_j$ such that $da$ is a symmetric matrix. 
\todo[inline]{Setup the Kac-Moody algebra etc depending on how much is introduced in the Symmetries section}

\section{Generalised diffemorphisms}
Generalised diffemorphisms play a central rôle in the construction of extended geometries, the general form of which is
\begin{equation}
    \delta_\xi V^M = \mathscr{L}_\xi V^M = \xi^N\pd_N V^M+\tensor{Z}{^M^N}_{PQ}\pd_N\xi^PV^Q,
\end{equation}
where $Z$ is a invariant tensor of $\mathfrak{g}$. The first term can be viewed as a transport term coming from a Taylor expansion of the argument and the second term is a projector of the $^N_P$ indices on the adjoint of $\mathfrak{g}\oplus \mathbb{R}$. The invariant tensor $Z$ is hence given by 
\begin{equation}
    \tensor{Z}{^M^N_P_Q} = -k\eta_{\alpha\beta}\tensor{T}{^\alpha^M_Q}\tensor{T}{^\beta^N_P}+\beta\delta^M_Q\delta^N_P,
\end{equation}
where $\eta_{\alpha\beta}$ is the inverse of the invariant bilinear form on $\mathfrak{g}$ and $T^{\alpha}$ are the generators in the represenation $R(\lambda)$ as well was $k$ and $\beta$ constant. This can be written in an equivalent way using the tensor
\begin{equation}\label{eq:Y}
    \tensor{Y}{^M^P_N_Q} = \tensor{Z}{^M^P_N_Q}+\delta^M_P\delta^N_Q = -k\eta_{\alpha\beta}\tensor{T}{^\alpha^M_Q}\tensor{T}{^\beta^N_P}+\beta\delta^M_Q\delta^N_P+\delta^M_P\delta^N_Q
\end{equation}
as 
\begin{equation}
    \mathscr{L}_\xi V^M = \xi^N\pd_N V^M-\pd_N\xi^MV^N+\tensor{Y}{^M^N}_{PQ}\pd_N\xi^PV^Q
\end{equation}
such that the contribution from the $Y$ tensor can be interpreted as the departure of the generalised Lie derivative in the structure algebra $\mathfrak{g}\oplus\mathbb{R}$ compared to that of ordinary geometry $\mathfrak{gl}$. 

A crucial consistency condition is now whether the algebra closes or not, to that and we examine 
\begin{equation}\label{eq:closure}
    \left(\left[\mathscr{L}_\xi,\mathscr{L}_\eta\right]-\mathscr{L}_{\frac{1}{2}\left(\mathscr{L}_\xi\eta-\mathscr{L}_\eta\xi\right)}\right)V^M\overset{?}{=}0.
\end{equation}
After a good deal of algebra one finds that \todo{Calculation in appendix?}
\begin{equation}
    \left(\left[\mathscr{L}_\xi,\mathscr{L}_\eta\right]-\mathscr{L}_{\frac{1}{2}\left(\mathscr{L}_\xi\eta-\mathscr{L}_\eta\xi\right)}\right)V^M = \frac{1}{2}\tensor{Z}{^{MP}_{QN}}\tensor{Y}{^{QR}_{ST}}\xi^T\pd_P\pd_R\eta^SV^N-\left(\xi\leftrightarrow\eta\right)
\end{equation}
if the $Y$-tensor satisfies certain constraints. The first is the (strong) section constraint 
\begin{equation}
    \tensor{Y}{^{MN}_{PQ}}\pd_M\otimes\pd_N = 0,
\end{equation}
where $\pd_M\otimes\pd_N$ indicates that each derivative act on an arbitrary field. To gets some intuition we look at the case for DFT with
\begin{equation}
    \tensor{Y}{^{MN}_{PQ}}=-2\tensor{P}{^{MN}_{PQ}}+\delta^M_P\delta^N_Q,
\end{equation}
where $\tensor{P}{^{MN}_{PQ}}=-\frac{1}{4}\eta^{IK}\eta^{JL}(T_{IJ})^M_Q(T_{KL})^N_P$ is the projection on the adjoint. The generators in the fundamental of $O(D,D)$ \cite{Berman2014} are $(T_{IJ})^M_P=\delta^M_I\eta_{JP}-\delta_J^M\eta_{IP}$ and a straightforward calculation shows that 
\begin{equation}\label{eq:section}
    \tensor{Y}{^{MN}_{PQ}}\pd_M\otimes\pd_N = \eta^{MN}\pd_M\otimes\pd_N,
\end{equation}
which indeed is the section condition for DFT introduced in section \ref{sec:DFT}. Moreover, the $Y$-tensor also needs to satisfy the following constraints for the algebra to close
\begin{subequations}
    \begin{align}
        \begin{split}
        \Big(\tensor{Y}{^{MN}_{TQ}}\tensor{Y}{^{TP}_{[SR]}}&+2\tensor{Y}{^{MN}_{[R|T|}}\tensor{Y}{^{TP}_{S]Q}}\\
        -\tensor{Y}{^{MN}_{[RS]}}\delta^P_Q&-2\tensor{Y}{^{MN}_{[S|Q|}}\delta_{R]}^P\Big)\pd_{(N}\otimes\pd_{P)}=0,
        \end{split}\label{eq:closure1}
        \\
        \begin{split}
        \Big(\tensor{Y}{^{MN}_{TQ}}\tensor{Y}{^{TP}_{(SR)}}&+2\tensor{Y}{^{MN}_{(R|T|}}\tensor{Y}{^{TP}_{S)Q}}\\
        -\tensor{Y}{^{MN}_{(RS)}}\delta^P_Q&-2\tensor{Y}{^{MN}_{(S|Q|}}\delta_{R)}^P\Big)\pd_{[N}\otimes\pd_{P]}=0.
        \end{split}\label{eq:closure2}
    \end{align}
    \end{subequations}

The equations \eqref{eq:closure1}-\eqref{eq:closure2} comes from the term involving $(\pd\xi)(\pd\eta)$. Whether the remaining term in \eqref{eq:closure} needed for closure vanish or not depends on the specific algebra together with the choice of coordinate module $R(\lambda)$, a simple criterion for this will be derived below using certain extensions of $\mathfrak{g}$. As of yet the particular form, nor any symmetry properties, of $Y$ has been used, however, inserting the explicit form \eqref{eq:Y} in \eqref{eq:closure1}-\eqref{eq:closure2} one finds that they are satisfied if the section condition is fulfilled. 

Up to the remaining term we have thus seen that the closure of the algebra is entirely based on solving the section condition \eqref{eq:section}. Solving the section condition can be done in two step, first one imposes the weak section stating that any momenta $|p\rangle\in \overbar{R(\lambda)}$ lie in a minimal orbit of $G$. This is equivalent to $|p\rangle\otimes|p\rangle \in \overbar{R(2\lambda)}$ \cite{Berman2013,Bossard2017}. Secondly we demand that the product of two arbitrary momenta $|p\rangle,|q\rangle\in\overbar{R(\lambda)}$ contains only the dual of the highest modules in the symmetric $R^s_h$ and anti-symmetric $R^a_h$ product of $R(\lambda)$ respectively. In other words, the second step means that we set 
\begin{equation}
    \left(\pd\otimes\pd\right)|_{R_2} = 0 \qquad \left(\pd\otimes\pd\right)|_{\tilde{R}_2} =0,
\end{equation}
where $R_2$ and $\tilde{R}_2$ are
\begin{align}
    R_2 &= \vee^2 R(\lambda)\ominus R_h^s,\\
    \tilde{R}_2 &= \wedge^2R(\lambda)\ominus  R_h^a
\end{align}
according to the discussion above. 
% CREATED BY DAVID FRISK, 2016
Extended geometries\\
Extensions of geometry motivated by the symmetries of string theory/M-theory\\
ROBIN KARLSSON\\
Department of Physics\\
Chalmers University of Technology \setlength{\parskip}{0.5cm}

\thispagestyle{plain}			% Supress header 
\setlength{\parskip}{0pt plus 1.0pt}
\section*{Abstract}
The low-energy effective field theories of string theory/M-theory compactified on torii posses ``hidden'' symmetries by the exceptional Lie groups. Moreover a discrete version, U-duality, appears to be unbroken in the full theory. Generically such transformations mix gravitational and non-gravitational degrees of freedom and a covariant formalism calls for a merger of the underlying fields. The goal of extended geometries is therefore to generalise ordinary geometry in order to include non-gravitational degrees of freedom as a part of an extended geometry. This is done by introducing spacetime coordinates in a module of an arbritrary structure group which do effectively merge gravity with other fields. The physically most interesting cases are double field theory and exceptional field theory which are introduced before the general construction of extended geometries. Especially we focus on closure of the algebra of generalised diffeomorphisms which for consistency requires a local embedding of ordinary geometry. Moreover, an invariant action for the generalised metric is derived as well as geometrical objects such as torsion and a generalised Ricci-tensor. Lastly a specific sector of an extended geometry based on the rank seven exceptional group is used to study some properties of flux compactifications from eleven dimensional supergravity to four dimensions. 



%In general such symmetry transformations mix gravitational and non-gravitational degrees of freedom and the goal of exceptional geometry is to unify these fields by generalising the notion of geometry. Extended geometry is a general construction, 
%Extended geometries aims to include such extra degrees of freedom 
%include these symmetries as a part of the underlying geometry by extending spacetime to transform in a module of an extended structure group. The physically most interesting cases are geometries based on structure groups being the exceptional Lie groups which 


%Moreover, we look at 


%A different approach of exceptional generalised geometry making the exceptional groups manifest in supergravity is also discussed. This is also used to study some details of flux compactifications rather naturally. 

% KEYWORDS (MAXIMUM 10 WORDS)
\vfill
Keywords: Extended geometry, U-duality, M-theory, symmetries, compactification.

\newpage				% Create empty back of side
\thispagestyle{empty}
\mbox{}
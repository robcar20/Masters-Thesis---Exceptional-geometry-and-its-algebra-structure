\documentclass{article}
\usepackage[utf8]{inputenc}
\usepackage{standard}

\title{Generalised geometry and string theory dualities\\
\Large{Thesis Proposal}}

\author{Robin Karlsson}
\date{\today}

\begin{document}
\maketitle

\section*{Background}
Physics today are explained in terms of four fundamental forces: the electromagnetic-, the weak-, the strong- and gravitational force. The first three of these are described within in the framework of quantum field theory and describe the microscopic world. Moreover are they (usually) formulated in a flat spacetime, i.e.\ without the presence of gravity. Gravity on the other hand is explained with Einstein's theory of general relativity, where gravity manifests itself as curvature of spacetime. The latter is a classical theory which one would like to quantize in order to merge all forces within the same framework. Naturally one therefore would want to reformulate general relativity as a quantum field theory, which, turns out to be incredibly hard. 

One reason for this is that one of the main difficulties with quantum field theory is that the naive theory one writes down usually ends up producing infinities as answer to physical observables. This is of course unacceptable unless one finds a way to consistently rewrite the theory such that it only predicts finite answers to physical observables; this procedure is called ``renormalization'' and when this is possible we say that the theory is renormalizable. It turns out that quantizing general relativity in form of a quantum field theory is ``non-renormalizable'' and it is thus impossible to construct a consistent theory. The problem of joining these quantum field theories with a quantum theory of gravity is therefore under investigation. 

One of the most prominent example of quantum gravity is String theory, which is a theory where the most fundamental object is a one-dimensional string. Quantization of strings result in oscillations and these oscillations are interpreted as the manifestation of different particles. Notably it turns out that the proposed particle mediating the gravitational force, the ``graviton'', is always present in the spectrum, quantization of strings thus seems the enforce gravity whether one wants to or not. Furthermore one finds that a theory of strings also naturally includes objects of higher dimensions, so called branes. 

In the 90s five different consistent theories of strings had been found and the hope was that one of these would explain our universe. It was then found by Witten \cite{Witten1995} that all these theories seems in fact to be related and be a part of a larger theory. This larger theory was coined ``M-theory'' and all five of the string theories seems be described by different limits within this theory. Moreover are the different theories related in terms of \emph{dualities} mapping one theory to another. 

These dualities came as a bit of a surprise since they are not manifest when formulating string theory. In contrast to this are \emph{symmetries} indeed manifest in the theory. In fact, the manifestation of symmetries in physics has been one of the most important guiding principle during the last century; especially quantum field theory and general relativity are constructed by requiring that the theory is manifestly invariant under certain symmetries. 

In this spirit it is interesting to formulate string/M-theory as a theory which is manifestly invariant under these, somewhat surprising, dualities. This is done by promoting dualities to manifest symmetries. How to do this in general is still under investigation and constitute an exciting area of research. One of the surprising differences one finds is that these extended objects ``probes'' the spacetime geometry differently. This together with manifest duality symmetry leads to, among other things, to the necessity to extend the usual notion of geometry that one is used to. 


\section*{Aim and objectives}
The aim of this thesis is to investigate how one can make these inherent dualities that are present in string/M-theory manifest as symmetries. As explained above this leads to the notion of a generalization of geometry; one of the main things therefore will be to try to understand how this ``geometry'' depends on the underlying symmetry/duality group.

Doing this naturally starts with a deeper understanding of the dualities between the different string theories and M-theory, how they work and why they appear. One important part of this will be to study the low-energy effective field theories describing these theories at low energy. 

Furthermore does the formalism generalize ordinary geometry as well as use some more exotic symmetry algebras that one first encounters when learning group theory. Therefore it is motivated to get a deeper understanding of ordinary geometry as well as group theory and their associated symmetry algebras. Continuing on this we will try to understand how to do this generalization of geometry from a mathematical point of view and how this applies to the different theories discussed. Likewise we continue onto a deeper understanding of symmetries, especially by looking at more exotic algebras such as the exceptional algebras and affine algebras. 

Using this knowledge we continue and try to merge these two fundamental mathematical structures. Especially we will start with an underlying symmetry algebra and see how a generalized geometry can be built, and what properties propagate from the algebra to the geometry. 






%\section*{''Method''}
%\newpage
\bibliographystyle{ieeetr}
\bibliography{references}
\end{document}

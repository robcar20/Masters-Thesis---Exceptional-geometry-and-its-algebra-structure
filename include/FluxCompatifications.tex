\chapter{Flux compactifications}

\section{Phenomenology and moduli stabilization}
As is well-knwon string theory/M-theory lives in a 10/11-dimensional spacetime, yet our everyday experience tells us otherwise. Hence, in order for string theory/M-theory to be able to describe the real world it has to have consistent low-energy description that matches our observations. 

Firstly such a theory should reproduce the standard model of particle physics which has been confirmed to great precision, i.e.\ among other things it should be described by a gauge theory with gauge group $SU(3)\times SU(2)\times U(1)$ together with the correct particle spectra. Another important notion is that of supersymmetry, notably the SM is not supersymmetric in contrast to string theory/M-theory. The latter two are maximally supersymmetric and thus most, or perhaps all, supersymmetry probably needs to be broken. Generically supersymmetry constrains a theory often yielding important physical consequences.

The idea of compactification, where a compact internal manifold is ``small'' compared to the typical size of say strings, is one way to obtain an effective theory in lower dimensions. A generic feature of compactification is presence of scalar fields typically due internal legs of some tensor field. Of especial importance are massless such scalar fields, called moduli. The vacuum expection value of moduli is not fixed and appears as free parameters in the theory. Since there possibly is hundred or thousands of such moduli which determine physical properties such as gauge couplings, Yukawa couplings as well as the size and structure of the internal manifold the theory loses most of its predictive power. Moreover, massless scalar fields would give rise to a long-range force which would invalidate Einstein's equivalence principle. Of course the coupling could be extremely small such that these forces are not experimentally observable. However, this approach comes with a lot of problems \todo{What problems?} and instead one tries to give mass to these scalar fields, this is called \emph{moduli stabilization}. Generating a potential for the scalar fields would imply that the moduli fields are no longer free parameters but rather fixed by the procedure that stabilises the moduli, e.g.\ complicated geometry or presence of fluxes. 

The easiest internal manifold to compactify on is a torus but such a compactification preserves all the supersymmetry, i.e.\ $\mathcal{N}=8$ in four external dimensions. In order two break some supersymmetry we could abandon the torus and compactify on something more complicated, e.g.\ a Calabi-Yau three-fold which we will discuss further below. Another possibility is to introduce to some non-zero flux for some form field 
\begin{equation}
    \int_{\Sigma_{p+1}}C_{p+1}\neq 0,
\end{equation}
corresponding to $p$-brane winding some non-trivial cycle $\Sigma_{p+1}$ in the internal manifold. Such a configuration schematically generates a potential for the moduli through the kinetic term of such a form field 
\begin{equation}
    V\sim \int_{\mathcal{M}_{\text{int}}} \dd C_{p+1}\wedge *_{g}\,\dd C_{p+1}, 
\end{equation}
where $\mathcal{M}_\text{int}$ denotes the internal manifold and the presence of the metric moduli in the hodge dual $*$ is the reason for a potential being generated by expanding $C_{p+1}$ around its vev. 

\todo[inline]{String landscape}

\section{Zero flux compactification}
To begin with we will start by reviewing compactifications on backgrounds without fluxes. The idea is to find vacua (solutions to the equations of motion and the Bianchi identities) such that spacetime seperates into an external spacetime $\mathcal{M}_{\text{ext}}$ and an internal spacetime $\mathcal{M}_{\text{int}}$ such that 
\begin{equation}
    \mathcal{M} = \mathcal{M}_{\text{ext}}\times \mathcal{M}_{\text{int}}.
\end{equation}
However, finding such solutions are generally rather difficult and we will take a different route. Supergravity in $D=11$ has $\mathcal{N}=1$ local supersymmetry, this is a maximal amount of supersymmetry containing $32$ supercharges. A generic vacuum solution will (spontaneuously) break all of this supersymmetry. By restricting to solutions that preserve some amount of supersymmetry, say $\mathcal{N}=1$ in $D=4$, we will find constraints that give a more tractable way to finding vacua. Moreover, the external spacetime is assumed to be Minkowski with possible generalisation to AdS/dS later on. 

Supersymmetry is generated by a supercharge $\mathcal{Q}$ and a parameter $\epsilon$, both being spinors of $SO(1,D-1)$ and a vacuum $|0\rangle$ is supersymmetric if 
\begin{equation}
    \overbar{\epsilon}\mathcal{Q}|0\rangle = 0.
\end{equation}
By taking an arbitrary field $\Theta$ and looking at its variation under the supersymmetry transformation we find 
\begin{equation}
    \langle \delta_{\epsilon}\Theta\rangle = \langle 0|\left[\overbar{\epsilon}\mathcal{Q},\Theta\right]|0\rangle = 0.
\end{equation}
Generally we have that (from now variations should be understood implicitly as valid as expectation values)
\begin{equation}
    \delta_\epsilon \Theta_{\text{boson}} \sim \overbar{\epsilon}\Theta_{\text{fermion}},\qquad \delta_\epsilon \Theta_{\text{fermion}}\sim \epsilon\Theta_{\text{boson}}.
\end{equation}
By the assumption that $\text{dim}\,\mathcal{M}{_{\text{ext}}}=d$ being a $d$-dimensional Minkowski space we have to decompose the spinor representations under $SO(1,d-1)\times SO(D-d)$ where $D$ is the dimension of the full spacetime. Typically these will transform in a non-trivial representation under $SO(1,d-1)$ and hence have to vanish in order to preserve Poincaré symmetry. Thus in order to make sure that the vacuum preserve some amount of supersymmetry we need to find solutions of $\delta_\epsilon\Theta_{\text{fermions}}=0$. Note if such a solution is found we also have to make sure the equations of motion and Bianchi identities are fulfilled.

With a constant dilaton and vanishing fluxes we find for the supersymmetry variation of the gravitino $\psi_M$ (that is always present in supergravity)
\begin{equation}
    \delta_\epsilon\psi_M = \nabla_M\epsilon = 0 \Leftrightarrow \nabla_\mu \epsilon = 0,\qquad \nabla_m\epsilon = 0,
\end{equation}
where we have used that there is no mixing between greek external indices and latin internal indices due to the metric being of a product form. Such a spinor is called a Killing spinor. From this one can show that a necessary condition for the existence of a covariantly constant spinor is that the internal manifold is Ricci-flat (obviously also the external spacetime is Ricci-flat)
\begin{equation}
    R_{mn} = 0.
\end{equation}

Curvature is closely connected to the way an object transforms upon parallel transport. As such that take some object $v$ and parallel transport it in a closed loop on a manifold. Generally the object need not be unchanged by this action but can transform in some representation of $\mathcal{H}\subseteq O(d)$ for a d-dimensional riemannian manifold since the norm does not change assuming a metric compatible connection. The group $\mathcal{H}$ is called the holonomy group. Likewise a spinor parallel transported in a loop transforms in representation of $\mathcal{H}$ and hence, especially, the Killing spinor. But since the Killing spinor is covariantly constant it does not transform and needs to be a singlet under the holonomy group. As such a necessary condition for the existence of a covariantly constant spinor is related to the restriction of the holonomy group and thus put constraints on the metric. Such features as reduction of holonomy and structure group is conveniently described in terms of bundles which motivates a short mathematical interlude.

\subsection{Mathematical interlude -- Fibre bundles}
A short introduction to the notion of fibre bundles are introduced in order to set the terminology which will be convenient for more complicated compactifications later on. Just as a manifold is a topological space that looks locally like $\mathbb{R}^n$, i.e.\ there exists some coordinate chart on $U_i\subseteq \mathcal{M}$ such that $\phi: \mathcal{M}\to \mathbb{R}^n$, a bundle is a topological space that looks locally like a product of two topological spaces. In the cases we are interested these topological spaces will be differentiable manifolds. 

A bundle is a triple $(P,\mathcal{M},\pi)$ typically also denoted $P\overset{\pi}{\longrightarrow}\mathcal{M}$, where $P$ is the total space, $\mathcal{M}$ the base space and $\pi: P\to \mathcal{M}$ the projection which is a surjection onto the base space. The notion of a bundle is a very general concept and we will instead only be interested in (differentiable) fibre bundles. A (differentiable) fibre bundle $(P,\mathcal{M},\pi,F,G)$ where $F$ is the fibre and $G$ the structure group. In order for a fibre bundle to ``look locally'' like a product manifold there exists on patches $U_i\subseteq \mathcal{M}$ diffemorphisms $\phi_i: U_i\times F\to \pi^{-1}(U_i)$ such that $\pi\circ \phi_i(u,f)=u$, where $u\in U_i$, $f\in F$ and $\pi^{-1}(U_i)=F_p\cong F$ is the fibre at $u\in U_i$. The maps $\phi_i$ are usually called local trivializations since the map $\phi^{-1}\circ \pi^{-1}$ maps any $U_i$ to $U_i\times F$ which ``looks like a product manifold''. In order for this to make sense there needs to be some compatibility on overlapping patches $U_i\cap U_j\neq \emptyset$. Take a point $p\in P$ such that $\pi(p)=u$ with $u\in U_i\cap U_j$ then applying the two local trivialisations $\phi^{-1}_i$ and $\phi^{-1}_j$ we find two points in $P$ according to 
\begin{equation}
    \phi^{-1}_i(u) = (u,f_i)\qquad \text{and}\qquad \phi^{-1}_j(u) = (u,f_j).
\end{equation}
Obviously $f_i = \phi^{-1}_i\circ \phi_j (f_j)$ where $\phi^{-1}_i\circ \phi_j = t_{ij}(u)$ is called a transition function $t_{ij}(u):F\to F$ defined trough a left action of the structure group $G$.\todo{left action of G} 

A principal fibre bundle ($P$,$\mathcal{M}$,$\pi$,$G$) is a fibre bundle with a typical fibre being the structure group $G$ itself, i.e.\ for some $u\in U\subseteq \mathcal{M}$ $\pi^{-1}(u)\cong G$. Moreover, there is a right action of $G$
\begin{equation}
    \triangleleft: P\times G \to P,
\end{equation}
which is free. An important example of a principle fibre bundle is the frame bundle. The frame bundle is a fibre bundle with a typical fibre consisting of all ordered basis, such a basis being a frame, of the tangent space. 

The existence of invariant tensors on $\mathcal{M}$ is equivalent to adding further structure to a principal $G$-bundle such that it is possible to impose a restriction to a principal $H$-bundle, where $H\subset G$ with the same base manifold. Typically this is defined the other way around,  


In order to define a reduced structure group we need a so-called bundle map. A bundle map is a linear map $\rho$ such that the following diagram commutes
\begin{center}
\begin{tikzpicture}
  \matrix (m) [matrix of math nodes,row sep=3em,column sep=4em,minimum width=2em]
  {
     F_t(x) & F(x) \\
     A_t & A \\};
  \path[-stealth]
    (m-1-1) edge node [left] {$\mathcal{B}_X$} (m-2-1)
            edge [double] node [below] {$\mathcal{B}_t$} (m-1-2)
    (m-2-1.east|-m-2-2) edge node [below] {$\mathcal{B}_T$}
            node [above] {$\exists$} (m-2-2)
    (m-1-2) edge node [right] {$\mathcal{B}_T$} (m-2-2)
            edge [dashed,-] (m-2-1);
\end{tikzpicture}
\end{center}

\todo{Bundle maps, when reduced structure group, Berger's list}


\subsection{Compactification on Calabi-Yau}
Having reviewed the notion of reduced structure group and special holonomy a Calabi-Yau n-fold $CY_{n}$ can be defined as a $2n$-dimensional compact Riemannian manifolds with holonomy contained in $SU(N)$\cite{Blumenhagen2013}. We are particularly interested in $CY_3$ such that starting in $D=10$ supergravity, or string theory, the external spacetime is 4-dimensional. As such we know that for a $CY_3$ there exists some invariant p-forms which can be found by looking at the decomposition of 1-,2- and 3-forms of $SO(6)\cong SU(4)$ under $\mathcal{H}=SU(3)$
\begin{align*}
    \mathbf{6}\to \mathbf{3}+\mathbb{\overbar{3}},\\
    \mathbf{15}\to \mathbf{8}+\mathbb{3}+\overbar{\mathbf{3}}+\mathbf{1},\\
    \mathbf{20}\to \mathbf{6}+\overbar{\mathbf{6}}+\mathbf{3}+\overbar{\mathbf{3}}+2\cdot\mathbf{1}.
\end{align*}
There thus exists a real 2-form $\omega$ and a complex 3-form $\Omega$ that are singlets under $\mathcal{H}$, such pair of nowhere vanishing tensors defines a \emph{structure}. Note that we already know that such a structure is equally well defined by a nowhere globally defined spinor $\mathbb{4}$ of $SU(4)\cong SO(6)$ that decomposes under $\mathcal{H}$ as 
\begin{equation}
    \mathbf{4}\to \mathbf{3}+\mathbf{1}.
\end{equation}
What follows will mostly consider structures defined by some bosonic forms such as ($\omega,\Omega$) but from the observation above we know that this corresponds to a Killing spinor which in turn depends on preserving some amount of supersymmetry. Moreover, we see that starting with $\mathcal{N}$ supersymmetries in the full theory compactification on a $CY_3$ yields a theory which again has $\mathcal{N}$ supersymmetries.

\section{Flux compactifications and generalised geometry}
In the case of vanishing flux we have seen that the Killing spinor equation led to special holonomy, which in put restrictions on the connection of the internal manifold. In the presence of fluxes the Killing spinor is no longer covariantly constant, however, the existence of globally defined nowhere vanishing spinor still implies a reduced structure group. This is summarized by 

\begin{align*}
    \text{vanishing flux}&\Longleftrightarrow \text{special holonomy},\\
    \text{non-zero flux}&\Longleftrightarrow \text{reduced structure group}.\\
\end{align*}
In the presence of fluxes the Killing spinor equations of type II supergravity are given by \cite{Blumenhagen2013}
\begin{equation}
    \delta_\epsilon \psi_M = \left(\nabla_M+\frac{1}{8}\Gamma^{NP}H_{MNP}\mathscr{P}+\frac{1}{16}\ee^\Phi\sum_{n}\slashed{F}_n\Gamma_M\mathscr{P}_n\right)\epsilon,
\end{equation}
\begin{equation}
    \delta_\epsilon \lambda = \left(\slashed{\pd}\Phi+\frac{1}{12}\Gamma^{MNP}H_{MNP}\mathscr{P}+\frac{1}{8}\ee^\Phi\sum_{n}(-1)^n(5-n)\slashed{F}_n\mathscr{P}_n\right)\epsilon,
\end{equation}
where $\mathscr{P}$ and $\mathscr{P_n}$ are $2\times 2$-matrices, $\slashed{F}_n=\frac{1}{n!}\Gamma^{M_1M_2\ldots M_n}F_{M_1M_2\ldots M_n}$ and $n\in\{0,1,\ldots,10\}$ is even(odd) for type IIA(B). In the case of vanishing flux and a constant dilation this is seen to reduce to a covariantly constant spinor $\epsilon$ as discussed above. 

The obstruction to the Killing spinor being covariantly constant is typically discussed in terms of intrinsic torsion. However, there is always possible to construct a connection $\nabla$ with torsion such that the Killing spinor is covariantly constant w.r.t. to that connection. Thus consider a connection $\nabla=\nabla'-\kappa$ such that $\nabla_n V_m=\pd_nV_m-\tensor{\Gamma}{^'_{nm}^p}V_p-\tensor{\kappa}{_{nm}^p}V_p$, where $\nabla'$ is the Levi-Civita connection and $\kappa$ is the so-called contorsion tensor. Generically the connection coefficents $\tensor{\Gamma}{_{mn}^p}$ take values in $\Lambda^1\otimes\mathfrak{g}$, where $\lambda^1$ is the one-form representation of $G=GL(d)$ and $\mathfrak{g}=\mathfrak{gl}(d)$ is the corresponding Lie algebra. Assuming metric compatibility fixes the ``symmetric'' part of the pair $_m^p$ which is contained in the Levi-connection. Note that the Levi-Civita connection have extra terms such that the torsion $\tensor{\Gamma}{_{[mn]^p}}$ vanishes. This implies that the contorsion take values in 
\begin{equation}
    \tensor{\kappa}{_{mn}^p}\in \Lambda_1\otimes \mathfrak{h},
\end{equation}
where $\mathfrak{h}=so(d)$ is the maximal compact subalgebra of $\mathfrak{gl}(d)$ (since this is the ``anti-symmetric'' part). Assuming a reduced structure group $G'\subseteq SO(d)$ implies that we can separate the contorsion into two parts 
\begin{equation}
    \kappa = \kappa_0+\kappa_{\text{int}}\quad \text{where} \quad \kappa_{\text{int}}\in \Lambda_1\otimes \mathfrak{g}'\quad \text{and}\quad \kappa_{0}\in \Lambda_1\otimes \tensor{\mathfrak{g}}{^'^{\perp}}.
\end{equation}
The point now being that since the Killing spinor is a singlet under $G'$ the failure of the of Killing spinor being covariantly constant \emph{with respect to the Levi-Civita connection} is given by $\kappa_0$. The part $\kappa_0$ can further be decomposed into irreducible representations of $\mathfrak{g}'$ called torsion classes. As an example take $G=GL(6)$ and $G'=SU(3)$. Under $G'$ we have the following decompositions $\Lambda_1\to \mathbf{3}\oplus\overbar{\mathbf{3}}$, $\mathfrak{g}'=\mathbf{8}$ and $\tensor{\mathfrak{g}}{^'^{\perp}}=\mathbf{1}\oplus\mathbf{3}\oplus\overbar{\mathbf{3}}$ and 
\begin{equation}
    \kappa_0 \in \bigoplus_{i=1}^5 \mathscr{W}_i,
\end{equation}
where $\mathscr{W}_i$ are the five torsion classes given by 
\begin{align*}
    \mathscr{W}_1 = \mathbf{1}\oplus\mathbf{1},\\
    \mathscr{W}_2 = \mathbf{8}\oplus\mathbf{8},\\
    \mathscr{W}_3 = \mathbf{6}\oplus\overbar{\mathbf{6}},\\
    \mathscr{W}_4 = \mathbf{3}\oplus\overbar{\mathbf{3}}.\\
\end{align*}







\documentclass{article}
\usepackage[utf8]{inputenc}
\usepackage{standard}

\title{Extended Geometry \\ Planning report -- Master Thesis}

\author{Robin Karlsson}
\date{\today}

\begin{document}

\maketitle

\section{Background}
During the last century fundamental physics has been governed by two main theories: the standard model of particle physics and the theory of general relativity. In the standard model the strong, weak and electromagnetic force are brought together to describe interactions between matter through the language of quantum field theories. Gravity on the other hand is a classical theory, where the ``force'' is due to the curvature of spacetime. Treating gravity separately from the other three forces leads to inconsistencies in certain extreme systems, as for example was shown by Hawking who looked at quantum effects of black holes. A unification of all forces has thus been an aim for fundamental physics research for a long time. 

String theory is perhaps the most prominent candidate for a theory of quantum gravity, where the fundamental objects are extended strings rather than pointlike particles. There are five consistent string theories and it was once thought that one of these would describe our physical world. However, it was shown that they are related to each other by certain duality transformations, and, moreover, they are all connected to a theory living in one dimension higher, M-theory. At present the full M-theory remains rather unknown while its formulation at low energies can be described. Finding the complete theory presents many challenges but the duality symmetries might be one way to learn more about it. 

%A common feature among these frameworks are that their formulation relies heavily on the notion of certain symmetries.

\section{Method and project plan}
Both general relativity and the standard model relies heavily on certain symmetries, theses are crucial for a consistent formulation of the theory, and the same is certainly also true for string theory/M-theory. The main goal of this thesis is therefore to study the symmetries of string theory/M-theory, especially the so called duality symmetries. These are interesting for many reasons, among other things certain dualities exchange perturbative excitations with solitons thus telling us something about the non-perturbative part of the theory. Moreover, some duality transformations exchange gravitational degrees of freedom with other degrees of freedom. In the spirit of unification we therefore construct an ``extended geometry'', which is a generalisation of geometry in order to make duality symmetries manifest.   

A common construction of extended geometries, where the geometry is based on a more general symmetry algebra, has recently been found. The first part of the thesis thus aims to study this general construction as well as specific theories such as double field theory (DFT) and exceptional field theory (EFT), which are examples of such extended geometries. Knowledge of DFT and EFT serve as valuable knowledge in construction of the general case as well as being the physically interesting theories related to string theory/M-theory. As of yet the general construction are restricted to ``internal'' dimensions whereas both DFT and EFT also has been formulated including the ``external'' space. 

The first part of the thesis thus consists of reproducing important results of the theories discussed above, which demands knowledge in several different fields of physics and math, such as field theory, general relativity, string theory and Lie algebras among others.  

The second part of the thesis aims to apply acquired knowledge to an interesting sector of the theory. At present we do not wish to restrict ourselves to what specific area will be subsumed, instead we list a few topics of interest.

\begin{itemize}
    \item Algebras and superalgebras in connection with extended geometry.
\end{itemize}
An extended geometry is based on a so called Kac-Moody algebra and the specific features of the theory relies on this algebra. Certain extensions of a given algebra turns out to possess important information about the theory. The rôle of such extended algebras are therefore of interest.

\begin{itemize}
    \item Non-geometric solutions in string theory
\end{itemize}
Background fields, such as the gravitational field and the gauge field, in overlapping patches of spacetime are typically connected by diffemorphisms and/or gauge transformations. However, there exists solutions that on overlapping patches are related by duality transformations on the background fields, these are called non-geometric solutions. Such solutions, as far as we are aware, are novel to these theories and are interesting to investigate further. 

\begin{itemize}
    \item Supersymmetric extended geometry.
\end{itemize}
Extensions of extended geometries to supergeometries, were the underlying algebra is replaced by a superalgebra.

\begin{itemize}
    \item Large transformations and discrete duality symmetries.
\end{itemize}
The general framework of extended geometries is currently a local description. In DFT, however, finite (large) transformations seems to be under control, while in EFT they are rather difficult to deal with at present. This would be interesting to get a grip on in order to learn something about global properties of the theory.

\begin{itemize}
    \item Consistent truncations and gauged supergravities.
\end{itemize}
The low-energy effective description of string theory compactifications in the presence of fluxes are the so-called gauged supergravities. These theories can be described naturally starting from a manifestly duality covariant theory.



\newpage
\bibliographystyle{ieeetr}
\bibliography{references}
\end{document}

\begin{itemize}
    \item The section condition
\end{itemize}
In extended geometries one extends the coordinates to lie in some module of the duality group, in DFT this corresponds to a doubling of coordinates to accommodate for coordinates dual to winding. To recover the original gravity, as well as to make sure that the generalised diffemorphisms closes, one applies a constraint on the momenta, the section condition. It is mainly motivated by analogy with translational invariance on strings (level matching) and due to its fundamental importance it would be interesting to learn more about why it is needed, and if it is possible to relax the constraint in some way.


1. Inledning
1.1 Bakgrund
I detta avsnitt beskrivs bakgrunden till frågeställningen, det vill säga en kort beskrivning av företagets situation och varför företaget vill ha uppdraget utfört. Observera att det inte är bakgrunden till att studenten gör examensarbetet som ska beskrivas.
 
1.2 Syfte
Syftet är en kort beskrivning av uppdraget och vilket resultat uppdraget ska leda till.
 
1.3 Avgränsningar
Under avgränsningar talar man om vad man inte ska behandla.
 
1.4 Precisering av frågeställningen
Utifrån "Syfte" ska frågeställningen preciseras, vilket exempelvis kan göras genom att skriva ett antal frågor som ska besvaras. Ett annat sätt är att skriva ett antal påståenden (hypoteser) som sedan ska verifieras eller förkastas under arbetets gång.
 
2. Metod
Metodkapitlet beskriver hur man lägger upp arbetet. Vilket bland annat omfattar arbetsgång, design av experiment och användning av olika metoder för datainsamling. Ett metodkapitel ska, i idealfallet, vara så utförligt att vem som helst med vissa baskunskaper inom området ska kunna utföra arbetet på det sätt som beskrivs i rapporten och nå samma resultat. Att beskriva metoden är viktigt för att uppdragsgivaren ska kunna bedöma om man kan nå målet på det föreskrivna sättet. Därför är det också viktigt att man förklarar varför den valda metoden ger ett tillförlitligt resultat.
 
3. Tidplan
Tidplanen presenteras lämpligen som ett Gantt-schema (flödesschema).

String theory/M-theory is a theory of extended objects which, like many other models, exhibits duality symmetries, under which solitons and perturbative excitations are interchanged. The solitons in string theory are typically branes winding on non-trivial cycles. Extended geometry is a collective term for any model that tries to give a “geometric” origin of these symmetries. Since the symmetries mix gravitational and other degrees of freedom, the notion of “geometry” must be generalised appropriately. Typically, the coordinates are extended to belong in a representation of the duality group. All basic concepts of geometry then need to be revisited and revised in order to obtain a consistent formulation describing a subsector of string theory related to massless fields. The most important classes of extended geometries are double field theory (DFT), based on T-duality, and exceptional field theory (EFT), based on U-duality. Recently, a unified framework for all extended geometries has been discovered.

The first aim of the thesis work is to study the literature on extended geometry, with focus on the unified framework mentioned above, and to reproduce important results. This requires applications of field theory, gravity, Lie algebra and representation theory. Furthermore, string theory is needed to understand the motivation and origin of the problem. About half of the time will be spent on this part. 
The second aim is to apply the formalism to one of the following areas:
\begin{itemize}
\item Algebras and superalgebras in connection with extended geometry
\item Consistent truncations and gauged supergravities
\item Non-geometric solutions in string theory 
\item Supersymmetric extended geometry
\item Large transformations and discrete duality symmetries
\end{itemize}
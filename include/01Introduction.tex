\chapter{Introduction}
During the 20th century fundamental physics split into two overarching categories: quantum physics governing the subatomic world through the strong-, weak- and electromagnetic force and the theory of general relativity describing gravity. The latter tells the story of gravity warping spacetime such that an object in it follows a geodesic, the shortest path in a curved space. On the other hand we have the other three forces. These are described by Quantum field theories, were particles arise as excitations of a quantized field. This framework works extraordinarily well to high precision tested by particle accelerators around the world, e.g.\ at Large Hadron Collider at CERN Geneva.

The theory of general relativity allows for black hole solutions, which are an object so dense that nothing can escape its gravitational pull once it enters inside the so called event horizon. These strange objects that have created a lot of controversy during the last century have recently, roughly 100 years after the discovery of general relativity, been discovered directly through gravitational waves emitted from two merging black holes. Indirectly they have been observed and puzzled people for quite some time. At the moment little is known about the fate of the singularity inside a black hole.

Quantum field theory is conventionally placed in a flat spacetime, in agreement with Einsteins ingenious equivalence principle stating roughly that in a sufficiently small region spacetime looks flat and the fact that the force of gravity is many order of magnitudes weaker than the other forces. However, a black hole presents a region of spacetime where gravity is very strong due its immense density and one should therefore start to worry about quantum effects and gravity being of similar strength. Stephen Hawking was one who thought about this and reached the conclusion that quantum effects would lead evaporation of the black hole, in stark contradiction to the classical result.

Except for gravity the development has been to take a theory, say electromagnetism coupling to matter, and quantize fields of the theory. One would therefore hope to do the same with the gravitational field, however, this turns out to be incredibly hard and pathological. Quantizing relativity turns out to give a so called non-renormalizable theory unable to predict physical observables in contrast the renormalizable theories of the other forces. 

A common tool for fundamental physics during the last 100 years have been the use of symmetries. The standard model of particle physics relies on the gauge group $SU(3)\times SU(2)\times U(1)$, without this symmetry the theory would lose its predictive power. In fact, the particle spectrum is in some sense determined by ensuring that these symmetries hold non-perturbatively through so-called anomaly cancellations. General relativity on the other hand is governed by the symmetry of diffeomorphism invariance, which roughly speaking implies that the laws of physics looks the same in an arbitrary coordinate system. 

The quantization of gravity, quantum gravity, has been an ongoing field of research for many years. One of the most promising frameworks are string theory. String theory replace the notion of the most fundamental objects as pointlike to be a 1-dimensional extended object, a string. Oscillations of strings are interpreted as different particles, of which it turns out that the graviton, the particle mediating gravity, is present in any consistent string theory. Extended objects perceive the physical world quite different than particles which seems to make the theory more well behaved than that for their pointlike counterparts. However, one striking implication of string theory is that it can be consistently formulated in ten spacetime dimensions only, which seems to be in contradiction with our everyday experience of four dimensions. One way of dealing with this is the idea of compactification, where certain spacelike directions are ``small'' compared to the others, such that the effective dimension perceived is lower. 

Five consistent theories of string has been developed: Type I, Type IIA/B and two heterotic string theories. These were thought to be five inequivalent theories and the hope was that one of them should describe our physical world. However, it was argued by Witten \cite{WittenDualities1995} that they all were in fact related to each other in certain ways, through dualities. Even more so, their description at low energies could all be related to a theory in 11 dimensions dubbed M-theory. This theory in one dimension higher is known at low energies while the full theory still remains unknown.


Dualities mix gravity and other degrees of freedom which points to a unification amongst them. Due to this the notion of geometry needs to be generalised in order to account for non-gravitational degrees of freedom as well. Extended geometries in this thesis describe a generalisation of ordinary geometry to include dualities as manifest symmetries of the theory. 



\begin{figure}
    \BigPicture{1}
    \caption{Caption}
    \label{fig:my_label}
\end{figure}


\section{Outline}
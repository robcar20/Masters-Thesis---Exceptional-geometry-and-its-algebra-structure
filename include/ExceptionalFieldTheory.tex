\chapter{Exceptional field theory}

Double field theory aimed to make T-duality a manifest symmetry of supergravity. In the same spirit exceptional field theory makes the U-duality manifest. The duality groups upon compactification of supergravity on a $d$-dimensional torus is $E_{d(d)}$. One inherent difference thus compared to DFT is that the behaviour of the duality group highly dependent on the dimensions one is looking at, making the general analysis more complicated. Moreover, for $d\geq 9$ the duality groups become infinite-dimensional. There is a conjecture by West \cite{West2011} that the full M-theory possess the $E_{11}(\mathbb{Z})$ as a symmetry. 

In this chapter we will review the construction of exceptional field theory. In section \ref{sec:Uduality} we review the U-duality group before moving on the construction of $E_{6(6)}$ exceptional field theory which will folllow \cite{E62014}. For a similar construction for $E_{7(7)}$ and $E_{8(8)}$, see \cite{E72014,E82014}. 



\section{U-duality}\label{sec:Uduality}



\section{$E_{6(6)}$ Exceptional field theory}
\section{Closure of generalised diffeomorphisms}
In this appendix we provide the explicit calculation of closure of generalised diffeomorphisms. We want to look at $[\mathscr{L}_\xi,\mathscr{L}_\eta] V^M$ and start with one term 
\begin{equation}
    \begin{aligned}
        \mathscr{L}_{\xi}(\mathscr{L}_\eta V^M) &= \mathscr{L}_\xi(\eta^P\pd_PV^M-\pd_P\eta^MV^P+\tensor{Y}{^{MR}_{ST}}\pd_R\eta^S V^T)\\
        &= (\mathscr{L}_\xi \eta^P)\pd_PV^M+\eta^P\mathscr{L}_\xi(\pd_PV^M)-(\mathscr{L}_\xi\pd_P\eta^M)V^P-\pd_P\eta^M\mathscr{L}_\xi V^P\\
        &+\tensor{Y}{^{MR}_{ST}}(\mathscr{L}_\xi\pd_R\eta^S)V^T+\tensor{Y}{^{MR}_{ST}}\pd_R\eta^S\mathscr{L}_\xi V^T.
    \end{aligned}
\end{equation}
The first term gives 
\begin{equation}
    \begin{aligned}
        (\mathscr{L}_\xi\eta^P)\pd_PV^M &= \xi^K\pd_K\eta^P\pd_PV^M-\pd_K\xi^M\pd_PV^K+\pd_K\xi^K\pd_KV^M\\
        &+\tensor{Y}{^{MR}_{ST}}(\pd_R\xi^S)\pd_PV^T-\underbrace{\tensor{Y}{^{TR}_{SP}}(\pd_R\xi^S)\pd_TV^M}_{=0\,\text{section condition}}.
    \end{aligned}
\end{equation}
The second term gives 
\begin{equation}
    \begin{aligned}
        \eta^P(\mathscr{L}_\xi\pd_PV^M) &= \eta^P\xi^K\pd_K\pd_PV^M-\eta^P\pd_K\xi^M\pd_PV^K+\eta^P\pd_P\xi^K\pd_KV^M\\
        &+ \tensor{Y}{^{MR}_{ST}}(\pd_R\xi^S)\pd_PV^T-\underbrace{\tensor{Y}{^{TR}_{SP}}(\pd_R\xi^S)\pd_TV^M}_{=0\, \text{section condition}},
    \end{aligned}
\end{equation}
where we note that the first terms is symmetric in $\xi\leftrightarrow\eta$ and will vanish in the commutator. The third term gives 
\begin{equation}
    \begin{aligned}
        -(\mathscr{L}_\xi\pd_PV^M)V^P&=-\xi^K\pd_K\pd_P\eta^MV^P+\pd_K\xi^M\pd_P\eta^KV^P-\pd_P\xi^K\pd_K\eta^MV^P\\
        &-\tensor{Y}{^{MR}_{ST}}(\pd_R\xi^S)\pd_\eta^TV^P+\underbrace{\tensor{Y}{^{TR}_{SP}}(\pd_R\xi^S)\pd_TV^P}_{=0\, \text{section condition}}.
    \end{aligned}
\end{equation}
The fourth term is given by 
\begin{equation}
    -\pd_P\eta^M\mathscr{L}_\xi V^P = -\pd_\eta^M\xi^K\pd_KV^P+\pd_P\eta^M\pd_K\xi^PV^K-\underbrace{\pd_P\eta^M\tensor{Y}{^{PR}_{ST}}(\pd_R\xi^S)\pd_P\eta^M}_{=0\, \text{section condition}}.
\end{equation}
The fifth term is given by 
\begin{equation}
    \begin{aligned}
        \tensor{Y}{^{MR}_{ST}}(\mathscr{L}_\xi \pd_R\eta^S)V^T &= \tensor{Y}{^{MR}_{ST}}V^T\left(\xi^K\pd_K\pd_R\eta^S-\pd_K\xi^S\pd_R\eta^K+\pd_R\xi^K\pd_K\eta^S\right)\\
        &+\tensor{Y}{^{MR}_{ST}}V^T\big(\tensor{Y}{^{SK}_{LQ}}(\pd_K\xi^L)\pd_R\eta^Q-\underbrace{\tensor{Y}{^{QK}_{LR}}(\pd_K\xi^L)\pd_Q\eta^S}_{=0\, \text{section condition}}\big).
    \end{aligned}
\end{equation}
And finally the last term is given by 
\begin{equation}
    \begin{aligned}
        \tensor{Y}{^{MR}_{ST}}(\pd_R\eta^S)\mathscr{L}_\xi V^T = \tensor{Y}{^{MR}_{ST}}\left(\pd_R\eta^S\xi^K\pd_KV^T-\pd_K\xi^TV^K+\tensor{Y}{^{TK}_{QP}}(\pd_K\xi^Q)V^P\right).
    \end{aligned}
\end{equation}
We also want to calculate $\frac{1}{2}(\mathscr{L}_{\mathscr{L}_\xi\eta}-\mathscr{L}_{\mathscr{L}_\eta\xi})V^M$ to this end calculate
\begin{equation}
    \mathscr{L}_{\mathscr{L}_\xi \eta}V^M = \mathscr{L}_{\xi^K\pd_K\eta}V^M-\mathscr{L}_{\pd_K\xi\eta^K}V^M+\mathscr{L}_{\tensor{Y}{^{\circ R}_{ST}}\pd_R\xi^S\eta^T}V^M.
\end{equation}
The first term gives 
\begin{equation}
    \begin{aligned}
        \mathscr{L}_{\xi^K\pd_K\eta}V^M &= \xi^K\pd_K\eta^P\pd_PV^M-\left(\pd_P\xi^K\pd_K\eta^M+\xi^K\pd_K\pd_P\eta^M\right)V^P\\
        &+\tensor{Y}{^{MR}_{ST}}\left(\pd_\xi^K\pd_K\eta^S+\xi^K\pd_K\pd_R\eta^S\right)V^T.
    \end{aligned}
\end{equation}
The second term gives 
\begin{equation}
    \begin{aligned}
        -\mathscr{L}_{\pd_K\xi\eta^K}V^M &= -\pd_K\xi^P\eta^K\pd_PV^M+\left(\eta^K\pd_K\pd_P\xi^M+\pd_K\xi^M\pd_P\eta^K\right)V^P\\
        &-\tensor{Y}{^{MR}_{ST}}\left(\pd_R\pd_K\xi^S\eta^K+\pd_K\xi^S\pd_R\eta^K\right)V^T.
    \end{aligned}
\end{equation}
And the last term gives 
\begin{equation}
    \begin{aligned}
        \mathscr{L}_{\tensor{Y}{^{\circ R}_{ST}}\pd_R\xi^S\eta^T}V^M &= \underbrace{\tensor{Y}{^{KR}_{ST}}(\pd_R\xi^S)\eta^T\pd_KV^M}_{=0\, \text{section condition}}-\tensor{Y}{^{MR}_{ST}}V^K\left(\pd_K\pd_R\xi^S\eta^T+\pd_R\xi^S\pd_K\eta^T\right)\\
        &+\tensor{Y}{^{MQ}_{LK}}\tensor{Y}{^{LR}_{ST}}V^K\left(\pd_Q\pd_R\xi^S\eta^T+\pd_R\xi^S\pd_Q\eta^T\right).
    \end{aligned}
\end{equation}
It is then straightforward but tedious to calculate $[\mathscr{L}_\xi,\mathscr{L}_\eta] V^M-\frac{1}{2}(\mathscr{L}_{\mathscr{L}_\xi\eta}-\mathscr{L}_{\mathscr{L}_\eta\xi})V^M$ and one finds that the contribution from the terms proportional to $\pd^2\xi$ is given by
\begin{equation}
    \frac{1}{2}\left(\tensor{Y}{^{MR}_{ST}}\pd_K\pd_R\xi^S-\tensor{Y}{^{MQ}_{LK}}\tensor{Y}{^{LR}_{ST}}\pd_Q\pd_R\xi^S\right)V^K\eta^T = -\frac{1}{2}\tensor{Z}{^{MQ}_{LK}}\tensor{Y}{^{LR}_{ST}}\pd_Q\pd_R\xi^SV^T\eta^K,
\end{equation}
which is not but an anciallary transformation. The remaining terms are proportional to $(\pd\xi\pd\eta)$ and given by 
\begin{equation}
    \begin{aligned}
        \frac{1}{2}\tensor{Y}{^{MR}_{ST}}\tensor{Y}{^{SK}_{LQ}}\pd_K\xi^L\pd_R\eta^QV^T+\tensor{Y}{^{MR}_{ST}}\tensor{Y}{^{TK}_{QP}}\pd_R\eta^S\pd_K\xi^QV^P=\\
        -\frac{1}{2}\tensor{Y}{^{MR}_{ST}}\pd_K\xi^T\pd_R\eta^SV^K-\tensor{Y}{^{MR}_{ST}}\pd_K\xi^S\pd_R\eta^KV^T-(\xi\leftrightarrow\eta) \overset{!}{=}0.
    \end{aligned}
\end{equation}
The identities that the $Y$ tensor need to satisfy given in \eqref{eq:closure1}-\eqref{eq:closure2} are then easily derived from this expression. 


\chapter{Conclusions}
String theory/M-theory are the most prominent frameworks of quantum gravity, but there is still much about these theories that we do not know. Especially the notion of describing extended objects has led to interesting dualities such as $T$-duality and $U$-duality. As we have argued the corresponding low-energy effective theories, supergravity theories, have related ``hidden'' symmetry groups when compactified on torii. The rôle of these symmetries in M-theory is the motivation for this thesis and for developing extended geometries. 

Gravitational degrees of freedom and non-gravitational degrees of freedom are typically mixed under these dualities and hence a covariant formalism suggest a merger of the underlying fields leading to the notion of extended geometries. This extension is done by introducing generalised diffeomorphisms and demanding that fields transform covariantly under such transformations. The latter is equivalent to the closure of the algebra of diffeomorphisms which in turn demands the section constraint, i.e.\ a local embedding of ordinary geometry. One can then construct an action invariant under generalised diffeomorphisms that encodes the dynamics of a generalised metric. Moreover, it is also possible to construct geometrical objects such as a covariant derivative, torsion and a generalised Ricci tensor which characterise the extended geometry. 

With a global solution to the section constraint and a ``twisting'' by form fields an extended geometry reduces to what is called generalised geometry. This formalism is potent in describing, among other things, compactifications of supergravity with fluxes. Including fluxes typically invalidates the geometrical tools that works well to describe compactifications without fluxes. Since extended geometries merge gravitational and non-gravitational degrees of freedom appropriate generalisations appear naturally and compactifications can dealt with using generalised geometrical tools. 

In chapter \ref{chap:ExtendedGeometries} a covariant formalism was introduced as well as a generalised Ricci tensor. The latter is a candidate for the equations of motion for the generalised metric and it would be interesting to compare this with the equations of motion derived from the action introduced in the same chapter. In order to do this one should find a metric-compatible and torsion-free connection in terms of the generalised metric. However, these two conditions do not fully specify the connection and we were not able to find such an expression for a general structure algebra. This would be an interesting direction for future studies. 

Another pressing issue that would be interesting to solve is that of the global structure and global transformations. As of yet the extended geometries are generally restricted to the local behaviour, exceptions are for example double geometries. Another interesting question is that of the section constraint which we have seen play a crucial part in defining a consistent theory. A truly different theory would be obtained if the section constraint could be dropped completely. 

Extended geometries ultimately presents an interesting way to merge gravitational and non-gravitational degrees of freedom in order to exploit a larger part of the underlying symmetry of M-theory. The full rôle of extended geometries in a complete formulation of M-theory remains to be seen, but unification and symmetries have been incredibly successful guiding tools to a better understanding of the fundamental laws of nature more than once before. 
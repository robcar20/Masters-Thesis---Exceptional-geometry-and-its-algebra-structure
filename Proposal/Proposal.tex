\documentclass{article}
\usepackage[utf8]{inputenc}
\usepackage{standard}

\title{Extended geometry\\
\large{MSc thesis proposal\\Spring term 2018}}
\author{Robin Karlsson}
\date{\nodate}

\begin{document}
\maketitle

\pagenumbering{gobble} 

String theory/M-theory is a theory of extended objects which, like many other models, exhibits duality symmetries, under which solitons and perturbative excitations are interchanged. The solitons in string theory are typically branes winding on non-trivial cycles. Extended geometry is a collective term for any model that tries to give a “geometric” origin of these symmetries. Since the symmetries mix gravitational and other degrees of freedom, the notion of “geometry” must be generalised appropriately. Typically, the coordinates are extended to belong in a representation of the duality group. All basic concepts of geometry then need to be revisited and revised in order to obtain a consistent formulation describing a subsector of string theory related to massless fields. The most important classes of extended geometries are double field theory (DFT), based on T-duality, and exceptional field theory (EFT), based on U-duality. Recently, a unified framework for all extended geometries has been discovered.

The first aim of the thesis work is to study the literature on extended geometry, with focus on the unified framework mentioned above, and to reproduce important results. This requires applications of field theory, gravity, Lie algebra and representation theory. Furthermore, string theory is needed to understand the motivation and origin of the problem. About half of the time will be spent on this part. 
The second aim is to apply the formalism to one of the following areas:
\begin{itemize}
\item Algebras and superalgebras in connection with extended geometry
\item Consistent truncations and gauged supergravities
\item Non-geometric solutions in string theory 
\item Supersymmetric extended geometry
\item Large transformations and discrete duality symmetries
\end{itemize}

%\bibliographystyle{ieeetr}
%\bibliography{references}
\end{document}
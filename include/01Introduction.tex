\chapter{Introduction}
During the 20th century fundamental physics has been split into two overarching categories: quantum physics governing the subatomic world through the strong, weak and electromagnetic forces and the theory of general relativity describing gravity. The latter tells the story of gravity warping space-time such that an object in it follows the shortest path in this curved space-time. On the other hand we have the other three forces. These are described by quantum field theories in which particles arise as excitations of quantized fields. This framework works extraordinarily well to high precision tested by particle accelerators around the world, e.g.\ at Large Hadron Collider at CERN Geneva. Quantum field theory is conventionally placed in a flat space-time, in agreement with Einsteins ingenious equivalence principle stating roughly that in a sufficiently small region space-time looks flat and the fact that the force of gravity is many order of magnitudes weaker than the other forces. However, a black hole presents a region of space-time where gravity is very strong due its immense density and one should therefore start to worry about quantum effects and gravity being of similar strength. Stephen Hawking was one who thought about this and reached the conclusion that quantum effects would lead to evaporation of the black hole, in stark contradiction to the classical result. Except for gravity the development has been to take a theory, say electromagnetism coupled to matter, and quantize the fields of the theory. One would therefore try to do the same with the gravitational field. However, this turns out to be incredibly hard and pathological. Quantizing relativity turns out to give a non-renormalisable theory unable to predict physical observables in contrast the renormalisable theories of the other forces. 

A common tool for fundamental physics during the last 100 years has been the use of symmetries. The standard model of particle physics relies on the gauge group $SU(3)\times SU(2)\times U(1)$ and without this symmetry the theory would lose its predictive power. In fact, even the particle spectrum is in some sense determined, or at least restrained, by ensuring that these symmetries hold non-perturbatively through so-called anomaly cancellations. General relativity on the other hand is governed by the symmetry of diffeomorphism invariance, which roughly speaking implies that the laws of physics look the same in an arbitrary coordinate system. 

Finding a consistent quantization of gravity, quantum gravity, has been an ongoing field of research for many years, the most promising theory being string theory. String theory replaces the notion of the most fundamental objects as pointlike to be one-dimensional extended objects, strings. Oscillations of such strings are interpreted as different particles of which it turns out that the graviton, the particle mediating gravity, is present in any consistent string theory. It turns out that extended objects perceive the physical world quite differently than particles which seems to make the theory more well behaved than that for their pointlike counterparts. However, one striking implication of string theory is that it can be consistently formulated only in ten space-time dimensions, which seems to be in contradiction with our everyday experience of only four dimensions. One way of dealing with this is the idea of compactification, where certain spacelike directions are ``small'' compared to the others, such that the effective dimension perceived is lower. Five consistent theories of strings have been developed: Type I, Type IIA/B and two heterotic string theories. These were thought to be five inequivalent theories and the hope was that one of them should describe our physical world. However, it was argued by Witten \cite{WittenDualities1995} that they all were in fact related to each other through dualities. Even more so, their description at low energies can be derived from a theory in 11 dimensions dubbed M-theory. This theory in one dimension higher is known at low energies while the full theory still remains unknown.

\section{Dualities and extended geometries}
A key feature of string theory dualities is that gravitational and non-gravitational degrees of freedom are mixed. However, gravity describes the geometry of space-time which is conceptually different from other fields living in this space-time. In order to to make the dualities manifest symmetries and unify these degrees of freedom we should either generalise the notion of geometry and space-time or in some way change our perspective of gravity. In this thesis we follow the former and try to extend the ordinary notion of geometry to also include non-gravitational degrees of freedom as part of the geometry and space-time. 

Compactification of eleven-dimensional supergravity, the low energy effective field theory of M-theory, on a torus $T^d$ has been known for a long time to possess a rigid global hidden symmetry group $E_{d(d)}$\cite{CREMMER197848,Cremmer:1979up,Cremmer:1997ct}, which turns out to be the continuous version of the discrete duality groups. The goal of exceptional geometry is to make the action of the duality group manifest in the formulation of the theory prior to any compactification. This is done by replacing the structure group $GL(d)$ by $E_{d(d)}\times\mathbb{R}$ as well as to consider an enlargened space-time which transforms in some module of the structure group. To relate these exceptional geometries to a theory based on ordinary geometry a choice of a ``physical subspace'' of space-time is chosen by the so-called section constraint. In other words, by defining a local embedding $GL(d)\hookrightarrow E_{d(d)}\times\mathbb{R}^+$ the ordinary notion of geometry is obtained. Solving the section constraint is crucial for the consistency of the theory and importantly this is done in a $E_{d(d)}\times\mathbb{R}^+$ covariant manner. 

Extended geometry \cite{CederwallPalmkvist2017} is a general formalism for describing a geometry with a structure algebra that is a split real form of a Kac-Moody algebra $\mathfrak{g}\times\mathbb{R}$, which exponentiates to a structure group $G\times \mathbb{R}^+$. One then introduce generalised diffeomorphisms, in analogy to ordinary diffeomorphisms, and demand that the theory is invariant under such transformations. Especially, in order to ensure covariance the algebra of generalised diffemorphisms needs to close which as in the exceptional case demands a section constraint, which is a local embedding of ordinary geometry. The dynamics of the generalised metric can then be formulated in terms of an invariant action and geometrical objects such as covariant derivatives, torsion and a generalised Ricci tensor can be constructed. Note that as special cases this formalism includes both double geometry and exceptional geometry which are the two most physically interesting cases.


Moreover, the construction of extended geometries hints about a deep connection between geometry on one side and certain algebraic structures on the other. Typically such algebraic structures arise from extensions of the structure algebra; important examples are given by adding an even or odd simple root to the Dynkin diagram of $\mathfrak{g}$ such that one obtains a Kac-Moody algebra or a Borcherds superalgebra respectively \cite{Palmkvist2015ExpGeomSuperAlg,CederwallPalmkvistSuperalgebras2015}. Such extensions are often infinite-dimensional and as such encode a rich algebraic structure of the theory. Even larger structures which seems to have even deeper connection to extended geometries are the tensor hierarchy algebra \cite{Palmkvist:2013vya,Carbone:2018njd} and $L_\infty$ algebras \cite{Cederwall:2018aab,Hohm:2017pnh}.


\section{Flux compactifications and generalised geometry}
String theory as well as M-theory is well defined in ten and eleven dimensions respectively and at face value seems far-fetched from our observed reality. However, assuming space-time is divided into an external and an internal space-time offers a solution to this problem. Since the extra dimensions of the compact internal space-time is not observed at low energies the characteristic length scale is supposedly small compared to the length scale with which we currently can probe nature.

Upon on compactification from an external point of view the purely internal degrees of freedom are described by scalar fields. These do e.g.\ determine the geometry of the internal space as well as any internal flux of gauge fields. Typically the simplest compactifications, e.g.\ on torii, produce an excess number of massless such scalar fields, or moduli, that would give rise to a fifth force that is not observed experimentally. To obtain a phenomenologically more interesting theory the internal geometry either needs to be more complicated or one could allow for gauge fields with fluxes along the internal directions. Flux compactifications \cite{Grana:2005jc} have received much interest in the last decade leading to the notion of a large string theory landscape. However, flux compactifications are rather difficult to deal with while the purely geometrical backgrounds can be dealt with more easily using geometrical tools. 

Such flux compactifications it turns out can be studied effectively using (exceptional) generalised geometry. These are geometries obtained by solving the section constraint globally, i.e.\ choosing a physical subspace of coordinates globally, and which by construction has a manifest action of the exceptional groups. Especially this formalism naturally includes the presence of non-trivial gauge fields as a part of the geometry. For this reason there exists appropriate generalisations of the geometrical tools used in the fluxless case. These allows us to study compatifications with fluxes using again a (generalised) geometrical language.\cite{Grana:2009im,Grana:2016dyl,Coimbra:2012af,Ashmore:2015joa}


\section{Outline}
The second chapter provides a short introduction to Lie algebras that will be crucial for the remaining part of the thesis. Moreover, Kac-Moody algebras as well as Borcherds superalgebras are introduced by two examples that are of much importance for the construction of extended geometries. Chapter \ref{chap:StringTheory} introduces the most basic notions of string theory with the goal of motivating the appearance of T-dualities of the bosonic string. Furthermore the field content of supergravity in ten and eleven dimensions is introduced as well as some instructive examples of compactifications. The goal of this is to put into context the somewhat abstract ideas of extended geometries that this thesis introduce. Lastly in this chapter the notion of ``dualities'' in general is discussed in a bit more detail. In chapter \ref{sec:DFT} and \ref{chap:ExceptionalFieldTheory} double field theory and exceptional field theory respectively are introduced. These theories are formulated with a manifest action of the duality groups and one retrieves ordinary supergravity theories as certain solutions of the section constraint. Especially these provide the physically two most interesting examples of extended geometries and, moreover, also show how to include the external space-time which is not included in the pure formulation of extended geometries. The main part of thesis are the formulation of extended geometries introduced in chapter \ref{chap:ExtendedGeometries}. By analysing consistency of generalised diffeomorphisms a complete list of extended geometries without so-called ancillary transformations are found. Furthermore, a pseudo-action invariant under generalised diffeomorphisms encoding dynamics of the generalised metric is formulated. We then continue by introducing geometrical objects such as a connection, torsion and generalised Ricci tensor characterising an extended geometry. The goal of the last chapter is to study flux compactifications with a (exceptional) generalised geometrical formulation. To do this some ordinary geometrical tools used for fluxless compactifications are introduced as well as the basic concepts of generalised geometry. A schematic diagram of the relations between the different theories discussed is presented in \figref{fig:TheGodFather}. 


\begin{figure}
    \BigPicture{1}
    \caption{Schematic diagram of \emph{some} of the relations between M-theory, string theories (ellipses), supergravity theories and extended geometries.}
    \label{fig:TheGodFather}
\end{figure}


%The theory of general relativity allows for black hole solutions, which are an object so dense that nothing can escape its gravitational pull once it enters inside the so called event horizon. These strange objects that have created a lot of controversy during the last century have recently, roughly 100 years after the discovery of general relativity, been discovered directly through gravitational waves emitted from two merging black holes. Indirectly they have been observed and puzzled people for quite some time. At the moment little is known about the fate of the singularity inside a black hole.


